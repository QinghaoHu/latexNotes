\section{Non-linear regression}
\subsection{Types of Regression}
\subsubsection{Linear}
\begin{equation*}
    y = ax + b
\end{equation*}
\begin{center}
    You should use correlation coefficient $r$ to measure the accuracy of regression
\end{center}

\subsubsection{Quadratic}
\begin{equation*}
    y = ax^2 + bx + c
\end{equation*}
\begin{center}
    You should use cofficient of Determination $R^2$ to measure the accuracy
\end{center}

\subsubsection{Power}
\begin{equation*}
    y = ax^b + c
\end{equation*}
\begin{center}
    You should use cofficient of Determination $R^2$ to measure the accuracy
\end{center}

\subsubsection{Exponental}
\begin{gather*}
    y = ab^x + c
\end{gather*}
\begin{center}
    You should use cofficient of Determination $R^2$ to measure the accuracy
\end{center}

\subsection{Natural Constant $\mathrm{e}$}
Similar to $\pi$, $\mathrm{e}$ is an irrational number and its value is approximately 2.71.\\

The inverse of $f(x) = \mathrm{e}^x$ is $f^{-1}(x) = \ln x$

Some software use $f(x) = a\mathrm{e}^x + b$ as Exponential Regression
