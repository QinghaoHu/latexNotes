\section{Causal Relationships}
\begin{redblock}
    A strong correlation between two variables is \textbf{not} enough evidence to say that changes in one variable causes changes in another variable
\end{redblock}

\subsection{Definitions}
\subsubsection{Cause and Effect Relationship} 
A change in the independent variable directly causes a change in the dependent variable

\subsubsection{Reverse cause and effect relationship}
The independent and dependent variables are reversed in the process of determining causality.

\subsubsection{Common cause relationship}
Both variables change as a result of a third common variable

\subsubsection{Accidential relationship}
The correlation between the two variables is based purely on coincidence. 

\subsubsection*{Presumed relationship}
Existed when a correlation does not seem to be accidential even though no cause and effect relationship or common cause is apparent

\subsubsection{Extraneous variable}
A variable is one that is not part of the correlation study but affects the way two variables in this study appear to be related. 
Such variable can lead to a false correlation or a fragmented trend. 