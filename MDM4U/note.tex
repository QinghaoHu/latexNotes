
\documentclass[11pt]{report}

% --- Packages ---
\usepackage[utf8]{inputenc}
\usepackage{amsmath, amssymb, amsthm}
\usepackage{geometry}
\usepackage{fancyhdr}
\usepackage{graphicx}
\usepackage{tikz}
\usepackage{enumitem}
\usepackage{hyperref}
\usepackage{xcolor}
\usepackage{indentfirst}

% --- Page Setup ---
\geometry{margin=1in}
\pagestyle{fancy}
\fancyhf{}
\rhead{Qinghao Hu}
\lhead{MDM4U}
\cfoot{\thepage}

% --- Theorem Environments ---
\newtheorem{theorem}{Theorem}[section]
\newtheorem{definition}[theorem]{Definition}
\newtheorem{lemma}[theorem]{Lemma}
\newtheorem{proposition}[theorem]{Proposition}
\newtheorem{corollary}[theorem]{Corollary}
\theoremstyle{remark}
\newtheorem*{remark}{Remark}
\newtheorem*{example}{Example}

% --- Custom Commands ---
\newcommand{\R}{\mathbb{R}}
\newcommand{\N}{\mathbb{N}}
\newcommand{\Z}{\mathbb{Z}}
\newcommand{\Q}{\mathbb{Q}}
\newcommand{\C}{\mathbb{C}}
\newcommand{\ds}{\displaystyle}

\newcommand{\mypic}[3]{
    \begin{figure}[h!]
        \centering
        \includegraphics[width=#3\textwidth]{#1}
        \caption{#2}
    \end{figure}
}

% --- Title ---
\title{\textbf{Data Management Notes} \\ \large Intro to Statistics}
\author{Qinghao Hu}
\date{\today}


\begin{document}

\maketitle
\tableofcontents
\newpage

\chapter{Unit 1}
\section{Lecture 1}
\subsection{The Fundamental or Multiplicative Counting Principle}
If a task is made up of \textcolor{blue}{\textit{several stages}}, then the number of choices is the \textbf{product} 
of the number of possibilities at each stage. 

\subsection{Additive Counting Principle}
In a situation with actions that cannot occurred at the "same time" than the number
of possibilities is the sum of the possibilities of all the actions

\begin{center}
    !!! Remember, $0! = 1$
\end{center}

\section{Lecture 1.2}
A permutation is an \textcolor{red}{ARRANGEMENT} of items in a definite order.
\[
    nPr = \frac{n!}{(n - r)!}
\]
and 
\[
    nPr = P(n, r)
\]

\section{Like Term Permutations}
The number of permutations of a set of $n$ objects containing $a$ identical objects of one kind, 
$b$ identical objects of a second kind, $c$ identical objects of a third kindand so on is $\frac{n!}{a! * b! * c!}$

\section{Pascal Triangle}
Do what the hell you want to do about Pascal 

\section{Venn Diagrams}
Concepts:
\begin{itemize}
    \item In mathematics, a set is a well-defined collection of \textit{distinct} objects/elements
    \item A Venn diagram is used to organize the (number of) elements in different \textit{set} of data.
    \item Elements that are in set $a$ and set $b$ are described as the intersection of $A$ and $B$. THe notation of $A \cap B$ describes this situation
    \item 
\end{itemize}

\end{document}

% Referenc
% \section{Introduction}
% Write your introduction or overview of the day's topic here.
%
% \section{Main Concepts}
% \begin{definition}
% A function \( f: A \to B \) is said to be \textbf{injective} if for every \( a_1, a_2 \in A \), \( f(a_1) = f(a_2) \Rightarrow a_1 = a_2 \).
% \end{definition}
%
% \begin{theorem}
% Let \( f: \R \to \R \) be a continuous and strictly increasing function. Then \( f \) is injective.
% \end{theorem}
%
% \begin{proof}
% Assume \( f(a) = f(b) \). Since \( f \) is strictly increasing, if \( a < b \), then \( f(a) < f(b) \), contradiction. So \( a = b \).
% \end{proof}
%
% \section{Examples}
% \begin{example}
% Let \( f(x) = x^2 \). Then \( f \) is not injective on \( \R \), but it is injective on \( [0, \infty) \).
% \end{example}
%
% \section{Diagrams}
% \begin{center}
% \begin{tikzpicture}[scale=1]
% \draw[->] (-2,0) -- (2,0) node[right] {$x$};
% \draw[->] (0,-1) -- (0,4) node[above] {$y$};
% \draw[domain=-1.5:1.5, smooth, variable=\x, blue, thick] 
%     plot ({\x}, {\x*\x});
% \node at (1.2,3.2) {$y = x^2$};
% \end{tikzpicture}
% \end{center}
%
% \section{Summary}
% Summarize what you learned today.
