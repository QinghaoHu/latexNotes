\section{Geometric Distributions}
\subsection{Example}
\begin{example}
Here is an example from our teacher's note:
    \begin{figure}[!h]
        \centering
        \includegraphics[width=1.0\textwidth]{pictures/5.4.1.png}
    \end{figure}

\end{example}

\subsection{Definitions}
\begin{definition}
    The distribution of a discrete random variable $X$ is a geometric probability distribution if:
    \begin{enumerate}
        \item each trail of this experiment is \textbf{INDEPENDENT}
        \item the outcome of each trial can be categorized into \textbf{Successful} or failed outcomes
        \item the random varaible $X$ represents the number of failed outcomes \textbf{before} a success occurs
    \end{enumerate}
\end{definition}

\noindent\hrulefill

\begin{theorem}
    If the distribution of X is a geometric probability distribution then:
    \begin{gather}
        P(x) = (1 - p)^x p\\
        E(x) = \frac{1 - p}{p}\\
        \sigma = \frac{\sqrt{1 - p}}{p}
    \end{gather}
\end{theorem}
\noindent\hrulefill