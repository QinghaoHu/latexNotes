\section{Hypergeometric Distributions}
\subsection{Example}
\begin{example}
    
    An example from Teacher's note!
    \begin{figure}[!h]
        \centering
        \includegraphics[width=1.0\textwidth]{pictures/5.5.1.png}
    \end{figure}

\end{example}

\subsection{Definitions}
\begin{definition}
    The distribution of a discrete random variable X is a hypergeometric probability distribution if:
    \begin{enumerate}
        \item the probability experiment is repeated a \textbf{fixed} number of times
        \item the outcome of each trial can be categorized into \textbf{success} or failed outcomes
        \item the random varaible $X$ represents the number of \textbf{successes}
        \item each trial of experiment is \textbf{dependant}
    \end{enumerate}
\end{definition}

\noindent\hrulefill

\begin{theorem}
    If the distribution of $X$ is hypergeometric with $n$ dependent trials, and $a$ represents the number of successful
outcomes initially among a total of $N$ possible outcomes in the beginning, then
\begin{gather}
    P(x) = \frac{
        \binom{a}{x} \binom{N - a}{n - x}
    }{
        \binom{N}{n}
    } \\
    E(x) = \frac{aN}{n}\\
    \sigma = \sqrt{n(\frac{a}{N})(\frac{N - a}{N})(\frac{N - n}{N - 1})}
\end{gather}
\end{theorem}