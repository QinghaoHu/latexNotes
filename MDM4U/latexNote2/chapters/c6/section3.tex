\section{Approximate Binomial}
At certain situation, we can use a \underline{normal distribution} to approximate a Binomial distribution
\begin{theorem}
    If $X$ is a binomial random distribution of $n$ independent trials, each with probability of success $p$,
    and if
    \begin{gather*}
        np > 5 \\
        n(1 - p) > 5
    \end{gather*}
    then the binomial random variable can be approximated by a normal distribution with 
    \begin{gather*}
        \mu = np\\
        \sigma = \sqrt{np(1 - p)}
    \end{gather*}
\end{theorem}

\begin{theorem}
    The distribution of a discrete random variable $X$ is a binomial probability distribution if:
    \begin{enumerate}
        \item The probability experiment is repeated a FIXED number of times (In the example, $n = 3$)
        \item The outcome of each trial can be CATEGORIZED into successful or failed outcomes
        \item The random varialbe $X$ represents the number of \underline{successes}
        \item Each trial of the experiment is \underline{Independent}
    \end{enumerate}
\end{theorem}