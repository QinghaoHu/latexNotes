\section{Sampling Distributions}
\begin{theorem}
    If a population has a mean of $\mu$ and a standard deviation of $\sigma$, then the means obtained from 
    repeatedly sampling the population are normally distributed with a mean of 
    \begin{equation*}
        \mu_{\bar{x}} = \mu
    \end{equation*}
    and a standard deviation of
    \begin{equation*}
        \sigma_{\bar{x}} = \frac{\sigma}{\sqrt{n}}
    \end{equation*}
    where $n$ is the size of each repeated sample $n \geq 30$ for population data that is not normally distributed
\end{theorem}

\begin{definition}
    [Point Estimation] The process of using a sample mean $\bar{x}$ to estimate the population mean $\mu$.
    The downside to point estimates is that we have no way of knowing if the sample statistics are actually 
    close to the population summary value. It could be that, because of variability, the sample mean is 'way off' 
    from the population mean. For that reason, interval estimation is more preferred. 
\end{definition}

\begin{definition}
    [Confidence Interval] An interval estimate accompanied by a confidence level. The confidence level indicates the percent of 
    confidence that the given interval size will, in the "long run", capture the population summary value. 
\end{definition}