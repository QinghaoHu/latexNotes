\chapter{One Variable Statistics}
\section{Variables and Data}
\subsection{Definitions}

\tbf{Categorical} variables represent data that are generally grouped into categories, and are also known as qualitative variables\\

\tbf{Ordinary} variables are categorical variables whose data has a natural order but the difference between values cannot be determined
or is not meaningful\\

\tbf{Nominal} variables describe names, labels, or categories that have no natural order\\

\tbf{Quantitative} variables describe data values that are numerical, and are also known as numerical variables\\

\tbf{Continuous} variables are numerical variables which can assume an infinite number of values in a given interval\\

\tbf{Descrete} variables are numerical variables that only take on a finite number of possible values in a given interval\\

\tbf{Primary} data are data that are collected by the statisticians who are analyzing the data, from first-hand sources such as surveys or 
experiments\\

\tbf{Secondary} data are data that the statisticians who are analyzing the data did not participate in the first hand data collection 
process (ie Surveys or experiments)\\

\tbf{Microdata} contains records for each individual surveyed\\

\tbf{Aggregate} or summary data are data that are combined or summarized in such a way that the individual microdata can no longer be 
determined.\\

Data gathered from a \tbf{cross} sectional study considers individuals from different groups at the same time\\

Data gathered from a \tbf{longitudinal} study considers how the characteristics of a specific sample changes over time\\

An \tbf{index} is a continous variable such that it is an arbitrarity defined number that provides a measure of scale. 
It is used to related the values of a variable to a base level\\

The \tbf{consumer} price index, CPI, provides a broad picture of the cost of living in Canada by comparing the cost of a wide variety
of consumer goods, such as food, clothing, fuel, heating cost, transportation, shelter, and recreation\\

Health officials use the \tbf{body} mass index to determine whether a person is overweight. The BMI is calculated by dividing a person's mass
in kilograms by the square of their height in meters\\

