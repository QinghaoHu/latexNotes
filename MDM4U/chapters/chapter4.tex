\chapter{Two variable Statistics}
\section{Lecture 1: Graphs}

\subsection{Definitions}

\subsubsection*{Relation}
In a \textbf{relation}, the variable that you need to first is called the \textbf{independent variable}. Its 
value determine the value of the \textbf{dependent} variable. On the scatter plot, the independent variable 
in located on the \textbf{horizontal} axis and the dependent variable is located on the \textbf{vertical} axis.
The title of the graph should be \tit{dependent variable vs independent variable}

\subsubsection*{Scatter Plot}
A \textbf{scatter plot} is used to determine if a correlation exists between two \textbf{numerical} variables.
An \textbf{outlier} is a data point that does not fit the pattern of the other data. 

\subsubsection*{Line of best fit}
A \textbf{line of the best fit} can be used to model the data on a scatter plot whose points follow the trend of 
a line. A \textbf{curve} of best fit can be used to model the data on a scatter plot whose points follow the trend
of a curve. The line/curve should be solid if the data is \textbf{continuous} and dashed/dotted if the data is \textbf{discrete}

\subsubsection{Two variable graphed}
Using a scatter plot has a \textbf{positive} correlation if the trend of the data points increase from left to right. 
The two variables graphed using a scatter plot has a \textbf{negative} correlation if the trend of the data points
decrease from left to right. 

\subsubsection{Correlation}
The correlation between two variables is strong if the points on the scatter plot follow a line or a curve very closely. 
The correlation between two variables is \textbf{moderate} if the points on the scatter plot nearly follow a line or curve.
The correlation between two variables is \textbf{weak} if the points on the scatter plot are dispersed more widely, but 
still show a recognizable trend.\\

Two variables graphed on a scatter plot shows \textbf{no} correlation if the points are so scattered that no trend is 
discernible. 

\subsubsection*{Interpolate}
To \textbf{interpolate} means to estimate values lying between given data. To interpolate from a graph means to estimate
coordinates of points between those that are plotted. 

\subsubsection*{extrapolate}
To \textbf{extrapolate} means to estimate values lying outside the given range of data. To extrapolate 
from a graph means to estimate coordinates of points beyond those that are plotted. 

\subsubsection{Contingency}
A \textbf{contingency} table shows the frequency or percentage distribution of two categorical variables. 

\section{Linear Correlation}
\subsection{Why Scatter plot?}
There are few advantages of a scatter plot:
\begin{itemize}
    \item Correlations
    \item Predictions
    \item Positive/negative
    \item Strong/weak
\end{itemize}

\subsection{Some boring Definitions}
\subsubsection{Linear Relationship}
A \textbf{linear} relationship is one in which a \textbf{change} in the independent (explanatory) variable corresponds a proportional change in the dependent
(response) variable.\\

We can use table to calculate the correlation of two variables:
\begin{figure}[h!]
    \centering
    \includegraphics[width=0.7\textwidth]{pictures/4.2.1.png}
\end{figure}

\begin{gather}
    \bar{x} = \frac{\sum x}{n}\\
    \bar{y} = \frac{\sum y}{n}
\end{gather}
In order to calculate the correlation coefficient of x and y, we need to get the sample standard deviation for x and y.
\begin{gather}
    s_x = \sqrt{\frac{\sum (x - \bar{x})}{n - 1}}\\
    s_y = \sqrt{\frac{\sum (y - \bar{y})}{n - 1}}
\end{gather}

Then, we need to calculate the covariance of $x$ and $y$:
\begin{gather}
    s_{xy} = \frac{\sum (x_i - \bar{x}) * (y_i - \bar{y})}{n - 1}
\end{gather}

Finally, the correlation coefficient is defined by this:
\begin{gather}
    r = \frac{s_{xy}}{s_x * s_y}
\end{gather}

\begin{figure}[h!]
    \centering
    \includegraphics[width=0.7\textwidth]{pictures/4.2.2.png}
\end{figure}