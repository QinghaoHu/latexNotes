
\documentclass[11pt]{report}

% --- Packages ---
\usepackage[utf8]{inputenc}
\usepackage{amsmath, amssymb, amsthm}
\usepackage{geometry}
\usepackage{fancyhdr}
\usepackage{graphicx}
\usepackage{tikz}
\usepackage{enumitem}
\usepackage{hyperref}
\usepackage{xcolor}
\usepackage{indentfirst}

% --- Page Setup ---
\geometry{margin=1in}
\pagestyle{fancy}
\fancyhf{}
\rhead{Hu}
\lhead{Intro to Math Proofs}
\cfoot{\thepage}

% --- Theorem Environments ---
\newtheorem{theorem}{Theorem}[section]
\newtheorem{definition}[theorem]{Definition}
\newtheorem{lemma}[theorem]{Lemma}
\newtheorem{proposition}[theorem]{Proposition}
\newtheorem{corollary}[theorem]{Corollary}
\theoremstyle{remark}
\newtheorem*{remark}{Remark}
\newtheorem*{example}{Example}

% --- Custom Commands ---
\newcommand{\R}{\mathbb{R}}
\newcommand{\N}{\mathbb{N}}
\newcommand{\Z}{\mathbb{Z}}
\newcommand{\Q}{\mathbb{Q}}
\newcommand{\C}{\mathbb{C}}
\newcommand{\ds}{\displaystyle}

% --- Title ---
\title{\textbf{Introduction to Problem Solving and Proof} \\ \large Introduction To Math Proofs}
\author{Qinghao Hu}
\date{\today}

\begin{document}

\maketitle
\newpage
\tableofcontents
\newpage

%section 1
\chapter{Proof}
\section{What is a Proof}
\begin{example}
	Let's proof that $\sqrt{2}$ is irrational\\
	\indent\textit{To start, we need to know what is the definition of irrational number} \\
	\indent Definition: An irrational real number cannot be expressed in the form $\frac{m}{n}$, where $n$ and $m$ are integers \\
	\begin{proof}
		Assume, $\sqrt{2}$ is rational \\
		\begin{equation}
			\sqrt{2} = {m \over n} \quad \text{(m and n are integers and $\frac{m}{n}$ is a reduced fraction)}
		\end{equation}
		We can square both side using the principle of Algebra \\
		\begin{equation}
			2 = \frac{m^2}{n^2} 
		\end{equation}
		\begin{equation}
			2n^2 = m^2
		\end{equation}
		\begin{gather}
		% $\because$ Definition: A real number, $N$, is Even if it can be written as $N = 2k$ where $k$ is an integer.\\
		% $\therefore m^2$ must be even \\
		% $\because m^2$ must be even \\
		% $\therefore m$ must be even \\
		\because \quad \text{Definition: A real number $N$ is even if it can be written as $N = 2k$, where $k$ is an integer}\\
		\therefore m^2 \quad\textit{is even} \\
		\therefore m \quad\textit{is even}
		\end{gather}
		According to the definition, assume $m = 2k$, where $k \in \mathbb{I}$ \\
		\\
		Sub $m = 2k$ into equation (3)
		\begin{gather}
			2n^2 = (2k)^2 \\
			n^2 = 2k^2 \\
			\text{$n$ is even}\\
			\because \text{According to (9) and (6), $n$ and $m$ are both even} \\
			\therefore \text{$m$ and $n$ have a common factor of 2} \\
			\therefore \text{$\frac{m}{n}$ is not an reduced fraction}\\
			\text{This violate our original assumpution in (1)}
		\end{gather}
		Therefore, $\sqrt{2}$ is not rational, $\sqrt{2}$ must be irrational
	\end{proof}
	\end{example}

	\section{Logical Rules}
	\subsection{Modus Ponens}
	\begin{center}

		\textbf{Definition: }If $p$ is ture and $p$ implies $q$, then $q$ is ture \\
		\textbf{Logical notation: } $p, p\to q\therefore q$ \\
		\textbf{Example: }$p$ = "It is raining", $q$ = "The ground is wet"\\
		\textbf{Given: }"it is raining" and "It is raining implies the ground is wet"\\
		\textbf{Conclusion: }"The ground is wet"\\
		
	\end{center}

	\subsection{Modus Tollens}
	\begin{center}
		\textbf{Definition: }If not $q$ is ture and $p$ implies $q$, then not $p$ is true \\
		\textbf{Logical notation: } $-q, p \to q \therefore -q$ \\
		\textbf{Example: } $p$ = "It is raining", $q$ = "The ground is wet"\\
		\textbf{Given: } "it is raining" and "It is raining implies the ground is wet"\\
		\textbf{Conclusion: } "It is not raining"\\
	\end{center}


	\subsection{Hypothetical Syllogism}
	\begin{center}
		\textbf{Definition: }If p implies q and q implies r, then p implies r. \\
		\textbf{Logical notation: } $(p \to q), (q \to r) \therefore (p \to r)$ \\
		\textbf{Example: } $p$ = "It is raining", $q$ = "The ground is wet", $r$ = "People use umbrellas"\\
		\textbf{Given: } "It is raining implies the ground is wet" and "The ground is wet implies people use umbrellas"\\
		\textbf{Conclusion: } "It is not raining implies people use umbrellas"\\
	\end{center}


	\subsection{Disjunctive Syllogism}
	\begin{center}
		\textbf{Definition: }If not p is true and p or q is true, then q is true. \\
		\textbf{Logical notation: } $-p, (p \vee q)\therefore q$ \\
		\textbf{Example: } $p$ = "It is raining", $q$ = "I will stay indoors"\\
		\textbf{Given: } "It is not raining" and "It is raining or I will stay indoors"\\
		\textbf{Conclusion: } "I will stay indoors"\\
	\end{center}


	\subsection{Additon}
	\begin{center}
		\textbf{Definition: }If p is true, then p or q is true \\
		\textbf{Logical notation: } $p\therefore (p\wedge q) $ \\
		\textbf{Example: } $p$ = "It is raining", $q$ = "I will go for a run"\\
		\textbf{Given: } "It is raining"\\
		\textbf{Conclusion: } "It is raining or I will go for a run"\\
	\end{center}


	\subsection{Simplification}
	\begin{center}
		\textbf{Definition: }If p and q are true, then p is true\\
		\textbf{Logical notation: } $(p\wedge q) \therefore p$ \\
		\textbf{Example: } $p$ = "It is raining", $q$ = "The ground is wet"\\
		\textbf{Given: } "It is raining and The ground is wet"\\
		\textbf{Conclusion: } "It is raining"\\
	\end{center}


	\subsection{Conjunction}
	\begin{center}
		\textbf{Definition: }If $p$ is true and $q$ is true, then $p$ and $q$ are true.\\
		\textbf{Logical notation: } $p, q\therefore (p \wedge q)$ \\
		\textbf{Example: } $p$ = "It is raining", $q$ = "The ground is wet"\\
		\textbf{Given: } "It is raining", "The ground is wet"\\
		\textbf{Conclusion: } "It is raining and The ground is wet"\\
	\end{center} 

	%Section 3, sets
	\section{Mathematical Sets}
	\begin{center}
		\textbf{(Collection of Objects)}\\
		$\{\}$ \\
		$\{$All Triangles$\}$ \\
		$\{3, 6, 11, 117\}$, $\{$Real Numbers$\}$
	\end{center}

	\begin{center}
	$\{$What's in the set | The condition to be in the set$\}$ \\
	Example: $\{x$ | $x$ is even and $x > 0\}$
	\end{center}

	\subsection{Important Sets}
	\begin{center}
		$\mathbb{N}$ - The Natural Numbers(Counting Numbers): $\{1, 2, 3, \cdots \}$\\
		$\mathbb{W}$ - The Whole Numbers: $\{0,1, 2, 3, \cdots \}$\\
		$\mathbb{Z}$ - The Integers: $\{\cdots, -3, -2, -1, 0,1, 2, 3, \cdots \}$\\
		$\mathbb{Q}$ - The Rational Numbers: $\{\frac{p}{q} | p, q \in \mathbb{Z}, q \neq 0\}$\\

		$\mathbb{I}$ - The Irrational Numbers: $\{x | x \notin \mathbb{Q}, x \in \mathbb{R} \}$\\

		$\mathbb{R}$ - The Real Numbers: $\{x \}$ = $\{x | x \in \mathbb{Q} $ or $ x \in \mathbb{I}\}$ \\

		$\mathbb{C}$ - The Complex (Imaginary) Numbers: $\{a + bi | a, b \in \mathbb{R}, i = \sqrt{-1}\}$\\

		$\o$ - Empty Set: $\{\}$\\
		Sets do not care about orders, or duplicates.
	\end{center}

	\subsection{Relationship of sets}
	\begin{center}
    Set A and B are equal ($A = B$). \par
    $A \subseteq B$, $B \subseteq A$. \par
    A is a subset of B if every element of A is also an element of B: \\
    $(a \in A \rightarrow a \in B)$. \par
    The Power Set of A, $P(A)$, is the set of all possible subsets of A. \par
    The Complement, $A^C$ or $A'$, of a set A is the set of all elements in the universal set that are NOT elements of A. \par
    The Union of two sets, $A \cup B$, is the set containing all the elements from either A or B.\\
	Theorem: if $A \subseteq B, B \subseteq C$, then $A \subseteq C$\\
	\end{center}

	\section{Quantifiers}

	\subsection{Universal Quantify}
	\begin{center}
		"For all"\\
		$\forall$\\
		Example: $\forall y \in \mathbb{R}, y^2 \geqslant 0$\\
	\end{center}
	\subsection{Essential Quantify}
	\begin{center}
		"There Exists"\\
		$\exists$\\
		Example: $\exists x$ such that $x + 3 = 5$\\
	\end{center}
	\subsection{Negations}
	\begin{center}
		$\neg$ \\ 
		This is a dog\\
		This is not a dog\\
		Examples: \\
		$\neg$($A$ or $B$) = $\neg A$ and $\neg B$ \\
		$\neg$($A$ and $B$) = $\neg A$ or $\neg B$ \\
		$\neg$($A$ $\Rightarrow$ $B$) = $A$ and $\neg B$ \\
		$\neg$($\forall x$, $y$) = $\exists x$ such that $\neg y$ \\

	\end{center}


	%Section 5
	\section{Proofs}
	\subsection{Direct Proofs}
	\begin{center}
		Assumption\\ 
		$\Rightarrow$ something \\ 
		$\Rightarrow \cdots$ \\
		$\Rightarrow$ conclusion \\
	\end{center}

	\begin{example}
		Prove: The sum of two Even integers equals an even integer\\
		\begin{proof}
			\begin{gather}
				\text{Let }x, y \in \mathbb{Z} \quad \text{Assume $x$ and $y$ are both even.} \\ 
				\Rightarrow x = 2a, y = 2b, (a, b \in \mathbb{Z})
				\\ \Rightarrow x + y = 2a + 2b \\ 
				\Rightarrow x + y = 2(a + b) \\
				\text{Given that each Even number N can be present in this form $N = 2k$}\\
				\Rightarrow \therefore \text{The sum of two Even integers equals an even integer}
			\end{gather}
		\end{proof}
	\end{example}
	
	\subsection{Contrapositive proof}
	\begin{center}
		$p$ $\rightarrow$ $q$ $\equiv$ $\neg q$  $\rightarrow$ $\neg p$\\
	\end{center}
	\begin{example}
		Theorem: if $x$ is an irrational number, then $\frac{1}{x}$ is also an irrational number.
	\end{example}
	$\text{we want to show: }\neg (\frac{1}{x} \text{ is not irrational}) \Rightarrow \neg (x \text{ is irrational}) $

	\begin{gather}
		x \in \mathbb{R} \setminus \mathbb{Q} \Rightarrow x \neq \frac{p}{q}, (p, q \in \mathbb{Z}) \\ 
		\text{Assume } \frac{1}{x} \text{ is Not Irrational}\\
		\Rightarrow \frac{1}{x} \text{ is rational, } (\frac{1}{x} \in \mathbb{Q})\\
		\Rightarrow \frac{1}{x} = \frac{m}{n}, (m, n \in \Z)\\
		\Rightarrow x = \frac{n}{m} \\
		\Rightarrow \text{x is rational}
	\end{gather}

	\subsection{Two Way Proof}
	Also known as (Two way Proof) 
	\begin{example}
		Theorem: A whole number is divisible by 9 iff the sum of its digits is divisible by 9.
		\begin{proof}
			\begin{gather}
				x \in \mathbb{W} = \{0, 1, 2, 3, \cdots\}\\ 
				x \text{ has digits }, a_n, a_{n - 1}, \cdots, a_2, a_1, a_0 \\
				x = 10^{n}a_n + 10^{n - 1}a_{n - 1} + \cdots + 100a_2 + 10a_1 + a_0\\
				\text{Assume } x \text{ is divisible by 9.}\\
				10^{n}a_n + 10^{n - 1}a_{n - 1} + \cdots + 100a_2 + 10a_1 + a_0 \text{(divisible by 9)}\\
				- (999\dots 99a_{n} + \dots + 99a_2 + 9a_1) \text{(is divisible by 9)} \\ 
				= a_n + a_{n - 1} + \dots  + a_2 + a_1 + a_0 \text{ (is divisible by 9)}\\ \\
				\text{Assume } a_n + a_{n - 1} + a_{n - 2} + \cdots + a_2 + a_1 + a_0  \text{ is divisible by 9} \\
				x = a_n + a_{n - 1} + a_{n - 2} + \cdots + a_2 + a_1 + a_0  \\
				+ ((999\dots 99)a_n + \dots + 99a_2 + 9a_1) \text{ (is divisible by 9)} \\
				\Rightarrow x \text{ is divisible by 9}
			\end{gather}
		\end{proof}
	\end{example}
	
	\subsection{Proof by Contradiction}
	Assume the opposite of what we want to prove, then show a contradiction.
	\begin{example}
		Theorem: $\sqrt{3}$ is irrational.
		\begin{proof}
			\begin{gather}
				\text{Assume $\sqrt{3} \in \mathbb{Q}$} \\
				\Rightarrow \sqrt{3} = \frac{m}{n} \text{ is a reduced fraction,($n, m \in \mathbb{Z}$ and $n \neq 0$)}\\
				\Rightarrow 3 = \frac{m^2}{n^2}\\
				\Rightarrow 3*n^2 = m^2 \\
				\Rightarrow m^2 \text{ is a multiple of $3$ and $m$ is a multiple of $3$} \\
				\Rightarrow m = 3k, (k \in \Z) \\
				\Rightarrow 3n^2 = (3k)^2 \\
				\Rightarrow 3n^2 = 9k^2 \\
				\Rightarrow 3n^2 = 3k^2\\
				\Rightarrow n^2 \text{ is a multiple of $3$ and $n$ is a multiple of $3$}\\
				\Rightarrow \frac{m}{n} \text{ is NOT in lowest terms}
			\end{gather}
		\end{proof}
	\end{example}
	
	\newpage
	\chapter{Problem Solving}
	\section{Problem Solving Techniques}
	\subsection{Intro}

	\begin{example}
		Problem 6\\
		\textcolor{blue}{Connie has a number of gold bars, all of different masses. She gives the 24 lightest bars, which represent 45%
		of the total mass, to Brennan. She gives the 13 heaviest bars, which represent 26% of the total mass. She gives
		rest of the bars to Blair. How many bars did Blair receive.}\\
		
	\end{example}
	
	\subsection{Trial and error}
	This method not always works in the question. You should avoid this question for big number problem

\end{document}

% Referenc
% \section{Introduction}
% Write your introduction or overview of the day's topic here.
%
% \section{Main Concepts}
% \begin{definition}
% A function \( f: A \to B \) is said to be \textbf{injective} if for every \( a_1, a_2 \in A \), \( f(a_1) = f(a_2) \Rightarrow a_1 = a_2 \).
% \end{definition}
%
% \begin{theorem}
% Let \( f: \R \to \R \) be a continuous and strictly increasing function. Then \( f \) is injective.
% \end{theorem}
%
% \begin{proof}
% Assume \( f(a) = f(b) \). Since \( f \) is strictly increasing, if \( a < b \), then \( f(a) < f(b) \), contradiction. So \( a = b \).
% \end{proof}
%
% \section{Examples}
% \begin{example}
% Let \( f(x) = x^2 \). Then \( f \) is not injective on \( \R \), but it is injective on \( [0, \infty) \).
% \end{example}
%
% \section{Diagrams}
% \begin{center}
% \begin{tikzpicture}[scale=1]
% \draw[->] (-2,0) -- (2,0) node[right] {$x$};
% \draw[->] (0,-1) -- (0,4) node[above] {$y$};
% \draw[domain=-1.5:1.5, smooth, variable=\x, blue, thick] 
%     plot ({\x}, {\x*\x});
% \node at (1.2,3.2) {$y = x^2$};
% \end{tikzpicture}
% \end{center}
%
% \section{Summary}
% Summarize what you learned today.
