\noindent \textbf{Unit 1: Limits and the Derivative.}
The transition from average to instantaneous rates of change. Introduction to the limit operator, continuity, and the formal definition of the derivative from first principles.

\vspace{0.8em}

\noindent \textbf{Unit 2: Rules of Differentiation.}
Development of the algebraic machinery for differentiation. The Power, Product, Quotient, and Chain rules are applied to polynomials and rational functions to bypass the limit definition.

\vspace{0.8em}

\noindent \textbf{Unit 3: Derivatives of Transcendental Functions.}
Extension of differential calculus to non-algebraic functions. Analysis of the derivatives of sinusoidal, exponential, and logarithmic functions, including applications to composite functions.

\vspace{0.8em}

\noindent \textbf{Unit 4: Applications of Derivatives.}
Utilization of the derivative to solve real-world problems. Topics include velocity and acceleration in kinematics, related rates of change, and mathematical optimization problems.

\vspace{0.8em}

\noindent \textbf{Unit 5: Curve Sketching.}
A systematic approach to analyzing function behavior. Using the first and second derivatives to determine intervals of increase/decrease, concavity, points of inflection, and asymptotic behavior.

\vspace{0.8em}

\noindent \textbf{Unit 6: Introduction to Vectors.}
The shift from scalar to vector quantities. Geometric and algebraic representations of vectors in $\mathbb{R}^2$ and $\mathbb{R}^3$, including operations such as vector addition, scalar multiplication, and the dot and cross products.

\vspace{0.8em}

\noindent \textbf{Unit 7: Lines and Planes in $\mathbb{R}^3$.}
Analytic geometry in three-dimensional space. Derivation of vector, parametric, and Cartesian equations for lines and planes, and the analysis of their intersections and distances.