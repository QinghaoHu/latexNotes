\subsection{Discontinuity and Limits}
\subsubsection{Jump Discontinuity}
\begin{example}
    Evaluate the limit of $f(x)$ at $x = 7$
    \[
        f(x) =  \left\{\begin{matrix}
                    \sqrt{x - 3}, x \geq 7\\
                    x - 4, x < 7
        \end{matrix}\right.
    \]
    If we approach from the left side
    \[
        \lim_{x \to 7^-} (x - 4) = 3
    \]
    If we approach from the right side
    \[
        \lim_{x \to 7^+}(\sqrt{x - 3}) = 2
    \]
    \begin{gather*}
        \because \lim_{x \to 7^-} f(x) \neq \lim_{x \to 7^+} f(x)\\
        \therefore \lim_{x \to 7} f(x) \text{ does not exist. }
    \end{gather*}
    This is called a jump discontinuity. 
\end{example}

\subsubsection{Infinite Discontinuity}
\begin{example}
    Evaluate the limit of $f(x)$ at $x = 2$
    \begin{equation*}
        f(x) = \frac{x + 3}{x - 2}
    \end{equation*}
    If we approach from the left side
    \[
        \lim_{x \to 2^-} (\frac{x + 3}{x - 2}) = \infty
    \]
    If we approach from the right side
    \[
        \lim_{x \to 2^+}(\frac{x + 3}{x - 2}) = -\infty
    \]
    \begin{gather*}
        \because \lim_{x \to 2^-} f(x) \neq \lim_{x \to 2^+} f(x)\\
        \therefore \lim_{x \to 2} f(x) \text{ does not exist. }
    \end{gather*}
\end{example}

\subsubsection{Removable Discontinuity}
\begin{example}
    Evaluate this limit.
    \begin{equation*}
        \lim_{x \to 5}\frac{2x^2 - 7x - 15}{x-5}
    \end{equation*}
    \begin{gather*}
        \begin{align}
            &= \lim_{x \to 5}\frac{(2x + 3)(x - 5)}{(x - 5)}  \\ 
            &= \lim_{x \to 5} 2x + 3 \\
            &= 13 
        \end{align}
    \end{gather*}
    However, this function is not defined at $x = 5$\\
    So there is a Removable discontinuity
\end{example}

\begin{example}
    Evaluate the following limit
    \begin{equation*}
        \lim_{x \to 4} \frac{1}{(x + 4)^2}
    \end{equation*}
    If you approach this limit from both side, you will figure it out that this limit equals to $\infty$
    This is not a "limit" because simply $\infty \notin \R$ 
\end{example}

\begin{definition}
    [Indeterminate form] is an expression in which generate values such as $\tfrac{0}{0}$ or $\frac{\infty}{\infty}$.
    Certain limits cannot be evaluated by direct substitution since we may end up with one of these indeterminate form.  
\end{definition}
\begin{remark}
    See the teacher's notes for more details!!!
\end{remark}