\subsection{Measuring Rates of Change Using a Defined Relation}
\subsubsection{Difference Quotient}
\begin{definition}
    [Difference Quotient] a measure of the average rate of change of the function over an interval. 
    In other words, it is a simplified expression for the slope of a secant. 
    It can be expressed as
    \[
        m = \frac{f(x + h) - f(x)}{h}
    \]
    where $h \neq 0$
\end{definition}
Let's proof the difference Quotient
\begin{proof}
    We have the formula of the slope: 
    \[
        m = \frac{y_2 - y_1}{x_2 - x_1}
    \]
    Assume we have two points: $(x, f(x))$ and $(x + h, f(x + h))$, sub these two points into the formula of slope.
    \begin{gather*}
        m = \frac{f(x + h) - f(x)}{(x + h) - x}\\
        m = \frac{f(x + h) - f(x)}{h}
    \end{gather*}
\end{proof}

Let's look at an example. 
\begin{example}
    Determine a general expression for the slope given the function $f(x) = -x^2 + 4x + 5$\\
    \begin{gather*}
        m = \frac{f(x + h) - f(x)}{h} \\
        m = \frac{
            -(x + h)^2 + 4(x + h) + 5 - (-x^2 + 4x + 5)
        }{
            h
        }\\
        m = \frac{
            -x^2 - 2xh - h^2 + 4x + 4h + 5 + x^2 -4x - 5
        }{
            h
        }\\
        m = \frac{
            -2xh - h^2 + 4h 
        }{h}\\
        m = -2x + 4 - h, h \neq 0
    \end{gather*}
\end{example}

\subsubsection{Approximating the slope of the tangent}
Recall:
\begin{itemize}
    \item The slope of a secant line will give the average rate of change of the 'y' values over a givern interval of 'x'.
    \item The slope of a tangent line will give the instantaneous rate of change of the 'y' value at a givern value of 'x'.
\end{itemize}

\noindent For difference quotient 
\[
    \frac{f(x + h) - f(x)}{h}
\]
if $h$ is extremely small, we can approximate the slope of the tangent line. \\

In later sections, we will discuss about principle of limits. At that time, you will have a better understanding about today's topic. 