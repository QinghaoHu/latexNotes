\subsection{The Concept of Limit}
\begin{definition} [Limits]
    Assume we have function $f(x)$, $x$, and 
    \[ \lim_{x \to a} f(x) = L \]
    Using the $\delta-\epsilon$ language:\\
    \[
        \forall \epsilon > 0, \exists \delta > 0 \text{ such that } 0 < \left|a - x\right| < \delta \implies 
        \left|f(x) - L\right| < \epsilon
    \]
    Remember, $f(a)$ does not have to be defined. 


    For limts, we are solving for the expected value instead of defined value. 
    When the following conditions are met, the limits exists
    \begin{itemize}
        \item \[\lim_{x \to a^{-}} f(x) \text{ must exist }\] 
        \item \[\lim_{x \to a^{+}} f(x) \text{ must exist }\]
        \item \[\lim_{x \to a^{-}} f(x) = \lim_{x \to a^{+}} f(x)\]
    \end{itemize}
\end{definition}

\begin{definition}
    [Continuity] a function which there are no "breaks" or "gaps" in the function. 
    
    These conditions must be met
    \begin{enumerate}
        \item \[\lim_{x \to a} f(x) \text{ exist }\]
        \item \[f(a) \text{ must be defined }\]
        \item \[\lim_{x \to a} f(x) = f(a)\]
    \end{enumerate}
\end{definition}