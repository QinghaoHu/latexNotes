\section{Trial and Error}

The first approach that we will look at is trial and error. Trial and error is sometimes a good place to start, but rarely a good place to end. Sometimes it will give us a sense of what's going on, but it has major limitations.
\\

Trial and error can be more useful in multiple-choice rather than full solution problems. It might get us to an answer but won't normally get us to a solution. 
\\

Trial and error works when we're asked to find an answer but rarely when we're asked to find all answers. Now when you see the phrase “find all” or “determine all,” this always implies “and justify why there are no more.” Trial and error rarely can do this for us.
\\

Trial and error is rarely helpful when the answers are not small integers. We're unlikely to try $4+\sqrt[3]{4}$ by trial and error. 
\\

Trial and error requires thinking and adjusting. I like the expression trial and error better than the expression guess and check, which to me implies no course correction along the way.
\\

For some people, trial and error is their default problem-solving strategy, and they don't have a lot of other tools. Hopefully, over the next several lessons, we'll develop some more of those tools for ourselves.