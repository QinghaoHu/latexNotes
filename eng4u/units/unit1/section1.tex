\section{Social Issues}
\begin{definition}
    [Social Issue]
    A state of affairs that negatively affects the personal or social lives of individuals or the 
    well-being of communities or larger groups within a society and about which there is usually public disagreement 
    as to its nature, causes or solution. 
\end{definition}

In the class, we are asked to write a Social Criticism based on a frictional media text. 
\begin{definition}
    [Social Criticism]
    a form of academic or journalistic criticism focusing on social issues in contemporary society, 
    in respect to perceived injustices and power relations in general. 
\end{definition}

\begin{definition}
    [CRAAP test]
    When you pick sources, you should consider 5 different aspects of the source:
    \begin{itemize}
        \item [\&] Currency: \textit{is it recent enough for the topic}
        \item [\&] Relevance: \textit{Does it actually help your research question}
        \item [\&] Authority: \textit{Who wrote it?}
        \item [\&] Accuracy: \textit{Is it based on data, evidence, citations?}
        \item [\&] Purpose: \textit{Why does it exist?}
    \end{itemize}
\end{definition}