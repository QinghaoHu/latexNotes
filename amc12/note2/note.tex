
\documentclass[10pt]{article}

% --- Packages ---
\usepackage[utf8]{inputenc}
\usepackage{amsmath, amssymb, amsthm}
\usepackage{geometry}
\usepackage{fancyhdr}
\usepackage{graphicx}
\usepackage{tikz}
\usepackage{enumitem}
\usepackage{hyperref}
\usepackage{xcolor}
\usepackage{indentfirst}
\usepackage{pgfplots}
\pgfplotsset{compat=1.18}

% --- Page Setup ---
\geometry{margin=1in}
\pagestyle{fancy}
\fancyhf{}
\rhead{Qinghao Hu}
\lhead{AOPS}
\cfoot{\thepage}

% --- Theorem Environments ---
\newtheorem{theorem}{Theorem}[section]
\newtheorem{definition}[theorem]{Definition}
\newtheorem{lemma}[theorem]{Lemma}
\newtheorem{proposition}[theorem]{Proposition}
\newtheorem{corollary}[theorem]{Corollary}
\theoremstyle{remark}
\newtheorem*{remark}{Remark}
\newtheorem*{example}{Example}

% --- Custom Commands ---
\newcommand{\R}{\mathbb{R}}
\newcommand{\N}{\mathbb{N}}
\newcommand{\Z}{\mathbb{Z}}
\newcommand{\Q}{\mathbb{Q}}
\newcommand{\C}{\mathbb{C}}
\newcommand{\ds}{\displaystyle}

% --- Title ---
\title{\textbf{AOPS AMC12 class Note 2} \\ \large functions}
\author{Qinghao Hu}
\date{\today}

\begin{document}

\maketitle
\tableofcontents
\newpage

% Lesson 4
\begin{center}
\section{Lesson 4: Functions and Polynomials}
\end{center}
Today we will look at the properties of certain functions, such as the "floor" function and logarithm, as well as polynomials,
which form a very special class of functions
\newline

\subsection{ABSOLUTE VALUE}
\textit{The absolute value signs make equation difficult to work with. How might we deal with those pesky bars ($|a|$)} \newline

\begin{example}
if $x < 0$, then what happens to the equation $|x| + x + y = 10$, if $x > 0$, then what happens to the equation $[x] + x + y = 10$
\end{example}

\begin{center}
\begin{tikzpicture}[scale=0.8]
  \draw[->] (-3.5,0) -- (3.5,0) node[right] {$x$};
  \draw[->] (0,-0.5) -- (0,4) node[above] {$y$};
  \draw[domain=-3:3, variable=\x, smooth, thick, blue] 
    plot ({\x}, {abs(\x)});
  \node at (2,2) [right] {$y = |x|$};
\end{tikzpicture}
\end{center}

\subsection{Floor Function}
In case you have not seen it before, $\left \lfloor x \right \rfloor$ is the greatest integer less than or equal to x, also called the floor of x. In other words, 
$\left \lfloor x \right \rfloor$ is $x$ rounded down to the nearest integer.
\begin{center}
(In general \\
	$\left \lfloor x + n \right \rfloor\ = \left \lfloor x \right \rfloor + n$ \\	
for any integer $n$)
\end{center}

\subsection{Logarithms}
\textit{Logarithms identity:}
\begin{center}
	$\log_{b}{x} + \log_{b}{y} = \log_{b}{xy}$ \textbf{(Product law)}\\
	$\log_{b}{x} - \log_{b}{y} = \log_{b}{\frac{x}{y}}$ \textbf{(Quotient Law)}\\
	$\log_{b}{x^n} = n\log_{b}{x}$ \textbf{(Power law)}\\
	$\log_b{x} = \frac{\log_a{x}}{\log_a{b}}$ \textbf{(Power Law)}\\
	$\log_{b^n}{x^n} = \log_{b}{x}$\\
\end{center}

\subsection{POLYNOMIALS}
Let $F(x)$ and $G(x)$ be polynomials. If we divide $G(x)$ into $F(x)$, then we will obtain a quotient $Q(x)$
and a remainder $R(x)$, where the degree of $R(x)$ is less than the degree of $G(x)$. The quotient $Q(x)$ and 
$R(x)$ are unique. \\

Also, if a polynomial has real coefficients, then its nonreal roots must come in conjugate pairs. 

\end{document}

% Referenc
% \section{Introduction}
% Write your introduction or overview of the day's topic here.
%
% \section{Main Concepts}
% \begin{definition}
% A function \( f: A \to B \) is said to be \textbf{injective} if for every \( a_1, a_2 \in A \), \( f(a_1) = f(a_2) \Rightarrow a_1 = a_2 \).
% \end{definition}
%
% \begin{theorem}
% Let \( f: \R \to \R \) be a continuous and strictly increasing function. Then \( f \) is injective.
% \end{theorem}
%
% \begin{proof}
% Assume \( f(a) = f(b) \). Since \( f \) is strictly increasing, if \( a < b \), then \( f(a) < f(b) \), contradiction. So \( a = b \).
% \end{proof}
%
% \section{Examples}
% \begin{example}
% Let \( f(x) = x^2 \). Then \( f \) is not injective on \( \R \), but it is injective on \( [0, \infty) \).
% \end{example}
%
% \section{Diagrams}
% \begin{center}
% \begin{tikzpicture}[scale=1]
% \draw[->] (-2,0) -- (2,0) node[right] {$x$};
% \draw[->] (0,-1) -- (0,4) node[above] {$y$};
% \draw[domain=-1.5:1.5, smooth, variable=\x, blue, thick] 
%     plot ({\x}, {\x*\x});
% \node at (1.2,3.2) {$y = x^2$};
% \end{tikzpicture}
% \end{center}
%
% \section{Summary}
% Summarize what you learned today.
