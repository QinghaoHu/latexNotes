\section{Universal Gravitation, Gravitational field}
\subsection{Force of Gravity}
The formula for the \textbf{Force of Gravity} acting between two objects is:
\[
    Fg = \frac{G * m_{1} * m_{2}}{R^2}
\]
\begin{center}
    $Fg$ = the magnitude of the force of gravity that $m_{1}$ exerts on $m_{2}$ and $m_{2}$ exerts on $m_{1}$\\
    $G$ = Universal Gravitational Constant\\
    $m_{1}$ = the mass of one of the objects (in kg)\\
    $m_{2}$ = the mass of the other object (in kg)\\
    $R$ is the distance separating the objects' \red{center of mass} in (m)
\end{center}

\red{Remainder:} \textbf{Altitude} refers to the distance between the Earth's surface and the object!\\

In reality, \red{every particle} in A exerts a force of gravity on every particle in B. 
If the objects (A and B) are relatively close together and large (relative to their separation distance)
then these forces are not parallel.
\begin{center}
    \includegraphics[width=0.8\textwidth]{pictures/fgparallel.png}
\end{center}

The \textbf{formula works} best for two objects who \red{seperation distance} is \red{very large} relative to their sizes, 
or when both object are perfect \red{sphere}\\

The \textbf{formula works well} for a very, very \red{large} sphere (whose mass is uniformly distributed through out)
and a relatively \red{small} object on its surface\\

You can not use this formula when one object is \red{inside} of another object!

\subsection{Gravational Fields}
\begin{definition}
    \textbf{A force field} is a region surrounding an object in which the object is capable of exerting a force on another object
\end{definition}

A \red{Gravational field} is a region surrounding an object in which the object is capable of exerting a force of gravity 
on another object.

\subsection{Differences between strength of gravity and acceleration}
Acceleration due to gravity:
\begin{center}
    Units: \blue{$\frac{m}{s^2}$}\\
    What does it imply?:  \blue{When an object is in free fall, it will accelerate at that rate}\\
    When is it true: \blue{Only when $Fg$ is the only force on the object}
\end{center}

Gravitational field strength:
\begin{center}
    Units: \blue{$\frac{N}{kg}$}\\
    What does it imply?:  \blue{When an object is in free fall, it will accelerate at that rate}\\
    When is it true: \blue{Gravity is exertings a force of $\mid \vec{g} \mid $ Newtons for each Kg of mass}
\end{center}

Specific types of field strengths are \red{additive}. The \red{net gravitational field strength} at a location is the \red{sum}
of all the individual strengths of gravitational fields at that location OR $\sum \vec{g} = \vec{g_{1}} + \vec{g_{2}}$\\

When you need to calculate \blue{magnitude} of Gravational field from that object $M$ exerts on $m$:
\begin{equation} \label{eq:1}
    Fg_{M/m} = \frac{GMm}{R^2}
\end{equation}
\begin{equation} \label{eq:2}
    Fg_{M/m} = mg
\end{equation}
Add \ref{eq:1} and \ref{eq:2}
\begin{equation}
    g = \frac{GM}{R^2}
\end{equation}
\begin{center}
    $g$ is the \blue{magnitude of the grav field strength} of $M$, at a specific location (in $N/kg$)\\
    $R$ is the distance that the location is from $M$'s centre of mass(in m)\\
    $G$ is the universal gravitational constant ($G$ = $6.67*10^{-11}\frac{Nm^2}{Kg^2}$)
\end{center}