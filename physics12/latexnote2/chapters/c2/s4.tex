\section{Motion of a car on Banked Turn}
\begin{center}
    \includegraphics[width=0.7\textwidth]{pictures/BankedTurn.png}
\end{center}
\subsection{Forces}
For \red{unbanked Turn}, \red{Static friction} contribute to the centripetal force

For \red{Banked Turn}, both \red{static friction} and \red{normal force} contribute to the centripetal force

\subsection{Critical Speed}
\begin{definition}
    Critical speed the minimum speed needed at which a vehicle can travel around a curve, baked road without relying on static friction
\end{definition}

The formula for critical speed is defined as:
% \begin{cyanblock}
    \[
        v = \sqrt{R*tan\theta *g}
    \]
    \begin{center}
        $v$ = Critical Speed\\
        $R$ = The radius of the banked turn\\
        $g$ = The acceleration by Gravity\\
    \end{center}
% \end{cyanblock}

Above the \red{critical speed}, the car wants to go \blue{up}. At this case, friction must act \red{down the bank} to prevent sliding outward

Below the \red{critical speed}, the car wants to go \blue{down}. At this case, friction must act \red{up the bank} to prevent sliding inward