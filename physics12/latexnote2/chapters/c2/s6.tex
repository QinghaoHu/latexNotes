\section{Satellites}
A satellite is an object that \red{orbits around another object}\\

There are \textbf{natural} satellites and \textbf{artifical} object
\begin{itemize}
    \item The moon is a \textbf{natural} satellite of the Earth
    \item The international space station (ISS) is an \textbf{artifical} object
\end{itemize}

\subsection{Netwon's Cannon}
His idea was: \textit{if a cannon is placed on the top of a very tall mountain, and if you could ignore air resistance. The cannon 
shoots a cannonball horizontal}\\

At the idea speed: the distance the cannonball has fall \textbf{equals} the distance that the Earth has \textbf{turned away}\\

If $v  < v_{idea}$, the distance between the ball and Earth's surface will \textbf{decrease}\\

If $v  > v_{idea}$, the distance between the ball and Earth's surface will \textbf{increase}\\

The cannon must has a constant speed and travel in the perfect circular path

\subsection{Geosynchronous}
They have the same orbital period as the \textbf{rotational speed} of the object they are on the ground\\

The period is around \textbf{24 hrs}\\

There is a special type of geosynchronous is called \textbf{Geostationary}
\begin{itemize}
    \item "Hang above" a location on Earth's \textbf{Equator}
    \item They orbit in the same \textbf{direction} that the Earth rotates
\end{itemize}

\subsection{Formulas related to satellite}
We will derive each formulas in this handout:\\

To start off, let's draw the FBD for the satellite\\
\begin{center}
    \begin{tikzpicture}[scale=1.2, >=Stealth]

    % Earth
    \shade[ball color=blue!40] (0,0) circle (1);
    \node at (0,0) {\textbf{Earth}};

    % Satellite position
    \coordinate (satellite) at (0,3);
    \filldraw[gray!60] (satellite) circle (0.15);
    \node[above=2pt of satellite] {Satellite};

    % Forces
    \draw[thick, red, ->] (satellite) -- ++(0,-1.2)
        node[midway, right] {$F_g$};
    \draw[thick, blue, ->] (satellite) -- ++(0,1.2)
        node[midway, right] {$F_{\text{fict}}$};

    % Orbit line
    \draw[dashed, gray] (0,0) circle (3);
    \node[right] at (0,1.5) {Orbit altitude};

    % Optional: radius vector
    \draw[->, thick, black!50] (0,0) -- (satellite)
        node[midway, left] {$r$};

    \end{tikzpicture}
\end{center}

\textbf{Down} is negative\\

Let's derive the formula for satellite:
\begin{gather*}
    \sum \vec{F} = m*\vec{a_{per}}\\
    F_g - F_{fict} = 0\\
    mg - m*\left|F_{FOR}\right| = 0\\
    mg = ma_c\\
    a_c = \frac{Gm}{R^2} 
\end{gather*}

Sub in $a_c = \frac{V^2}{R}$:
\begin{gather*}
    \frac{v^2}{R} = \frac{GM}{R^2}\\
    v = \sqrt{\frac{GM}{R}}
\end{gather*}

\begin{center}
    $v$ = orbital speed\\
    $M$ = mass of the object that is obited in (kg)
\end{center}

Sub in $a_c = \frac{4\pi^2 R}{T^2}$
\begin{gather*}
    \frac{4\pi^2 R}{T^2} = \frac{GM}{R}\\
    T^2 = \frac{4\pi^2 R^3}{GM}\\
    T = \sqrt{\frac{4\pi^2 R^3}{GM}}
\end{gather*}