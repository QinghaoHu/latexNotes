% \newpage
\section{Linear Momentum \& Impulse}
\subsubsection{Linear Momentum}
\textbf{Linear Momentum} is the product of an object's mass and its velocity:
\begin{gather}
    \vec{p} = m\vec{v}
\end{gather}
\begin{center}
    $\vec{p}$ is the Momentum in $(kg * \frac{m}{s})$
\end{center}

Newton called momentum "the \textbf{true Quantity of motion}". Why? Momentum is a combination of an object's 
\textbf{inertia}(its mass basically) and what it is doing (its \textbf{velocity}). He felt that it provided a 
more complete picture of what was required to cause a specific change in what an object was doing. 

\subsubsection{Impulse}
Impulse is the \textbf{product of that force} acting on an object and the \textbf{duration} of time that the 
force acted on the object. 

\begin{gather}
    \vec{J} = \vec{F} * \Delta t
\end{gather}
\begin{center}
    $\vec{J}$ = the impulse in (N*s)
\end{center}

The formula has a similar limitation to the formula for the work done on an object. Both formulas assume that the
force acting on the object. \\

Thus, if the force acting on the object is not constant, we can find the impulse that the force provides by 
finding the area between the line/curve on a \textbf{Force vs Gravity graph}\\

Let's see some formula:
\begin{gather*}
    \sum \vec{J} = \sum \vec{F} * \Delta t\\
    \sum \vec{J} = (m * \vec{a}) * \Delta t\\
    \sum \vec{J} = (m * \frac{\vec{v_2} - \vec{v_1}}{\Delta t}) * \Delta t\\
    \sum \vec{J} = m * \vec{v_2} - m * \vec{v_1}\\
    \sum \vec{J} = \vec{p_1} - \vec{p_1}\\
\end{gather*}
\begin{equation}
    \sum \vec{J} = \Delta \vec{p}
\end{equation}