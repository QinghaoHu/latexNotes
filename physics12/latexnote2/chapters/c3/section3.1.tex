\section{Work and Kinetic Energy}
\subsection{Kinetic Energy}
% \begin{cyanblock}
    \begin{definition}[Kinetic Energy]
         is the energy of \red{motion}. There are two types of kinetic energy
    \end{definition}
% \end{cyanblock}

\subsubsection*{Type 1: Traslational Kinetic Energy}
The kinetic energy that an object has because it is \textit{moving from on location to another}. In can be describe by this 
formula:
\begin{equation}
    E_k = \frac{1}{2}mv^2
\end{equation}
\begin{center}
    $E_k$ is the translational kinetic energy of the object (in $J$ or $kgm^2/s^2$)\\
    $m$ is the mass in $kg$\\
    $v$ is the speed in $m/s$
\end{center}

Because all motion is relative, an object's speed (and therefore $E_k$) \textit{depends on the choose Frame of Reference}\\

\red{Notes:}\\

\red{Do not solve the conservation of energy problem involving a change of Frame of Reference. Start from your perspective}\\

\red{$E_k$ is a scalar, not a vector}

\subsubsection*{Type 2: Rotational Kinetic Energy}
    Not testable, don't give a shit about this qusetion. 

\subsection{Mechancial Work}
% \begin{cyanblock}
    \begin{definition}[Mechancial Work]
    Transfer of energy into $E_k$ or the transfer of kinetic energy into another type of energy
\end{definition}

\begin{definition}[Potential Energy]
   Energy which is stored in a system of objects due to forces acting in between thos objects
\end{definition}
% \end{cyanblock}


\subsubsection{What does the sign of the work mean?}
\begin{itemize}
    \item \textit{Positive} work on a system means it receives energy from its surroundings
    \item \textit{Negative} work on a system means it gave energy to its surroundings 
    \item \textit{Negative} work occurs when the force has a component in the direction opposite the displacement
\end{itemize}
The formula to descirbe the work is:
\begin{equation}
    W = F_{A/B} * \vec{\Delta d_B} * cos\theta
\end{equation}
\begin{center}
    $W_{A/B}$ is the work that $F_{A/B}$ does on the object $B$. This is also the amount of $E_k$ that object A transfers into object B 
    when A exerts a force on B(in J)\\

    $F_{A/B}$ is the magnitude of force that A exerts on B(in J)\\

    $\theta$ is the angle between $F_{A/B}$ and $B$'s displacement
\end{center}

\begin{remark}
    \red{Remainder: Only forces on the direction of displacement is responsible for the work}\\
    
    You should always write $\cos \theta$ when use the formula
\end{remark}


