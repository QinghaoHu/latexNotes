\section{Review of Projectile Motion}
\subsection{basic}
A simple projectile is an object that has a single, \textit{non-uniform} acting on it. This single force must be a 
\textit{non-contact} force between objects without the two objects being contact. Ex. Gravity forces, magnetic forces,
electric forces \\

There are some kind of simple questions:
\begin{enumerate}
    \item An object dropped
    \item A soccer ball is kicked
    \item A bullet is fired from a gun 
    \item An electron is moving through a uniform electrical field (Gravity is negligible)
\end{enumerate}

Remainder: The object only accelerate in the direction of the net force. In any direction perpendicular to net force, 
the object main a constant velocity. 

Some important points to remember:
\begin{itemize}
    \item At maximum height all projectiles have a \textbf{vertical} velocity equal to \textbf{zero}
    \item When an object starts and ends at the same vertical location, the $\vec{\Delta d_{y}} = 0$
    \item When an object is dropped or launched horizontally, then $\vec{v_{y1}} = 0$
\end{itemize}

\subsection{Special formula}
\begin{equation} \label{eq:Special}
    R = \frac{ v_{i}^2 * sin{2\theta} }{g} 
\end{equation}
\begin{center}
    R is the range of the projectile (Horizontal distance in m) \\
    $v_{i}$ is the launch speed of the projectile in (m/s) \\
    $\theta$ is the launch angle of the projectile \\
    $g$ is acceleration due to gravity \\
\end{center}

The formula \ref{eq:Special} can only be used if the project \textcolor{red}{starts and ends at the same vertical location}
\newpage
\subsection{An example question}
Remainder, you should always label your direction conventions

\mypic{pictures/projectileReview.png}{}{0.8}