\section{Coulomb's Law}
\subsection{Background}
Coulomb was a scientist who studied electricity in the early 1800's. He wanted to find out what factor affect 
\underline{the electrostatic force with two charged objects}. \\

Coulomb based his experiment on \underline{Carendish's} experiment.\\

To be able to perform the experiment Coulomb needed to electrically charge each of the pith balls and 
know the \underline{magnitude of the charge} on each bal. His solution for this was to find the 
\underline{relative} magnitude of the charge on each pith ball. 

\subsection{Formula}
By measuring the amount of force, the separation distance between the charged objects and the relative 
charge of the pith balls, Coulomb was abl to find the following relationships:
\begin{gather}
    F_{E} \propto \tfrac{1}{R^2} \nonumber \\
    F_{E} \propto q_{1}q_{2} \nonumber
\end{gather}

We can bring these proportionalities together:
\begin{equation*}
    \left| F_E \right| = \frac{k \left |q_1 \right |  \left |q_2 \right |}{R^2}
\end{equation*}
\begin{center}
    $F_E$ is the magnitude of the electrical force in between \underline{two point charges}\\
    $q_a$ and $q_b$ is the absolute value of the charge of each object (in C)\\
    $R$ is the separation distance betweeen the objects (in m)\\
    $k$ is Coulomb's law constant of proportionality ($k = 8.99 \times 10^9 \tfrac{Nm^2}{C^2}$)
\end{center}

\begin{remark}
    When using equations for electrical forces, \underline{don't substitute in the sign of the charge}. Find the direction of the force \underline{conceptually}!
\end{remark}