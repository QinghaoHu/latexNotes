\section{Electric Potential Energy \& Electric Potential}
\subsection{Electric Potential Energy}
\begin{definition}
    [Electric Potential Energy] The energy \underline{stored} in a system of \underline{two or more objects} due to the \underline{electrical force} acting in between the charges.
\end{definition}

\subsubsection{Formula}
Formula for electrical potential energy stored in a system of two charges:
\begin{equation*}
    E_E = \frac{k q_A q_B}{R}
\end{equation*}
\begin{remark}
    Remember, always input the sign of $q_A$ and $q_B$
\end{remark}

You may notice, there is no negative sign for the formula of electrical potential energy compare to gravitional potential energy.\\

Gravity is always a force of \underline{attraction} (this is what causes the negative in the formula)\\

However, electrical forces can either be forces of attraction or repulsion, which means that electrical energy can either be \underline{negative} or \underline{positive}.

\begin{gather*}
    \text{Repulsion:} + \\
    \text{Attraction:} -
\end{gather*}

For electrical potential energy, electrical potential, or electric potential difference, always include 
\textit{\underline{the sign of charges}} into the formula\\

Now we have a new type of mechancial energy to add to our expression!

\begin{equation*}
    E_M = E_g + E_k + E_s + E_E
\end{equation*}

\begin{remark}
    Gravitational Potential Energy is typically \underline{negligible} in compression to electric Potential Energy
\end{remark}

\subsection{Electric Potential for Point Charges}
\begin{definition}
    [Electric Potential]The electrical Potential per coulomb of charge at a location. 
\end{definition}

Let's discuss the difference between \textit{electric field} and \textit{electric potential}\\

\begin{paracol}{2}
    \centering
    \begin{leftcolumn}
        Electric Field
        \begin{itemize}
            \item Can exist without there being an electrical force
            \item To have an electric force, a charge needs to be at a location where there is an electric field
            \item $\mathcal{E}$
            \item $\tfrac{N}{C}$
        \end{itemize}
    \end{leftcolumn}

    \begin{rightcolumn}
        Electric Potential
        \begin{itemize}
            \item Can exist without there being electrical potential energy
            \item To have elctric potential energy, a charge needs to be at a location where there is electrical potential
            \item $V$
            \item $\tfrac{J}{C}$
        \end{itemize}
    \end{rightcolumn}
\end{paracol}

\subsubsection{Formula}
Electric Field:
\begin{equation}
    \vec{F_E} = \vec{\mathcal{E}} \times q
\end{equation}
\begin{remark}
    Do \underline{not} substitue the sign of the charge
\end{remark}

Electric Potential:
\begin{equation}
    E_E = Vq
\end{equation}
\begin{remark}
    Substitue the sign of the charge
\end{remark}

\subsubsection{Calculate the electrical potential around a point charge}
\begin{gather}
    E_E = Vq_1 \nonumber \\
    \frac{k \times q \times q_1}{R} = Vq_1 \nonumber \\
    V = \frac{k \times q}{R}
\end{gather}
\begin{center}
    $V$ is the elec potential (of $q$) at a distance of $R$ away from $q$
\end{center}