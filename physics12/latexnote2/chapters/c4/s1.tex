\section{Review of Electronstatics}
In this section, we will briefly review basic Electronstatics which we learned from Grade 9 Science

\subsection{Electric Charge}

\subsubsection{Electron}
By the early 1900s, physicists had identified the subatomic particles called the electron and the proton
as the basic units of charge. All protons carry the same amount of positive charge, $e$, and all electrons 
carry an equal but opposite charge, $-e$. Charges interact with each other in very specific ways governed by the 
\textbf{law of electric charges}

\begin{theorem}[Law of Electric Charges]
    Like charges repel each other; unlike charges attract.
\end{theorem}

\noindent\hrulefill

\subsubsection{Charge of atom}
\begin{center}
    Cation: a positive ion. $\#$ of protons $> \#$ of electrons\\
    Anion: a negative ion. $\#$ of protons $< \#$ of electrons\\
\end{center}

The \textbf{Total Charge} is the sum of all the charges in that object and can be positive, negative or zero. The charge is equal to zero when the 
negative charge equals to negative charge. 

\begin{theorem}[Law of Conservation of Charge]
    Charge can be transferred from one object to another, but the total charge of a closed system remains constant. 
\end{theorem}
\noindent\hrulefill

\subsubsection{Coulomb}
The basic unit of charge is called the coulomb (C). The charge of electron, $-e$, is $-1.60*10^{-19}C$, and the charge of a single proton, $+e$, is 
$1.60*10^{-19}C$\\

Symbol $e$ often donotes the magnitude of the charge of an electron or a proton. \\

The symbol $q$ denotes teh amount of charge, such as the total charge onf a small piece of paper. In other words, the total charge of a particle 
is $q$.

% \noindent\hrulefill

\subsection{Conductors and Insulators}
\begin{definition}[Conductor]
    A conductor is a substance in which electrons can move easily among atoms.
\end{definition}

\begin{definition}[Insulator]
    any substance in which electrons are not free to move easily from one atom to another.
\end{definition}

Insulator hold the electron when other electron come in. There are no free electrons in the insulator, and insulator does not allow the extra electrons 
to move about easily. 

% \noindent\hrulefill

\subsection{Different methods of charging}
\subsubsection{Charging an Object by Friction}
In reality, some object has stronger ability to hold on electrons than others. Assume we have two neutral object, when these two objects touch, 
electrons will follow from the object with weaker hold on electrons to the other one with stronger hold on electrons. 

% \noindent\hrulefill

\subsubsection{Carging an object by Induced Charge Separation}
Assume we have two objects, one with zero charge and other one with negative charge. When we put the negative object towards the positive object, 
electrons in the neutral object will repel to the electrons in the negative object. As a result, electrons in the neutral object will redistributed 
throughout the material. The positive side of the netural object is closer than the negative side of the object, in which makes the neutral object 
attrack to the negative object. 

% \noindent\hrulefill
\subsubsection{Charging by Contact}

\begin{figure}[!h]
    \centering
    \includegraphics[width=0.7\textwidth]{pictures/4.1.1.png}
    \caption{Picture from my textbook}
\end{figure}

\subsubsection{Charging by Induction}
\begin{center}
    Using a negative object to create a positive object
\end{center}

\begin{figure}[!h]
    \centering
    \includegraphics[width=0.7\textwidth]{pictures/4.1.2.png}
    \caption{Picture from my textbook}
\end{figure}