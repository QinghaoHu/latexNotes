\section{Simple Harmonic Motion}
\subsection{Definitions}
\subsubsection{Simple Harmonic Motion}
An object undergoes simple harmonic motion, if two conditions are met:
\begin{enumerate}
    \item The net force acting on the object is \textbf{directly proportional} to the object's displacement away from equilibrium
    (The location where $\sum F = 0$)
    \item The direction the net force acts on the object must be \textbf{opposite} the object's displacement from equilibrium. 
\end{enumerate}

\begin{figure}[h!]
    \centering
    \includegraphics[width=0.7\textwidth]{graph/SHM.png}
\end{figure}

\newpage

Let see the FBD for $m$:

\columnratio{0.4, 0.6}

\begin{paracol}{2}
    \begin{leftcolumn}
        \begin{figure}[h!]
            \centering
            \includegraphics[width=0.2\textwidth]{FBD/FBD1.png}
        \end{figure}
    \end{leftcolumn}
    \begin{rightcolumn}
        \begin{gather}
            \sum \vec{F} = \vec{F_{spring}}\\
            \sum \vec{F} = -k\vec{x}
        \end{gather}        
    \end{rightcolumn}
\end{paracol}
Changing to proportionality:
\begin{equation}
    \text{Net Force} \propto -\vec{x}
\end{equation}

\subsubsection{Expectations}
If there was \textbf{friction} acting on the object, it would no longer undergo simple harmonic motion. We would call the motion of the object: 
\textbf{Damaged Harmonic Motion}\\

Some Real life examples:
\begin{itemize}
    \item Car shock absorbers
    \item A guitar string
    \item A pendalam or a string
    \item Bungee jumping
\end{itemize}

\subsection{How can we solve form the acceleration of SHM}
Returning to our FBD of the mass and the net force statement from it:
\begin{gather}
    \sum \vec{F} = m \vec{a}\\
    \vec{F_s} = m\vec{a}\\
    -kx = m\vec{a}\\
    a = \frac{-kx}{m}
\end{gather}
\begin{center}
    $\vec{a}$ is the acceleration of the mass (in $m/s^2$)\\
    $k$ is the force constant of the spring (in $N/m$)\\
    $\vec{x}$ is the displacement of the mass from its equilibrium position (in m)\\
    $m$ is the mass of the object that is attached to the spring (in kg)
\end{center}

\begin{remark}
    Because the acceleration is a vector, we need to make a direction convention to use the equation. 
\end{remark}

\newpage
\subsection{The period of the Simple Harmonic Motion}
The y-component of the uniform circular motion is similar to the acceleration of the simple Harmonic Motion
\begin{paracol}{2}
    \begin{leftcolumn}
        For the object going around circle:
        \begin{equation}
            a_c = \frac{4 \pi^2 R}{T^2}
        \end{equation}
    \end{leftcolumn}
    \begin{rightcolumn}
        For the mass on the end of the spring:
        \begin{gather}
            \vec{a} = -\frac{k \vec{x}}{m}\\
            \left | \vec{a} \right | = \frac{kx}{m}
        \end{gather}
    \end{rightcolumn}
\end{paracol}

When the object is on the top/bottom of the perfect circular motion, the acceleration is equal to the magnitude of the acceleration of the object at the equilibrium position:
\begin{gather}
    a_c = \left | a \right |\\
    \frac{4 \pi^2 R}{T^2} = \frac{kx}{m}\\
    T = +/- \sqrt{\frac{m * 4\pi^2 R}{kx}} (\text{At this time $R$ = $x$})\\
    T = 2\pi * \sqrt{\frac{m}{k}}
\end{gather}