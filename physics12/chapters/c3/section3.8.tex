\section{Types of Collisions}
\subsection{Definitions}
Collisions are typically classified based on the amount of \textbf{kinetic energy} the syste has after the 
collision, in comparsion to the amount of kinetic energy the system had before the collision. In other words,
\textbf{how does $E_k'$ with $E_{k}$}?

\subsection{Elastic Collisions}
In an elastic collision, the kinetic energy of the system after the collision is \textbf{equal} to the kinetic
energy of the system before the collision. In mathematics, the equation can be represented by:
\begin{gather*}
    E_k` = E_k
\end{gather*}

\begin{remark}
    This does not mean the kinetic energy of the system after the collision is \textbf{equal} to the kinetic 
    energy of the system before the collision (Unlike momentum)
\end{remark}

\subsubsection{Steps of the collision} 
\begin{remark}
    This is on a horzontal frictionless surface, so we can ignore Gravational Potential Energy
\end{remark}
\textbf{Before the collision}:The mechancial of energy is entirely in the form of kinetic energy \\

\textbf{First half ot collision}:The interaction forces cause the object to start deform. As the object deform, 
they transfer $E_k$ to $E_s$.\\

\textbf{At the approximate midpoint of the collision}:The deformation of the object is at a maximum. $E_S$ is the maxium and $E_k$ is the minimum.\\

\textbf{During the second half of the collision}: The restoring forces are now doing \textit{positive work} on the system, transferring 
elastic potential energy \textbf{back into $E_k$}\\

\textbf{After the collision}: The system's mechanical energy is now entirely $E_k$, at this time, $E_s = 0$. All $E_s$ is transfered to $E_k$. 

\subsubsection{Head-on Collision} 
\textbf{Before the Collision}: System's mechancial energy is entirely $E_k$\\

\textbf{During the first half of the collision}: The spring get compressed. $E_k$ is transformed into $E_s$.\\

\textbf{At the mid-point of the collision}: \begin{itemize}
    \item Spring is at the most compressed point.
    \item Distance between cars are minimumized
    \item $\vec{v_{A}} = \vec{v_{B}}$
    \item $E_k$ is minimumized
    \item $E_s$ is maximumized
\end{itemize}

\textbf{During the second half of the collision}: \begin{itemize}
    \item $\vec{v_A} < \vec{v_B}$
    \item Distance between the carts is increasing
    \item $E_s$ is being transferred back into $E_k$
\end{itemize}

\textbf{After the collision}:
The system is entirely $E_k$ now

\begin{remark}
    Elastic collisions \textbf{cannot occur} between visible objects in rael life. At least some of the enery will be lost as thermal or sound. 
\end{remark}


\subsection{Inelasic Collision}
So for this collision, $E_k`$ is less than $E_k$\\

It is impossible for a system to have more kinetic energy after the collision, than it had before the collision, unless:
\begin{enumerate}
    \item One of the object had \textbf{stored energy} before the colllision, which was transferred into kinetic enery during the collision. 
    \item An \textbf{external} force (such as force of gravity) is doing positive work on the system, during the collision. 
\end{enumerate}

\subsubsection{Completely inelastic collision}
In this collision, the maximum amount of kinetic energy that could be "lost" is lost as a result of the collision. \\

After the collusion, the objects involved in the collision will be \textbf{stack/attached together}\\

\textbf{Apple and Arrow} is an example of this question. \\

The following condition must be met for a Perfectly Inelastic Collision:
\begin{itemize}
    \item $\vec{v_A}` = \vec{v_B}` = \vec{v}`$
\end{itemize}