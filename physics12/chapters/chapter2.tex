\chapter{Unit 1B}

\section{Fictitious Forces and Apparent Weight}
\subsection{Fictitious Forces}
\red{Fictitious forces} are also called \red{apparent forces} or \red{perceived forces}

\begin{redblock}
    \textbf{Explanation:} When the object is viewed from a \red{non-intertial F.O.R, we created fictitious force to explain the motion and behavior}
\end{redblock}

The \red{fictitious forces} will always act in the direction opposite to the direction of acceleration of 
the frame of reference.\\

\begin{cyanblock}
    The magnitude of each fictitious force can be calculated by:
    \[
        F_{fict} = m|\vec{a_{F.O.R}}|
    \]
\end{cyanblock}

Perceived acceleration could be represented by \red{$\vec{a_{per}}$}

\begin{redblock}
    \textbf{Note: } The object's actual acceleration would be measured relative to an inertia FOR
\end{redblock}

\subsection{Apparent Weight}
Technically, this would be the sum of the \red{normal force} and the force of \red{friction} that a surface exerts 
on an object. 

\subsection{Some of the formulas}
\[
    \sum \vec{F} = m\vec{a_{per}}
\]