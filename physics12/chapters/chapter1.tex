\chapter{Unit 1A}
\section{Review of Describing and Graphing Motion}
\subsection{Position: $\vec{d}$}
Position is the \textbf{straight-line distance} from a fixed reference point to a location, with a direction to the location from the reference point.

\subsection{displacement: $\Delta\vec{d}$}
Displacement is the \textbf{change of position} \\
Formula: \\
\[
	\Delta\vec{d} = \vec{d_{2}} - \vec{d_{1}}
\] 
or \\
\begin{center}
    $n$ = the amount of displacement you want to add
\end{center}
\[
	\Delta\vec{d_{tot}} =  \sum_{i = 1}^{n} \Delta\vec{d_{i}}
\]

\subsection{Velocity: $\vec{v}$}
Velocity is the \textbf{rate of change of position}
\[
	\vec{v} = \frac{\Delta\vec{d}}{\Delta t}
\]

\subsection{Acceleration: $\vec{a}$}
Acceleration is the \textit{rate of change} of velocity, always in the form of $\frac{m}{s^2}$
\[
\vec{a} = \frac{\Delta \vec{v}}{\Delta t}
\]

\subsection{Graphing motion}
For a \textbf{Position vs Time} graph:
\begin{itemize}
\item \textbf{For a position/displacement vs time graph, the velocity = the \textit{slope}} 
\item A slope of zero = The object is not moving
\item Instantaneous velocity ($\vec{v}_{inst}$) = the slope of \textbf{tangent} line to the graph at that point in time
\item Average velocity ($\vec{v}_{avg}$) = the slope of \textbf{secant} line for that time interval
\end{itemize}
For a \textbf{Velocity vs Time} graph:
\begin{itemize}
    \item Can get an object's instantaneous velocity directly from the graph
    \item slope = \textbf{acceleration}
    \item Displacement = The \textbf{area} between the graph and the time-axis for that time interval
\end{itemize}

\newpage
\section{Equations of Motion}
To start off, there are five equations that are used in the calculation of motion
\[
    \vec{v_{f}} = \vec{v_{i}} + \vec{a}*\Delta t
\]
\[
    \Delta\vec{d} = \frac{1}{2}(\vec{v_{i}} + \vec{v_{f}}) * \Delta t
\]
\[
    \Delta\vec{d} = \vec{v_{i}}\Delta t + \frac{1}{2}\vec{a} * \Delta t^2
\]
\[
    \Delta\vec{d} = \vec{v_{f}}\Delta t - \frac{1}{2}\vec{a} * \Delta t^2
\]
\[
    \vec{v_{f}}^2 = \vec{v_{i}}^2 + 2 \vec{a} \Delta \vec{d} 
\]

\subsection{Format requirements for answering Motion questionss}
\begin{enumerate}
    \item You should always include a diagram that contains every known information from the question
    \item $\vec{v}$ should be presented for at least two decimal places
    \item Always list steps in your answer
    \item When using quadratic solving function on calculator, always write \textit{*Using Quadratic Eq} in your answer
    \item When you form an equation system, you should label $1. 2. 3.$ on each equation in the system
    \item Follow the Sig Digit rules
    \begin{itemize}
        \item[!] For \textbf{add} and \textbf{subtract}, keep the least \textbf{decimal places}
        \item[!] For \textbf{multiply} and \textbf{divide}, keep the least amount of \textbf{signficiant digit} 
    \end{itemize}
\end{enumerate}

\newpage
\section{Adding and Subtracting 2-Dimensional Vectors}
\subsection{Vector addition and subtraction key words}
\begin{itemize}
    \item[+] Addition: Find "the \textbf{resultant}", "the total", or "the net". 
    \item[-] Subtraction: Find "the \textbf{difference}" or "the change in".
\end{itemize}

\subsection{Steps for solving a vector problem}
\begin{enumerate}
    \item Read the question carefully
    \item Show unit conventions
    \item Write "givens" (It helps to roughly sketch each vector and their compoents)
    \item Set direction conventions
    \item Solve for each compoents (ex. $\Delta\vec{d_{1y}}, \Delta\vec{d_{2x}}$)
    \item Choose one component direction (ex. Just the 'x' direction) and solve the equations for that direction
    \item Repeat with the other direction
    \item Sketch your resulting $x$ and $y$ vectors, joining them head-to-tail. 
    \item Calculate the magnitude and direction of the resultant. (Trigonometry)
    \item State the final answer, including the real-world direction
\end{enumerate}

\mypic{graph/graph1.png}{A sample answer for vector question}{0.8}

\subsection{Another question type}
\mypic{graph/teacherNote2.png}{Remainder, multiply first}{0.9}

\newpage
\section{Frame of Reference}
\subsection{1 Dimension Frame of Reference}
For certain \textcolor{red}{Chase and Collision} questions, you can use \textcolor{blue}{frame of reference} 
to solve them
\begin{example}
    How to write givens\\
    \mypic{graph/graph2.png}{You should always write it}{0.7}
\end{example}
\newpage
\section{Relative Velocities in Two Dimensions}
\subsection{Recall}
\begin{enumerate}
    \item All velocities are related to an F.O.R that we consider to be \textcolor{red}{motionless or stopped}
    \item A frame of reference is just a \textcolor{red}{perspective} from which we observe/or measure the motion of objects
\end{enumerate}
\subsection{Definition}
We sometimes refer to something that another thing can travel through as \textit{medium}. We also sometimes refer to the thing that is moving through the medium as \textit{object(O)}, and the 
ground use the symbol \textit(G)

Therefore the equation can be written as:
\[
\vec{v_{OG}} = \vec{v_{OM}} + \vec{v_{MG}}
\]

\mypic{graph/graph4.png}{An example question}{0.9}
\newpage
\section{F.O.R in 2-D}
\mypic{graph/graph5.png}{Teacher's note}{0.9}
\newpage

\section{Review of Netwon's Laws of Montion}
\subsection{Netwon's First Laws}
Inertia is an object's \textcolor{blue}{resistance} to a change in its state of uniform Motion.
\begin{example}
    An object at rest will \textcolor{red}{remain at rest} And an object in motion will continue
    to \textcolor{red}{move in a straight line constant speed} UNLESS a non-zero net force acts on 
    the object
\end{example}

\subsection{Newton's second Law}
Newton's $2^{nd}$ Law is the formula that explains the behaviour of object when the forces on the object 
are not zero. \\
We can orgnize to these formulas:
\begin{equation}
    \vec{a} = \frac{\sum \vec{F}}{m}
\end{equation}
and\\
\begin{equation}
    \sum \vec{F} = m * \vec{a}
\end{equation}
\begin{center}
    $\vec{a}$ = acceleration of the object \\ 
    $m$ = mass of the object in kg \\ 
    $\sum \vec{F}$ = The sum of the net force
\end{center}

\subsection{Newton's Third Law}
For every force, there is another force, which is \textit{equal in magnitude} to the first force, 
but \textit{opposite in direction}. These two forces will \textit{act on separate objects},
unless they are "internal force"\\ 
This mean's that all forces Always \textcolor{blue}{come in pairs}, but two forces may not be 
acting on the same object. \\

To fit the Newton's $3^{rd}$ Law pair forces, the two force must:
\begin{enumerate}
    \item Be the same type of force
    \item $\vec{F_{A/B}} = -\vec{F_{B/A}}$
\end{enumerate}
\newpage
\subsection{Free Body Diagrams (FBD)}
\mypic{graph/graph6.png}{An example Free Body Diagram}{0.7}

\subsection{Application of Newton's second Law}
Here is an example:
\mypic{graph/graph7.png}{}{0.8}

\section{Review of Projectile Motion}
\subsection{basic}
A simple projectile is an object that has a single, \textit{non-uniform} acting on it. This single force must be a 
\textit{non-contact} force between objects without the two objects being contact. Ex. Gravity forces, magnetic forces,
electric forces \\

There are some kind of simple questions:
\begin{enumerate}
    \item An object dropped
    \item A soccer ball is kicked
    \item A bullet is fired from a gun 
    \item An electron is moving through a uniform electrical field (Gravity is negligible)
\end{enumerate}

Remainder: The object only accelerate in the direction of the net force. In any direction perpendicular to net force, 
the object main a constant velocity. 

Some important points to remember:
\begin{itemize}
    \item At maximum height all projectiles have a \textbf{vertical} velocity equal to \textbf{zero}
    \item When an object starts and ends at the same vertical location, the $\vec{\Delta d_{y}} = 0$
    \item When an object is dropped or launched horizontally, then $\vec{v_{y1}} = 0$
\end{itemize}

\subsection{Special formula}
\begin{equation} \label{eq:Special}
    R = \frac{ v_{i}^2 * sin{2\theta} }{g} 
\end{equation}
\begin{center}
    R is the range of the projectile (Horizontal distance in m) \\
    $v_{i}$ is the launch speed of the projectile in (m/s) \\
    $\theta$ is the launch angle of the projectile \\
    $g$ is acceleration due to gravity \\
\end{center}

The formula \ref{eq:Special} can only be used if the project \textcolor{red}{starts and ends at the same vertical location}
\newpage
\subsection{An example question}
Remainder, you should always label your direction conventions

\mypic{graph/projectileReview.png}{}{0.8}

\section{Friction}
In high school we deal with two types of friction: static and Kinetic

\subsection{Kinetic Friction}
Used when the two surfaces that are in contact are \textit{slides relative to each other}.
\begin{equation}
    F_{fk} = \mu_{fk} * F_{n}
\end{equation}

Kinetic friction on an object can make an object \underline{slow down} or \underline{speed up}

\subsection{Static Friction}
Used when the two surfaces are \underline{Not Sliding} relative to each other
\begin{equation}
    F_{fs.max} = \mu_{fs} * F_{n}
\end{equation}

\subsection{Remainder}
Don't forget to mention to prove \underline{$F_{N} = F_{g}$}
\mypic{graph/example9.png}{}{0.7}

\section{Tension, compression and Pulleys}
\textbf{\textit{Internal Forces}} are Netwon's third Law pair of forces that act inside of an object or a system of objects
that are in contact with each other. \\

According to the Netwon's third law, when we analyze the system, the sum of the forces equals 0. 
Thus, these forces will have no impact on the net force acting on the system and will have 
\textbf{\textit{no effect on the system's acceleration}}

\subsection{Tention: T}
If one object is pulling the other object then you are dealing with \textbf{Tension}.

\subsection{Compression: C}
If one object is pushing the other object then you are dealing with \textbf{Compression}
\\

Here is an example question
\mypic{graph/example10.png}{Pulley question example}{0.6}

\section{Inclined plane with Friction}
To deal with inclined plans with friction, you need to determine the direction that the system will likely to accelrate. 
\\

Think of this as \textbf{tug of war} between these 2 forces,. Whichever force has a greater magnitude will be the direction that the system will 
likely to \textbf{accelerate}

\subsection{How to determine the direction that the system will likely to accelerate}
Look at the \textcolor{blue}{forces} acting on the system. (Active forces are forces that are \textit{trying to make the system move})
\textcolor{red}{Do not include internal forces} \\

Along the potential line of the system's acceleration:
\begin{itemize}
    \item Find the sum of the active forces, which are in one Dimension
    \item Find the sum of the active forces which are acting in the opposite direction
\end{itemize}
Compare the magnitude of these two force sums, and whichever direction has the larger magnitude of active forces is the directions that 
system is likely to accelerate

\subsection{Example template}
\begin{center}
    Active forces[label directions]\\
    Calculate for it\\
    Process$\cdots$\\
    Active forces[Label directions]\\
    Calculate for it\\
    $\therefore$ the system will more likely to move in which direction. The friction will move $\cdots$
\end{center}