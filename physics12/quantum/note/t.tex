\documentclass[10pt]{article}

% --- Packages ---
\usepackage[utf8]{inputenc}
\usepackage{amsmath, amssymb}
\usepackage{geometry}
\usepackage{fancyhdr}
\usepackage{graphicx}
\usepackage{tikz}
\usepackage{enumitem}
\usepackage{hyperref}
\usepackage{xcolor}
\usepackage[framemethod=tikz]{mdframed}
\usepackage{indentfirst}
% \usepackage[most]{tcolorbox}
\usepackage{blindtext}
\usepackage{multicol}
\usepackage{paracol}
\usepackage{amsthm}
\usepackage{titlesec}



% --- Page Setup ---
\setlength{\columnsep}{0.3cm}
% \geometry{margin=1. 3in}
\pagestyle{fancy}
\fancyhf{}
\rhead{Hu}
\lhead{Relativity Theory + Quantum Mechanicsm}
\cfoot{\thepage}
% \setsecnumdepth{subsection}
% \setsecnumdepth{subsubsection}


% --- Colored Example Box (orange) ---
\newtheoremstyle{bluehead}        % name
  {\topsep}                        % space above
  {\topsep}                        % space below
  {\itshape}                       % body font
  {}                               % indent
  {\color{blue}\bfseries}          % HEAD FONT (blue & bold)
  {}                               % punctuation after name
  {\newline}                            % space after head
  {}       

  \newtheoremstyle{redhead}        % name
  {\topsep}                        % space above
  {\topsep}                        % space below
  {\itshape}                       % body font
  {}                               % indent
  {\color{red}\bfseries}          % HEAD FONT (blue & bold)
  {}                               % punctuation after name
  {\newline}                            % space after head
  {}       
% --- Theorem Environments ---
% \theoremstyle{bluehead}
\newtheorem{theorem}{Theorem}[section]
\newtheorem{definition}[theorem]{Definition}
\newtheorem{lemma}[theorem]{Lemma}
\newtheorem{proposition}[theorem]{Proposition}
\newtheorem{corollary}[theorem]{Corollary}
\newtheorem{postulate}{Postulate}
% \theoremstyle{redhead}
\newtheorem{example}{Example}
\theoremstyle{remark}
\newtheorem*{remark}{Remark}

% \titleformat{\section}
%   {\normalfont\Large\bfseries}            % format of the title text
%   {\textcolor{red}{\S\thesection}}        % label (this is what becomes red)
%   {0.50em}                                % separation between label and title
%   {}                                      % before-code

% % Red § + red number for \subsection
% \titleformat{\subsection}
%   {\normalfont\large\bfseries}
%   {\textcolor{red}{\S\thesubsection}}
%   {0.50em}
%   {}

% \titleformat{\chapter}[hang]
%   {\normalfont\huge\bfseries}          % chapter title style
%   {\textcolor{red}{\S\thechapter:}}    % label (no "Chapter", just §<num>:)
%   {0.50em}                             % space between label and title
%   {}           



% --- Custom Commands ---
\newcommand{\R}{\mathbb{R}}
\newcommand{\N}{\mathbb{N}}
\newcommand{\Z}{\mathbb{Z}}
\newcommand{\Q}{\mathbb{Q}}
\newcommand{\C}{\mathbb{C}}
\newcommand{\ds}{\displaystyle}
\newcommand{\blue}[1]{\textcolor{blue}{#1}}
\newcommand{\red}[1]{\textcolor{red}{#1}}

\newcommand{\mypic}[3]{
    \begin{figure}[h!]
        \centering
        \includegraphics[width=#3\textwidth]{#1}
        \caption{#2}
    \end{figure}
}

\newenvironment{worddef}[1]
  {\par\noindent\textbf{#1:} \itshape}
  {\par\normalfont}

% --- Title ---
\title{\textbf{Special theory of Relativity and Quantum Mechanics} \\ \large SPH4U}
\author{Qinghao Hu}
\date{\today}

% \includeonly{chapters/c1/chapter}

\begin{document}

\maketitle

In unit 5, you are expecting to self-study special relativity and quantum thoery. 

\newpage
\tableofcontents


\chapter{Probability Distributions}
\newpage
\section{The Wonders of the Solar System}
\subsubsection{Q1}
The further a planet is from the sun, the \textbf{slower} it's speed, and the \textbf{longer} one revolution around the 
Sun takes. \\

\subsubsection{Q2}
The type of celestial object appears to change its position amongst the stars from night to night is called \textbf{planet}\\

\subsubsection{Q3}
When the Earth "overtakes" Mars in orbit, Mars appears to move \textbf{backward} against the backgrounds of the stars.\\

\subsubsection{Q4}
Everything in our solar system was formed from a \blue{nebula}, a giant \textbf{cloud} of gas and dust. \\

\subsubsection{Q5}
What type of event is thought to have disturbed the nebula and to have led to the formation of the solar system?\\
\begin{itemize}
    \item  A shock wave from a nearby supernova explosion. 
\end{itemize}

\subsubsection*{Q7}
The ring \textbf{nearest} Saturn are the fartest., just like planets orbiting the Sun. \\

\subsubsection{Q8}
Whater material makes up Saturn's rings? \textbf{Water ice}\\


\newpage
\section{Uniform Distributions}
\subsection{Different Distributions}
Distributions of data can be classified by considering the general shape of its graph. This picture is a table of distributions which is copied from 
the teacher's note

\begin{figure}[!h]
    \centering
    \includegraphics[width=0.9\textwidth]{pictures/5.2.1.png}
    \caption{Thanks to \textbf{Mr Tang}}
\end{figure}

\subsection{Characteristics of Uniform Distribution}
If an distribution is considered Uniform, it will have following characteristics:
\begin{enumerate}
    \item Each outcome is EQUALLY likely in any single trial of experiment
    \item If $X$ is discrete and $n$ is the number of possible outcomes in the probability experiment, then 
    \begin{gather*}
        P(X = x) = P(x) = \frac{1}{n}\\
        E(x) = \frac{\Sigma x_i}{n} \\
        \sigma = \sqrt{
            \frac{1}{n} \sum (x - E(x))^2
        }
    \end{gather*}
    \item If $X$ is continous with values in range from $a$ to $b$, then the expected value $E(x)$ will be $\frac{a + b}{2}$
\end{enumerate}

\begin{example}
    I want to discuss
\end{example}

\begin{proof}
    
\end{proof}
\newpage
\section{Kepler's Law of Planetory Motion}

\begin{figure}[h!]
    \centering
    \includegraphics[width=0.5\textwidth]{pictures/Kepler'sEllipse1.jpg}
\end{figure}

The sun is at one focus\\

There are two points in this diagram:
\begin{itemize}
    \item The point closet to the Sun is called \textbf{\textit{Perihelion}}. \textit{Peri} means \textit{near} in the Latin.
    \item The point farthest from the Sun is called \textbf{\textit{Aphelion}}. \textit{Ape} means \textit{far} in the Latin.
\end{itemize}

\subsection{Kepler's first law}
\begin{definition}
    Planet's orbit in ellipses with the Sun at one focus
\end{definition}

Ellipses can be classified based on their \textbf{eccentricity}

\[
    e = \frac{c}{a}
\]

\begin{center}
    e = Eccentricity \\
    c = Distance from centre to a focus (in m(or Au))\\
    a = Length of semi-major axis (in m(or Au))
\end{center}

The eccentricity of Earth's orbit is 0.02.\\

The most eccentric planetary orbit in our solar system is \textit{Mercury}, which has a eccentricity of 0.2.\\

Coments tend to the the largest eccentricity very close to 1.\\

Here I want to discuss about the meaning of eccentricity:
\begin{itemize}
    \item If an ellipse has an eccentricity of \textbf{0}, the object is orbit its sun in a \textbf{perfect circle}.
    \item If an ellipse has an eccentricity of \textbf{1}, the object is not in an \textbf{orbit}. 
\end{itemize}

\subsection{Kepler's second law}
\begin{definition}
    A line segment joining a planet and the sun \textbf{sweeps out equal areas in equal amount time}
\end{definition}

By the second law, we can make a conclusion. A planet moves fastest with it is at the \textbf{perihelion} and slowest when it is at the \textbf{aphelion}.

\subsection{Kepler's Third Law}
\begin{definition}
    The square of the orbital period of a planet directly proportional to the cube of the length 
    of the semi-major axis of its orbit
\end{definition}

\[
    p^2 = a^3
\]
\begin{center}
    $p$ = orbital period in (years)\\
    $a$ = Length of semi-major axis (in Au)
\end{center}

\begin{figure}[h!]
    \centering
    \includegraphics[width=0.7\textwidth]{pictures/Kepler'sLaw3.png}
    \caption{This is a log graph}
\end{figure}

The semi-major axis of an orbit is sometimes referred to as the \textbf{average} distance from the sun.
\newpage
\section{Approximate Hypergeometric}
If the population size $N$ is fairly large, and the sample size is relatively small, we can use a \underline{Normal Distribution}
to simulate \underline{Hypergeometric}
\begin{theorem}
    If $X$ is a hypergometric random distribution of $n$ independent trials, each with probability of success $p$,
    and if
    \begin{gather*}
        \frac{n}{N} < \tfrac{1}{10}\\
        n(1 - p) > 5
    \end{gather*}
    then the binomial random variable can be approximated by a normal distribution with 
    \begin{gather*}
        \mu = np\\
        \sigma = \sqrt{np(1 - p)}
    \end{gather*}
\end{theorem}
\newpage
\section{The wonders of the Solar System: Episode: 5}
\begin{itemize}
    \item After he sees the tubeworms at the bottom of the ocean... The underwater city is one of the most bizarre environments on our planet. It's built around a \textbf{Hydrothermal bent}, a volcanic opening in the Earth's crust that
    pumps out clouds of sulphurous chemicals water heated to nearly 300 Celsius. 
\end{itemize}

For life to exist, we only need three things:
\begin{itemize}
    \item right \textbf{chemistry} set. Human body is made up with 40 elements, but actually 96\% of human is only made of four of them, carbon, nitrogen, oxygen and hydrogen. 
    \item We need a \textbf{power source}. We need a battery, something to make a flow of electrons that powers the processes of life. Most life on Earth uses the power of the sun. 
    \item We need some kind of \textbf{medium} for life to play itself out in, for process to happen. On the Earth, the medium is \textbf{water}.
\end{itemize}

What is the \textbf{fundamental link} that is driving the search for life in our solar system?
\begin{itemize}
    \item The link between liquid water and life. 
\end{itemize}

For life to get a foothhold, you need more than that. You need areas of \textbf{standing water}.\\

So large deposites of gypsum on the surface of Mars tells you that there must have been big areas of \textbf{water}.\\

What gas has been detected in the atmosphere of Mars? \textbf{Methane}\\

Which of the Jupiter's moons has the greatest chance of finding life? \textbf{Europa}\\

If there lis life out there in the solar system, it will almost certainly be simple \textbf{single-celled} organisms like bacteria eking out an existence in the most hostile of environments. \\

There is only one world where the laws of phyiscs have conspired to combine all these features in one place. On Earth:
\begin{itemize}
    \item The temperature and atmosphere pressure are just right to allow oceans of liquid water to exist on the surface of the planet. 
    \item Magnetic
    \item allow that life to evolve into such complex creatures as ourselves requiers one more ingredient. And that's \textbf{time}
\end{itemize}
\chapter{Probability Distributions}
\newpage
\section{The Wonders of the Solar System}
\subsubsection{Q1}
The further a planet is from the sun, the \textbf{slower} it's speed, and the \textbf{longer} one revolution around the 
Sun takes. \\

\subsubsection{Q2}
The type of celestial object appears to change its position amongst the stars from night to night is called \textbf{planet}\\

\subsubsection{Q3}
When the Earth "overtakes" Mars in orbit, Mars appears to move \textbf{backward} against the backgrounds of the stars.\\

\subsubsection{Q4}
Everything in our solar system was formed from a \blue{nebula}, a giant \textbf{cloud} of gas and dust. \\

\subsubsection{Q5}
What type of event is thought to have disturbed the nebula and to have led to the formation of the solar system?\\
\begin{itemize}
    \item  A shock wave from a nearby supernova explosion. 
\end{itemize}

\subsubsection*{Q7}
The ring \textbf{nearest} Saturn are the fartest., just like planets orbiting the Sun. \\

\subsubsection{Q8}
Whater material makes up Saturn's rings? \textbf{Water ice}\\


\newpage
\section{Uniform Distributions}
\subsection{Different Distributions}
Distributions of data can be classified by considering the general shape of its graph. This picture is a table of distributions which is copied from 
the teacher's note

\begin{figure}[!h]
    \centering
    \includegraphics[width=0.9\textwidth]{pictures/5.2.1.png}
    \caption{Thanks to \textbf{Mr Tang}}
\end{figure}

\subsection{Characteristics of Uniform Distribution}
If an distribution is considered Uniform, it will have following characteristics:
\begin{enumerate}
    \item Each outcome is EQUALLY likely in any single trial of experiment
    \item If $X$ is discrete and $n$ is the number of possible outcomes in the probability experiment, then 
    \begin{gather*}
        P(X = x) = P(x) = \frac{1}{n}\\
        E(x) = \frac{\Sigma x_i}{n} \\
        \sigma = \sqrt{
            \frac{1}{n} \sum (x - E(x))^2
        }
    \end{gather*}
    \item If $X$ is continous with values in range from $a$ to $b$, then the expected value $E(x)$ will be $\frac{a + b}{2}$
\end{enumerate}

\begin{example}
    I want to discuss
\end{example}

\begin{proof}
    
\end{proof}
\newpage
\section{Kepler's Law of Planetory Motion}

\begin{figure}[h!]
    \centering
    \includegraphics[width=0.5\textwidth]{pictures/Kepler'sEllipse1.jpg}
\end{figure}

The sun is at one focus\\

There are two points in this diagram:
\begin{itemize}
    \item The point closet to the Sun is called \textbf{\textit{Perihelion}}. \textit{Peri} means \textit{near} in the Latin.
    \item The point farthest from the Sun is called \textbf{\textit{Aphelion}}. \textit{Ape} means \textit{far} in the Latin.
\end{itemize}

\subsection{Kepler's first law}
\begin{definition}
    Planet's orbit in ellipses with the Sun at one focus
\end{definition}

Ellipses can be classified based on their \textbf{eccentricity}

\[
    e = \frac{c}{a}
\]

\begin{center}
    e = Eccentricity \\
    c = Distance from centre to a focus (in m(or Au))\\
    a = Length of semi-major axis (in m(or Au))
\end{center}

The eccentricity of Earth's orbit is 0.02.\\

The most eccentric planetary orbit in our solar system is \textit{Mercury}, which has a eccentricity of 0.2.\\

Coments tend to the the largest eccentricity very close to 1.\\

Here I want to discuss about the meaning of eccentricity:
\begin{itemize}
    \item If an ellipse has an eccentricity of \textbf{0}, the object is orbit its sun in a \textbf{perfect circle}.
    \item If an ellipse has an eccentricity of \textbf{1}, the object is not in an \textbf{orbit}. 
\end{itemize}

\subsection{Kepler's second law}
\begin{definition}
    A line segment joining a planet and the sun \textbf{sweeps out equal areas in equal amount time}
\end{definition}

By the second law, we can make a conclusion. A planet moves fastest with it is at the \textbf{perihelion} and slowest when it is at the \textbf{aphelion}.

\subsection{Kepler's Third Law}
\begin{definition}
    The square of the orbital period of a planet directly proportional to the cube of the length 
    of the semi-major axis of its orbit
\end{definition}

\[
    p^2 = a^3
\]
\begin{center}
    $p$ = orbital period in (years)\\
    $a$ = Length of semi-major axis (in Au)
\end{center}

\begin{figure}[h!]
    \centering
    \includegraphics[width=0.7\textwidth]{pictures/Kepler'sLaw3.png}
    \caption{This is a log graph}
\end{figure}

The semi-major axis of an orbit is sometimes referred to as the \textbf{average} distance from the sun.
\newpage
\section{Approximate Hypergeometric}
If the population size $N$ is fairly large, and the sample size is relatively small, we can use a \underline{Normal Distribution}
to simulate \underline{Hypergeometric}
\begin{theorem}
    If $X$ is a hypergometric random distribution of $n$ independent trials, each with probability of success $p$,
    and if
    \begin{gather*}
        \frac{n}{N} < \tfrac{1}{10}\\
        n(1 - p) > 5
    \end{gather*}
    then the binomial random variable can be approximated by a normal distribution with 
    \begin{gather*}
        \mu = np\\
        \sigma = \sqrt{np(1 - p)}
    \end{gather*}
\end{theorem}
\newpage
\section{The wonders of the Solar System: Episode: 5}
\begin{itemize}
    \item After he sees the tubeworms at the bottom of the ocean... The underwater city is one of the most bizarre environments on our planet. It's built around a \textbf{Hydrothermal bent}, a volcanic opening in the Earth's crust that
    pumps out clouds of sulphurous chemicals water heated to nearly 300 Celsius. 
\end{itemize}

For life to exist, we only need three things:
\begin{itemize}
    \item right \textbf{chemistry} set. Human body is made up with 40 elements, but actually 96\% of human is only made of four of them, carbon, nitrogen, oxygen and hydrogen. 
    \item We need a \textbf{power source}. We need a battery, something to make a flow of electrons that powers the processes of life. Most life on Earth uses the power of the sun. 
    \item We need some kind of \textbf{medium} for life to play itself out in, for process to happen. On the Earth, the medium is \textbf{water}.
\end{itemize}

What is the \textbf{fundamental link} that is driving the search for life in our solar system?
\begin{itemize}
    \item The link between liquid water and life. 
\end{itemize}

For life to get a foothhold, you need more than that. You need areas of \textbf{standing water}.\\

So large deposites of gypsum on the surface of Mars tells you that there must have been big areas of \textbf{water}.\\

What gas has been detected in the atmosphere of Mars? \textbf{Methane}\\

Which of the Jupiter's moons has the greatest chance of finding life? \textbf{Europa}\\

If there lis life out there in the solar system, it will almost certainly be simple \textbf{single-celled} organisms like bacteria eking out an existence in the most hostile of environments. \\

There is only one world where the laws of phyiscs have conspired to combine all these features in one place. On Earth:
\begin{itemize}
    \item The temperature and atmosphere pressure are just right to allow oceans of liquid water to exist on the surface of the planet. 
    \item Magnetic
    \item allow that life to evolve into such complex creatures as ourselves requiers one more ingredient. And that's \textbf{time}
\end{itemize}

% \chapter{Appendax}
% \section{Slope}
% \section{Review of Electronstatics}
In this section, we will briefly review basic Electronstatics which we learned from Grade 9 Science

\subsection{Electric Charge}

\subsubsection{Electron}
By the early 1900s, physicists had identified the subatomic particles called the electron and the proton
as the basic units of charge. All protons carry the same amount of positive charge, $e$, and all electrons 
carry an equal but opposite charge, $-e$. Charges interact with each other in very specific ways governed by the 
\textbf{law of electric charges}

\begin{theorem}[Law of Electric Charges]
    Like charges repel each other; unlike charges attract.
\end{theorem}

\noindent\hrulefill

\subsubsection{Charge of atom}
\begin{center}
    Cation: a positive ion. $\#$ of protons $> \#$ of electrons\\
    Anion: a negative ion. $\#$ of protons $< \#$ of electrons\\
\end{center}

The \textbf{Total Charge} is the sum of all the charges in that object and can be positive, negative or zero. The charge is equal to zero when the 
negative charge equals to negative charge. 

\begin{theorem}[Law of Conservation of Charge]
    Charge can be transferred from one object to another, but the total charge of a closed system remains constant. 
\end{theorem}
\noindent\hrulefill

\subsubsection{Coulomb}
The basic unit of charge is called the coulomb (C). The charge of electron, $-e$, is $-1.60*10^{-19}C$, and the charge of a single proton, $+e$, is 
$1.60*10^{-19}C$\\

Symbol $e$ often donotes the magnitude of the charge of an electron or a proton. \\

The symbol $q$ denotes teh amount of charge, such as the total charge onf a small piece of paper. In other words, the total charge of a particle 
is $q$.

% \noindent\hrulefill

\subsection{Conductors and Insulators}
\begin{definition}[Conductor]
    A conductor is a substance in which electrons can move easily among atoms.
\end{definition}

\begin{definition}[Insulator]
    any substance in which electrons are not free to move easily from one atom to another.
\end{definition}

Insulator hold the electron when other electron come in. There are no free electrons in the insulator, and insulator does not allow the extra electrons 
to move about easily. 

% \noindent\hrulefill

\subsection{Different methods of charging}
\subsubsection{Charging an Object by Friction}
In reality, some object has stronger ability to hold on electrons than others. Assume we have two neutral object, when these two objects touch, 
electrons will follow from the object with weaker hold on electrons to the other one with stronger hold on electrons. 

% \noindent\hrulefill

\subsubsection{Carging an object by Induced Charge Separation}
Assume we have two objects, one with zero charge and other one with negative charge. When we put the negative object towards the positive object, 
electrons in the neutral object will repel to the electrons in the negative object. As a result, electrons in the neutral object will redistributed 
throughout the material. The positive side of the netural object is closer than the negative side of the object, in which makes the neutral object 
attrack to the negative object. 

% \noindent\hrulefill
\subsubsection{Charging by Contact}

\begin{figure}[!h]
    \centering
    \includegraphics[width=0.7\textwidth]{pictures/4.1.1.png}
    \caption{Picture from my textbook}
\end{figure}

\subsubsection{Charging by Induction}
\begin{center}
    Using a negative object to create a positive object
\end{center}

\begin{figure}[!h]
    \centering
    \includegraphics[width=0.7\textwidth]{pictures/4.1.2.png}
    \caption{Picture from my textbook}
\end{figure}
% \newpage
% \section{Equations of Motion}
To start off, there are five equations that are used in the calculation of motion
\[
    \vec{v_{f}} = \vec{v_{i}} + \vec{a}*\Delta t
\]
\[
    \Delta\vec{d} = \frac{1}{2}(\vec{v_{i}} + \vec{v_{f}}) * \Delta t
\]
\[
    \Delta\vec{d} = \vec{v_{i}}\Delta t + \frac{1}{2}\vec{a} * \Delta t^2
\]
\[
    \Delta\vec{d} = \vec{v_{f}}\Delta t - \frac{1}{2}\vec{a} * \Delta t^2
\]
\[
    \vec{v_{f}}^2 = \vec{v_{i}}^2 + 2 \vec{a} \Delta \vec{d} 
\]

\subsection{Format requirements for answering Motion questionss}
\begin{enumerate}
    \item You should always include a diagram that contains every known information from the question
    \item $\vec{v}$ should be presented for at least two decimal places
    \item Always list steps in your answer
    \item When using quadratic solving function on calculator, always write \textit{*Using Quadratic Eq} in your answer
    \item When you form an equation system, you should label $1. 2. 3.$ on each equation in the system
    \item Follow the Sig Digit rules
    \begin{itemize}
        \item[!] For \textbf{add} and \textbf{subtract}, keep the least \textbf{decimal places}
        \item[!] For \textbf{multiply} and \textbf{divide}, keep the least amount of \textbf{signficiant digit} 
    \end{itemize}
\end{enumerate}
% \newpage
\section{Continuity and Limits}
\section{Electric Fields}
\begin{definition}[Field]
    The region where an appropriate object would feel a force!
\end{definition}
\begin{itemize}
    \item If there's a gravitational field, a \underline{mass} will feel a force. 
    \item If there's an electric field, a \underline{charge} will feel a force.
    \item If there is an magnetic field, a \underline{magnet} (or a moving charge) will feel a force.
\end{itemize}

Visualizing Electric Fields - Field lines show how a small \underline{\textit{positive}} charge would move.

\begin{figure}[!h]
    \centering
    \includegraphics[width=0.5\textwidth]{pictures/4.3.1.png}
    \caption{Electric field of Positive and negative charge}
\end{figure}

\begin{figure}[!h]
    \centering
    \includegraphics[width=0.5\textwidth]{pictures/4.3.3.png}
    \caption{Electric field between two charges}
\end{figure}

\subsubsection{Parallel Plates}
Two charged metal plates that are parallel to each other $\rightarrow$ "parallel plates"
\begin{itemize}
    \item The field strength outside of the plates is very weak and can be considered negligible
    \item The field lines in between the plates are equdistant stand. 
\end{itemize} 

\subsubsection{Formulas}
\begin{equation}
    \mathcal{E} = \frac{k \left | q \right |}{R^2}
\end{equation}
\begin{center}
    $\mathcal{E}$ is the magnitude of the \underline{electric field strength} around a point charge (in $\tfrac{N}{C}$)\\
    $k = 8.99 \times 10^9 \tfrac{Nm^2}{C^2}$\\
    $R$ is the distance away from the point charge (q) where you want to know the field strength (in $m$)
\end{center}

\noindent\hrulefill
\begin{remark}
    Electric fields are \underline{vector}. The direction of the field will be based on the direction of force that would be 
    exerted on a \underline{positively-charged} object!
\end{remark}
\noindent\hrulefill

\begin{equation}
    \vec{F_E} = q \times \vec{\mathcal{E}}
\end{equation}

\begin{center}
    $\vec{F_E}$ is the magnitude of the electrical force exerted on $q$ (in $N$)\\
    $q$ is the that is \underline{in the electric field}(in $C$)\\
    $\vec{\mathcal{E}}$ is the strength of the electrical field that the charge is in (in $\frac{N}{C}$)
\end{center}
\newpage
\subsection{Continuity and Limits}
\begin{definition}
    [Limits Law]
    Assume we have function $f(x)$ and $g(x)$, which 
    \[
        \lim_{x \to a} f(x) \text{ and } \lim_{x \to a} g(x) \text{ both exist}
    \]
    We also have constant $c$
    We get these properties:
    \begin{enumerate}
        \item \[\lim_{x\to a} c = c\]
        \item \[\lim_{x \to a} x = a\]
        \item \[
            \lim_{x \to a} \left[f(x) + g(x) \right] = \lim_{x \to a} f(x) + \lim_{x \to a} g(x)
        \]
        \item \[
            \lim_{x \to a} \left[f(x) - g(x) \right] = \lim_{x \to a} f(x) - \lim_{x \to a} g(x)
        \]
        \item \[
            \lim_{x \to a} \left[f(x) * g(x) \right] = \lim_{x \to a} f(x) * \lim_{x \to a} g(x)
        \]
        \item \[
            \lim_{x \to a}\frac{f(x)}{g(x)} = \frac{
                \lim_{x \to a} f(x) 
            }{
                \lim_{x \to a} g(x)
            }  
        \]
        \item \[
        \lim_{x \to a} \left[f(x)\right]^n = (\lim_{x \to a}f(x))^n 
        \]
        \item \[
        \lim_{x \to a} \sqrt[n]{f(x)} = \sqrt[n]{(\lim_{x \to a}f(x))} 
        \]
    \end{enumerate}
\end{definition}

Somes cases you should be careful
\begin{example}
    Evaluate and simplify
    \[
        \lim_{x \to a} \sqrt{x}
    \]
    Recall: if a limit exist
    \begin{enumerate}
        \item \[ \lim_{x \to a^{-}} f(x) \text{ exists}\]
        \item \[ \lim_{x \to a^{+}} f(x) \text{ exists}\]
        \item \[ \lim_{x \to a^{-}} f(x) = \lim_{x \to a^{+}} f(x)\]
    \end{enumerate}
    In this case, \[ \lim_{x \to 0^{-}} \sqrt{x} \text{ does not exist}\] because 
    \[
        D: {x \in \mathcal{R} | x \geq 0}
    \]
\end{example}

\begin{example}
    Evaluate and simplify: 
    \[
        \lim_{x \to 1} x^2 - 5x
    \]
    \begin{remark}
        Please be carefull with the bracket, don't mix with 
        \[
            \lim_{x \to 1} (x^2 - 5x)
        \]
    \end{remark}
    The solution should be:
    \begin{gather*}
        = (\lim_{x \to 1}x)^2 - 5x \\
        = (1)^2 - 5x\\
        = 1 - 5x
    \end{gather*}
\end{example}


\end{document}
