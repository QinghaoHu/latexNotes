\section{Wave Properties of Classical Particles}
\subsection{Wave-like Properties of Classical Particles}
In 1924, Louis de Broglie first suggested that all classical particles have wave-like Properties.
In the previous section, you read that a photon has a momentum given by 
\[
    P_{photon} = \frac{h}{\lambda}
\]
De Broglie turned this result around and hypothesized that a particle with momentum $p$ has a wavelength of
\begin{equation}
    \lambda = \frac{h}{p}
\end{equation}

\begin{definition}
    [de Broglie wavelength] The wavelength associated with the motion of a particle possessing momentum of magnitude $p$
\end{definition}

If a particle has a wavelength, the particle should exhibit interference just as wave do. \\

According to his equation, a longer wavelength means that the value for momentum has to be small. 

\begin{definition}
    [Matter wave] The wave-like behaviour of particles with mass
\end{definition}

In 1927, physicists Clinton Davisson and Lester Germer performed an experiment in which they aimed a beam of electrons at a crystal target. 
the atoms in the target were space at regular intervals, acting as a series of slits for the electrons. 
Just as with the diffraction of light, the Davisson-Germer experiment exhibts interference when the wavelength of electrons is similar to the spacing between the atoms in the crystal. 
The diffraction technique used in the Davisson=Germer experiment is still used today as a way to measure molecule spacing within a crystal. 

It is possible to determine the de Broglie wavelength of larger objects, such as baseballs. 
However, the momentum of large objects tends to be so large that it implies an incredibly small wavelength.
That is why we are unable to see the interference of these objects. 


\subsection{Interpreting the Double-Slit Experiment}
\subsubsection{Collapse Interpretation}
The electron leaves its source behaving as a particle, but then it spreads out and travels as a wave until it is measured at the screen. \\

This law claims that an electron physically changes from a particle to a wave and back again. 
These two behaviours and the physical laws that go with them alternative in a way that is not predicted by quantum mechanics. 
\begin{figure}
    [!h]
    \centering
    \includegraphics[width=9cm]{pictures/2.3.1.png}
\end{figure}

\newpage
\subsubsection{Pilot Wave Interpretation}
The electron is just a simple particle whose motion is described by a single law. 
The motion of the electron depends on a mysterious pilot wave. 
To obtain the interference pattern in the double-slit experiment, the behaviour of the pilot wave must depend on everything everywhere in the universe, 
including future events. 
For example, the pilot wave "knows" whether one or two slits are open, and whether or not a detector is turned on at the screen. 
\begin{figure}
    [!h]
    \centering
    \includegraphics[width=9cm]{pictures/2.3.2.png}
\end{figure}

\subsubsection{Many Worlds Interpretation}
A parallel universe exists for each of the electron's possible states. 
The Universe constantly splits into many versions of itself. 

\subsubsection{Copenhagen Interpretation}
This interpretation view interprets the physical laws in terms of information about actual measurement made on a quantum-mechanical system. 
Certain questions do not have answers, such as what electrons are "doing" as they travel to the detection screen. 
You can only ask what the results will be if you do a certain experiment. 

\subsection{The Wave Function: A mathematical Description of Wave-Particle Duality}
A wave function gives the probability for a particle to take any possible path, or for the particle to show up at any possible location 
on the detection screen in the double-slit experiment. 

Wave fuction can be calculated by an equation from Erwin Schrodinger. 
Researchers use the Schrodinger equation to determine the wave function and how it varies with time. 

Consider an electron confined to a particular region of space. A classical particle moving around inside the box would simply travel back and forth, 
bouncing from one wall to another. 
The wave function for a particle-wave inside this box is described by standing waves, similar to those you would see on a string. 
\begin{figure}
    [!h]
    \centering
    \includegraphics[width=11cm]{pictures/2.3.3.png}
\end{figure}

The figure 4(b) shows two possible wave function solutions corresponding to electrons with different kinetic energies. 
The wavelengths of these standing waves are different, since the wavelength of an electron depends its kinetic energy. 
In the case of mechanical waves in classical mechanics, these two solutions correspond to two standing waves with different wavelength.

After get the wave function for a particular situation, such as for the electron in (b), you can try to calculate the position and speed of the electron. 
However, the results do not give a simple single value for x. The probability of finding the electron at certain value of x is large 
in some regions and small in others. The probability distribution is different for each wave function. 

\subsection{The Heisenberg Uncertainty Principle}
\begin{theorem}
    [Heisenberg Uncertainty Principle] A mathematical statement that says that if $\Delta x$ is the uncertainty in a particle's position, and 
    $\Delta p$ is the uncertainty in its momentum, then 
    \[
        \Delta x\Delta p \ge \frac{h}{4\pi}
    \]
    where $h$ is Planck's constant
\end{theorem}
This principle says that there is a limit to how accurately simultaneous measurements of the position and momentum of a quantum object can be. 

If you measure the position of a quantum object with great accuracy, then you can only measure its momentum with little accuracy. 

The act of measuring the system itself disturbs the system. 