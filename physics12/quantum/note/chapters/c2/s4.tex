\section{The Standard Model of Elementary Particles}
\subsection{\textcolor{blue}{Ernest Rutherford}}
In 1909, Ernest Rutherford and his students did an experiment. 
They aimed high-speed, positively charged particles at a thin sheet of gold foil. 
Under the traditional theory, Rutherford and his team expected most of the particles to pass through the foil. 

Instead, they discovered that a small number of particles were deflected. 
Rutherford realized that this result meant that all the positive charge in an atom must be concerntrated in a very small volume. 

Rutherford suggested that the atom is like a miniature solar system, with electrons orbiting the nucleus just as planets orbit the sun. 
The model asl proposed that they could move in any orbit. 

The strong force hold the protons together. 
\begin{figure}
    [!h]
    \centering
    \includegraphics[width=5cm]{pictures/2.4.1.png}
\end{figure}

The electrons must move in orbits to avoid "\textit{falling}" into the nucleus as a result of the electric force. 

\subsection{Problems with the Planetary Model}
Maxwell's classical theory of electromagnetic predicts that an electron emits electromagnetic radiation when it orbits a proton. 
The radiation carries away energy. 
If the electron in a hydrogen atom loses energy in this way, Newtonian mechanics predicts that it will spiral inward to the nucleus. 

If this model were correct, all atoms would collapse, which is not the case. 
Physicists were unable to modify the planetary model to make the atoms stable. 
\begin{figure}
    [!h]
    \centering
    \includegraphics[width=4cm]{pictures/2.4.2.png}
\end{figure}


\subsection{The Bohr Model of the Atom}
Bohr proposed a quantum-mechanical approach to the motion of electrons within the atom. 
He was inspired by the Planck-Einstein introduction of quanta into the theory of electromagnetic radiation. 
Bohr's theory went against the well-established classical laws of mechanics and electromagnetism. 

Bohr proposed that an electron in an atom can have only certain orbits with particular values for the radius of each orbit. 
The special values of the orbital radius meant that the electron could only have special values of potential energy and kinetic energy. 
The total energy could take on only certain discrete, quantized values. 
Each value of energy corresponds to what is now called an energy level. 

Bohr's model was partially successful. It provided a physical model of the hydrogen atom.
The model matched the internal energy levels to the levels observed in a hydrogen spectrum. 
At the same time, the model accounted for the stability of the hydrogen atom. 

However, Bohr's model, was incomplete. 
When applied to atoms with many electrons, the model broke down. 

An explanation for the BOhr model came from de Broglie's matter waves, which were developed 10 years after Bohr's work. 
As mentioned in previous chapter, the electrons in a given energy level have special values of kinetic energy. 
Therefore, those electrons have certain values of momentum. 

Using de Broglie's model, the allowed electron orbits in hydrogen correspond exactly to those orbits in which electron waves from circular standing waves around the nucleus. 

\begin{figure}
    [!h]
    \centering
    \includegraphics[width=13cm]{pictures/2.4.3.png}
\end{figure}

\subsection{Twenty-First-Century Physics and Antimatter}
\begin{definition}
    [Antimatter] a form of matter in which each particle has the same mass and an opposite charge as its counterpart in ordinary matter
\end{definition}

Two examples of antimatter are 
\begin{center}
    Anti-protons\\
    Anti-neutrons
\end{center}

Although the neutron and anti-neutron are both neutral, they are different particles. 
Neutron is made up of a certain combination of quarks, and the anti-neutron is made up of the corresponding combination of anti-quarks

\begin{figure}
    [!h]
    \centering
    \includegraphics[width=13cm]{pictures/2.4.4.png}
\end{figure}

Anti-particles offer researchers a chance to see special relativity at work at the microscopic level. 
For example, when an electron encounters its anti-particle, the positron, the two undergo a reaction that destroys both particles. 

\subsection{The Standard Model}
\begin{definition}
    [Quark] an elementary particle that makes up protons, neutrons, and other hadrons. \\
    Was first discovered in collision experiments involving protons
\end{definition}

\begin{definition}
    [Hadrons]
    A class of particles that contains the neutron, the proton, and the pion; composed of combinations of quarks and anti-quarks
    \begin{itemize}
        \item Hadrons composed of three quarks are called baryons. 
        \item A quark and an anti-quark can also combine to form a particle. Hadrons composed of just two quarks are called mesons. 
    \end{itemize}
\end{definition}

\begin{definition}
    [Leptons] A class of particles that includes the electron, the muon, the tauon, and the three types of neutrinos; not composed of smaller particles
\end{definition}

\begin{definition}
    [Standard model] the modern theory of fundamental particles and their interactions
\end{definition}

When a high-energy electron collides with a proton, the way that the electron scatters (that is, its outgoing direction and energy) gives information about how mass and charge are distributed inside the proton. 

All hadrons are composed of quarks, so the interactions between quarks determine the properties of hadrons and how they relate to one another. 
The two most important hadrons are the proton and the neutron, so the behaviour of quarks also determines the properties of nuclei. 

Quarks are charged, so they act on each other through the electric force. They also interact through the strong force mentioned earlier. 
Quarks bind together to form protons and neutrons (nucleons), and the strong force is responsible for holding protons and neutrons together to make nuclei. 

\subsubsection{Leptons}
Leptons are naturally grouped into three pairs: 
\begin{itemize}
    \item Electron and Electron neutrino
    \item Muon and Muon neutrino 
    \item Tau and Tau Neutrino
\end{itemize}

The muon and the tau are not stable. Electrons are stable, but the behaviour of neutrinos is more complicated. 
Neutrinos travel through space, they change from one to another of the three types of neutrinos, an effect called neutrino oscillation. 

\subsubsection{Bosons: Force Mediating Particles}
The fundamental forces of nature are the ways in which individual particles interact with each other. 
Every interaction in the universe can be described using only three forces. 
\begin{itemize}
    \item Strong nuclear force
    \item Weak nuclear force
    \item electromagnetism
\end{itemize}

The strong nuclear force holds the subatomic particles of the nucleus together. \\

The weak nuclear force causes radioactive decay and starts the process of hydrogen fusion and other nuclear process in star\\

The electromagnetic force is responsible for the attraction and repulsion among electrical charges. 

\begin{definition}
    [Fermion] A fundamental particle that forms matter\\
    Including quark and leptons
\end{definition}

\begin{definition}
    [Bosons] The particle responsible for transmitting electromagnetic, strong, and weak force. 
\end{definition}

From this view, 
\begin{itemize}
    \item The electromagnetism force is "mediated" by the photon, which means that photons transmit the electromagnetic force acting on charged particles. 
    \item The strong nuclear force is mediated between quarks by particles called Gluons.  Eight different types of gluons exist, and the force exerted between two quarks depends on type of each quark. 
    \item The weak nuclear force is mediated by a family of three particles called the $W^{+}, W^{-} \text{ and } Z$ bosons. 
    Unlike the photon and gluons, which have zero mass, the $W^{+}, W^{-} \text{ and } Z$ bosons have mass. 
\end{itemize}

According to the standard model, another boson, the Higgs boson, exists.
\begin{definition}
    [Higgs Bosons]
    the theoretical particle thought to play a role in giving mass to other particles. 
\end{definition}