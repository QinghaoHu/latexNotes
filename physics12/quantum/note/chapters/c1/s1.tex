\section{The Special Theory of Relativity}
At the turn of the twentieth century, most of the physics community enjoyed a sense of accomplishment. 
Newtonian mechanics, provided the principles to the atomic level and established the idea of energy conservation. 
Maxwell's equation successfully unified the subjects of electricity, magnetism and optics. 
Light is a combination of oscillating electric and magnetic fields. 

However, few famous physics experiments came to a similar statement. The speed of light is $3.0 \times 10^8\tfrac{m}{s}$ at whatever frame of reference.   
Einstein developed the \underline{\textit{Special Theory of Relativity}} to explain why the speed of light is constant at different frames of reference. \\


\subsection{Maxwell's assumption}
He proposed that \textit{electromagnetic wave} needed a medium through which to travel. 
The medium was called \underline{\textbf{ether}}. 
\begin{definition}
    [ether] the proposed medium through which electromagnetic wave were once believed to propagate.\\
    Few properties of ether: 
    \begin{itemize}
        \item [=] It had no mass 
        \item [=] It had no drag effect on the motions of the planets. 
        \item [=] It filled the vacuum of space. 
    \end{itemize} 
\end{definition}

Maxwell stated that speed of light with respect to the ether will always be $c = 3.0 \times 10^8 \tfrac{m}{s}$
That meant that if you were at rest and observed a light source move relative to ether, you would measure the speed of light to be different from $c$.
It means that speed of light is different from different perspective. 

\subsubsection{\textit{Failure}}
Experimental evidence did not support the idea that the speed of light varied with the speed of the inertial frame. 
No change in the speed of light with the motion of Earth. 
Experiments proved that electromagnetic waves do not require a medium in which to propagate, and the existence of ether could not be proven experimentally. \\

In the magnet-and-coil thought experiment, Maxwell's theory predicts that when a magnet moves toward a coil of wire, an electric field forms near the moving magnet. 
This electric field moves charges with the coil, thus inducing an electric current. 
However, Maxwell's theory also predicts that if the coil moves and the magnet remains at rest, a current exists in the coil, not because an electric field forms, but because the magnetic field exerts a force on the charges in the moving coil. 

To einstein, it seemed illogical that selecting a frame of reference in which either the magnet or the coil is at rest would change the way we understand what is happening. 


\subsection{The Special theory of relativity}
The \underline{\textit{Special Theory of Relativity}} has two postulate:
\begin{postulate} [The Principle of Relativity]
    The laws of physics are the same in all inertial frames of reference. No physics experiment can ever determine whether you are at rest or moving at a constant velocity. 
\end{postulate}

\begin{postulate}
    [The Speed of Light Principle]There is at least one inertial frame of reference in which, for an observer
    at rest in this frame of reference, the speed of light, $c$, in a vacuum is independent of the motion of the source of the light. 
\end{postulate}

Postulate 1 implies that if Postulate 2 is true in one inertial frame of reference, it must be true in all frames of reference. 

The consequence of this is that the speed of light must be constant and the same in all inertial frame of reference because 
the laws of physics do not prefer one frame of reference over another. 

\begin{theorem}
    [Special Theory of Relativity] 
    All physical laws are the same in all inertial frames of reference, and the speed of light is independent of the motion of the light source 
    or its observer in all inertial frames of reference. 
\end{theorem}
