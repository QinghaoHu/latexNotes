\section{The Special Theory of Relativity}
At the beginning of 1900s, few famous physics experiments came to a similar statement. The speed of light is $3.0 \times 10^8\tfrac{m}{s}$ at whatever frame of reference.   
Einstein developed the \underline{\textit{Special Theory of Relativity}} to explain why the speed of light is constant at different frames of reference. \\

The \underline{\textit{Special Theory of Relativity}} has two postulate:
\begin{postulate} [The Principle of Relativity]
    The laws of physics are the same in all inertial frames of reference. No physics experiment can ever determine whether you are at rest or moving at a constant velocity. 
\end{postulate}

\begin{postulate}
    [The Speed of Light Principle]There is at least one inertial frame of reference in which, for an observer
    at rest in this frame of reference, the speed of light, $c$, in a vacuum is independent of the motion of the source of the light. 
\end{postulate}

\noindent\hrulefill

The consequence of this is that the speed of light must be constant and the same in all inertial frame of reference because 
the laws of physics do not prefer one frame of reference over another. 

\begin{theorem}
    [Special Theory of Relativity] 
    All physical laws are the same in all inertial frames of reference, and the speed of light is independent of the motion of the light source 
    or its observer in all inertial frames of reference. 
\end{theorem}