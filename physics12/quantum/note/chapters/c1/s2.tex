\section{Time Dilation}
\begin{remark}
    Please go to Mr. Yang's video \href{https://www.youtube.com/watch?v=RtzFzYz0K1E&list=PLFXBrCJVstjiXNKHnM6E7zShEnUiG_zJP&index=1&t=1377s}{here}!
    Here is just a brief personal review note about \textit{time dilation} in that video.\\
    Please check the textbook first, this note is only a summary! 
\end{remark}

\underline{\textit{Time Dilation}}, the model explains the slowing down of time in one reference frame moving relative to an observer in another 
reference frame. 

\subsubsection{Proper time vs non-proper time measurement}
\begin{figure}[!h]
    \centering
    \includegraphics[width=0.4\textwidth]{pictures/1.2.4.png}
\end{figure}

\begin{definition}
    [Synchroinzed Clock Measurement/Proper time]
    A people measures a time at the same location/coordinate from his frame of reference!
\end{definition}

\begin{definition}
    [Moving Clock Measurement/Non-proper time measurement]
    A people measures a time at the different location/coordinate from his frame of reference!
\end{definition}

\subsubsection{$\Delta t$ vs $\Delta t_s$}
$\Delta t_s$ is a Synchronized clock measurement of the time from observer 1's frame of reference. $\Delta t$ is the moving clock measurement from a stationary perspective(Observer 2). 

\begin{figure}[!h]
    \centering
    \includegraphics[width=0.2\textwidth]{pictures/1.2.1.png}
    \caption{From the inertial FOR which has recorded $\Delta t_s$}
\end{figure}

\begin{figure}[!h]
    \centering
    \includegraphics[width=0.4\textwidth]{pictures/1.2.2.png}
    \caption{From the inertial FOR which has recorded $\Delta t$}
\end{figure}

\begin{figure}[!h]
    \centering
    \includegraphics[width=0.4\textwidth]{pictures/1.2.3.png}
    \caption{Path of light from observer 2's FOR}
\end{figure}

\subsubsection{Derivation of Special Relativity equation}
\begin{lemma}
The time can be determined by this formula: 
    \begin{equation*}
        \Delta t = \frac{\text{distance}}{\text{speed}}
    \end{equation*}
\end{lemma}

\begin{proof}
    To start off, we need to solve for $z$:
    \begin{gather}
        z = \sqrt{d^2 + (\frac{v\Delta t}{2})^2} \label{eq1.2.1}
    \end{gather}
    Then, let's solve for $\Delta t$
    \begin{gather*}
        \Delta t = \frac{2z}{c}\\
        \Delta t = \frac{2}{c} \times \sqrt{d^2 + (\frac{v\Delta t}{2})^2}\\
        (\Delta t)^2 = \frac{4}{c^2} \times (d^2 + (\frac{v\Delta t}{2})^2)\\
        \Delta t^2 = (\frac{2d}{c})^2 + (\frac{v\Delta t}{c})^2\\
        \Delta t^2 = (\Delta t_s)^2 + (\frac{v\Delta t}{c})^2\\
        \Delta t^2  - \frac{v^2\Delta t^2}{c^2} = (\Delta t_s)^2\\
        \Delta t^2(1 - \frac{v^2}{c^2}) = (\Delta t_s)^2\\
        \Delta t^2 = \frac{(\Delta t_s)^2}{1 - (\frac{v^2}{c^2})}\\
        \Delta t = \frac{\Delta t_s}{\sqrt{1 - \frac{v^2}{c^2}}}
    \end{gather*}
\end{proof}

According to the theory of special relativity, $\Delta t > \Delta t_s$. As a result, 
\begin{equation*}
     \frac{1}{\sqrt{1 - \frac{v^2}{c^2}}} \ge 1   
\end{equation*}

\subsubsection{Lorentz factor $\gamma$}
\begin{equation*}
    \gamma = \frac{1}{\sqrt{1 - \frac{v^2}{c^2}}}
\end{equation*}

If we look at the factor of the $\gamma$, it only matters when the speed of the rocket is fast enoguh, or 
at least $0.3c$!