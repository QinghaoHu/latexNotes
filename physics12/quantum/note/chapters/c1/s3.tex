\section{Length Contraction, Simultaneity, and Relativistic Momentum}
\subsection{Length Contraction}
In the previous section, we discussed about time dilation. In the same paper, Herr Einstein stated that the length of a
moving object is different from this object's moving frame and stationary rest frame. 

\begin{definition}
    [proper length($L_s$)] the length of an object or distance between two points as measured 
    by an observer who is stationary relative to the object or distance
\end{definition}

\begin{example}
    Assume we want to measure the length of the car from observer 1 's perspective and observer 2's perspective.
    \begin{figure}
        [!h]
        \centering
        \includegraphics[width=0.3\textwidth]{pictures/1.3.1.png}
    \end{figure}

    \subsubsection{How can we solve this little problem}
    \noindent\hrulefill
    \begin{lemma} \label{lemma 1.3.1}
        Length = speed $\times$ time
        \begin{equation*}
            L = v\Delta t 
        \end{equation*}
    \end{lemma}
    \noindent\hrulefill

    From \ref{lemma 1.3.1}, we understand that the length of the object can be easily find by calculating 
    the time used by the object to pass point B.\\
    
    From observer 1's perspective, the time that this object pass point B is $\Delta t$ (measure at front and back 
    accroding to his FOR, so this is a non-proper measurement)\\

    From observer 2's perspective, the time that this object pass point B is $\Delta t_s$ (time measure at the same 
    point relative to the observer 2)\\

    As a result,
    \begin{equation*}
        L_s = v\Delta t
    \end{equation*}
    and 
    \begin{equation*}
        L = v \Delta t_s
    \end{equation*}
    Let's start from the time dilation formula:
    \begin{gather*}
        \Delta t = \frac{\Delta t_s}{\sqrt{1 - \frac{v^2}{c^2}}}\\
        v\Delta t = \frac{v\Delta t_s}{\sqrt{1 - \frac{v^2}{c^2}}}\\
         L_s = L\times \frac{1}{\sqrt{1 - \frac{v^2}{c^2}}}
    \end{gather*}
    \begin{equation}
         L = L_s \sqrt{1 - \frac{v^2}{c^2}}
    \end{equation}
\end{example}

\subsubsection{Muons and Evidence for Length Contraction and Time Dilation}
Muons:
\begin{itemize}
    \item Travel at speeds of about $0.99c$
    \item About 207 times as massive as electrons
    \item Decay in $2.2ms$
    \item Source: Cosmic radiation that collides with atoms in Earth's upper atmosphere
    \item Should decay after 660m, but in reality, 4800m
    \item Use special reality to explain this
    \item Due to time dilation, the clock of the Muons run relative slow. 
\end{itemize}

\subsection{Relativity of Simultaneity}
It's really hard to explain it here. I suggest you to read the textbook first!
\begin{figure}
    [!h]
    \centering
    \includegraphics[width=0.7\textwidth]{pictures/1.3.2.png}
\end{figure}

\textbf{\textit{Will observer 1 see both lightning at the same time?}}
\textbf{\textcolor{red}{NO}}\\

Observer 2 observes that the railway car moves to the right, and because observer 1 is 
moving, the flash at B will reach him before the flash from A. Even with the distortions 
of time and space that arise from relativity, events do not occur out of sequence.
So observer 1 will see the lightning strike at B before the lightning strike at A. The
speeds of the light pulses from A and B are the same (a consequence of Einstein’s pos-
tulates), and the distances that the pulses travel are the same. Therefore, observer 1
must conclude that the light pulses were not emitted at the same time.\\

Let's assume clock there are three clocks, A, B and C.\\

Clock A, B are stationary at point A and B. Observer 1 hold the clock C\\

From stationary frame of reference, clock c elapse 10 seconds, while 20s elapse on A and B.\\

From observer 1's frame of reference, clock c elapse 10 seconds, while 5s elapse on A and B.\\

Where did the "extra" 15 seconds go? The answer is that while clocks A and B are synchroinzed in their frame of reference, they are not synchroinzed in the 
frame of reference of C. \\

When C get to A, the reading on A should be 0 seconds, B should be 15 seconds!

\subsubsection{The Twin Paradox}
The rock one is not inertial frame of reference. It has acceleration!

\subsection{Relativistic Momentum}
In Newtonian physics, momentum can be calculated as this, $\vec{p} = m\vec{v}$\\

However, as $v$ approaches the speed of light, we have to take special relativity into account. 
\begin{figure}
    [!h]
    \centering
    \includegraphics[width=0.3\textwidth]{pictures/1.3.3.png}
\end{figure}

The effects of time dilation and length contraction are not included in the Newtonian momentum used in classical mechanics. To account for the relativistic 
effects on the momentum of objects moving near the speed of light, Einstein showed that proper time should be used to calculate momentum. 
This amounts to using a clock that travels along with the object. At the same time, an observer who watches the object moving with speed $v$ should 
take the measurement of length. The proper time is given by this expression: $\Delta t_s = \Delta t_m \sqrt{1 - \frac{v^2}{c^2}}$.\\

Let us derive for the formula:
\begin{gather*}
    \vec{p} = m_s\vec{v}\\
    \vec{p} = m_s\frac{\Delta x}{\Delta t_s}\\
    p = \frac{m_s\Delta x}{\Delta t \sqrt{1 - \frac{v^2}{c^2}}}    
\end{gather*}
\begin{equation}
    p = \frac{m_sv}{\sqrt{1 - \frac{v^2}{c^2}}}
\end{equation}

\begin{definition}
    [Rest mass] the mass of the object are measured at rest with respect to the observer, also called a proper mass. In the equation, $m_s$ is a proper mass. 
\end{definition}

\begin{definition}
    [Relativistic mass] the mass of an object measured by an observer moving with speed v with respect to the object. 
\end{definition}

\textbf{Relative mass proper mass equation}
\begin{equation}
    m = m_s \frac{1}{\sqrt{1 - \frac{v^2}{c^2}}}
\end{equation}