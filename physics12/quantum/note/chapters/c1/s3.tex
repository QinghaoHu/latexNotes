\section{Length Contraction, Simultaneity, and Relativistic Momentum}
\subsection{Length Contraction}
In the previous section, we discussed about time dilation. In the same paper, Herr Einstein stated that the length of a
moving object is different from this object's moving frame and stationary rest frame. 

\begin{definition}
    [proper length($L_s$)] the length of an object or distance between two points as measured 
    by an observer who is stationary relative to the object or distance
\end{definition}

\begin{example}
    Assume we want to measure the length of the car from observer 1 's perspective and observer 2's perspective.
    \begin{figure}
        [!h]
        \centering
        \includegraphics[width=0.3\textwidth]{pictures/1.3.1.png}
    \end{figure}

    \subsubsection{How can we solve this little problem}
    \noindent\hrulefill
    \begin{lemma} \label{lemma 1.3.1}
        Length = speed $\times$ time
        \begin{equation*}
            L = v\Delta t 
        \end{equation*}
    \end{lemma}
    \noindent\hrulefill

    From \ref{lemma 1.3.1}, we understand that the length of the object can be easily find by calculating 
    the time used by the object to pass point B.\\
    
    From observer 1's perspective, the time that this object pass point B is $\Delta t$ (measure at front and back 
    accroding to his FOR, so this is a non-proper measurement)\\

    From observer 2's perspective, the time that this object pass point B is $\Delta t_s$ (time measure at the same 
    point relative to the observer 2)\\

    As a result,
    \begin{equation*}
        L_s = v\Delta t
    \end{equation*}
    and 
    \begin{equation*}
        L = v \Delta t_s
    \end{equation*}
    Let's start from the time dilation formula:
    \begin{gather*}
        \Delta t = \frac{\Delta t_s}{\sqrt{1 - \frac{v^2}{c^2}}}\\
        v\Delta t = \frac{v\Delta t_s}{\sqrt{1 - \frac{v^2}{c^2}}}\\
         L_s = L\times \frac{1}{\sqrt{1 - \frac{v^2}{c^2}}}\\
         L = L_s \sqrt{1 - \frac{v^2}{c^2}}
    \end{gather*}
\end{example}