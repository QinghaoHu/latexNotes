
\documentclass[11pt]{report}

% --- Packages ---
\usepackage[utf8]{inputenc}
\usepackage{amsmath, amssymb, amsthm}
\usepackage{geometry}
\usepackage{fancyhdr}
\usepackage{graphicx}
\usepackage{tikz}
\usepackage{enumitem}
\usepackage{hyperref}
\usepackage{xcolor}

% --- Page Setup ---
\geometry{margin=1in}
\pagestyle{fancy}
\fancyhf{}
\rhead{Qinghao Hu}
\lhead{SPH4U}
\cfoot{\thepage}

% --- Theorem Environments ---
\newtheorem{theorem}{Theorem}[section]
\newtheorem{definition}[theorem]{Definition}
\newtheorem{lemma}[theorem]{Lemma}
\newtheorem{proposition}[theorem]{Proposition}
\newtheorem{corollary}[theorem]{Corollary}
\theoremstyle{remark}
\newtheorem*{remark}{Remark}
\newtheorem*{example}{Example}

% --- Custom Commands ---
\newcommand{\R}{\mathbb{R}}
\newcommand{\N}{\mathbb{N}}
\newcommand{\Z}{\mathbb{Z}}
\newcommand{\Q}{\mathbb{Q}}
\newcommand{\C}{\mathbb{C}}
\newcommand{\ds}{\displaystyle}

\newcommand{\mypic}[3]{
    \begin{figure}[h!]
        \centering
        \includegraphics[width=#3\textwidth]{#1}
        \caption{#2}
    \end{figure}
}

% --- Title ---
\title{\textbf{Grade 12 Physics} \\ \large SPH4U}
\author{Qinghao Hu}
\date{\today}

\begin{document}

\maketitle
\newpage
\tableofcontents
\newpage

\chapter{Unit 1A}
\newpage
\section{Review of Describing and Graphing Motion}
\subsection{Position: $\vec{d}$}
Position is the \textbf{straight-line distance} from a fixed reference point to a location, with a direction to the location from the reference point.

\subsection{displacement: $\Delta\vec{d}$}
Displacement is the \textbf{change of position} \\
Formula: \\
\[
	\Delta\vec{d} = \vec{d_{2}} - \vec{d_{1}}
\] 
or \\
\begin{center}
    $n$ = the amount of displacement you want to add
\end{center}
\[
	\Delta\vec{d_{tot}} =  \sum_{i = 1}^{n} \Delta\vec{d_{i}}
\]

\subsection{Velocity: $\vec{v}$}
Velocity is the \textbf{rate of change of position}
\[
	\vec{v} = \frac{\Delta\vec{d}}{\Delta t}
\]

\subsection{Acceleration: $\vec{a}$}
Acceleration is the \textit{rate of change} of velocity, always in the form of $\frac{m}{s^2}$
\[
\vec{a} = \frac{\Delta \vec{v}}{\Delta t}
\]

\subsection{Graphing motion}
For a \textbf{Position vs Time} graph:
\begin{itemize}
\item \textbf{For a position/displacement vs time graph, the velocity = the \textit{slope}} 
\item A slope of zero = The object is not moving
\item Instantaneous velocity ($\vec{v}_{inst}$) = the slope of \textbf{tangent} line to the graph at that point in time
\item Average velocity ($\vec{v}_{avg}$) = the slope of \textbf{secant} line for that time interval
\end{itemize}
For a \textbf{Velocity vs Time} graph:
\begin{itemize}
    \item Can get an object's instantaneous velocity directly from the graph
    \item slope = \textbf{acceleration}
    \item Displacement = The \textbf{area} between the graph and the time-axis for that time interval
\end{itemize}

\newpage
\section{Equations of Motion}
To start off, there are five equations that are used in the calculation of motion
\[
    \vec{v_{f}} = \vec{v_{i}} + \vec{a}*\Delta d
\]
\[
    \Delta\vec{d} = \frac{1}{2}(\vec{v_{i}} + \vec{v_{f}}) * \Delta t
\]
\[
    \Delta\vec{d} = \vec{v_{i}}\Delta t + \frac{1}{2}\vec{a} * \Delta t^2
\]
\[
    \Delta\vec{d} = \vec{v_{f}}\Delta t - \frac{1}{2}\vec{a} * \Delta t^2
\]
\[
    \vec{v_{f}}^2 = \vec{v_{i}}^2 + 2 \vec{a} \Delta \vec{d} 
\]

\subsection{Format requirements for answering Motion questionss}
\begin{enumerate}
    \item You should always include a diagram that contains every known information from the question
    \item $\vec{v}$ should be presented for at least two decimal places
    \item Always list steps in your answer
    \item When using quadratic solving function on calculator, always write \textit{*Using Quadratic Eq} in your answer
    \item When you form an equation system, you should label $1. 2. 3.$ on each equation in the system
    \item Follow the Sig Digit rules
    \begin{itemize}
        \item[!] For \textbf{add} and \textbf{subtract}, keep the least \textbf{decimal places}
        \item[!] For \textbf{multiply} and \textbf{divide}, keep the least amount of \textbf{signficiant digit} 
    \end{itemize}
\end{enumerate}

\newpage
\section{Adding and Subtracting 2-Dimensional Vectors}
\subsection{Vector addition and subtraction key words}
\begin{itemize}
    \item[+] Addition: Find "the \textbf{resultant}", "the total", or "the net". 
    \item[-] Subtraction: Find "the \textbf{difference}" or "the change in".
\end{itemize}

\subsection{Steps for solving a vector problem}
\begin{enumerate}
    \item Read the question carefully
    \item Show unit conventions
    \item Write "givens" (It helps to roughly sketch each vector and their compoents)
    \item Set direction conventions
    \item Solve for each compoents (ex. $\Delta\vec{d_{1y}}, \Delta\vec{d_{2x}}$)
    \item Choose one component direction (ex. Just the 'x' direction) and solve the equations for that direction
    \item Repeat with the other direction
    \item Sketch your resulting $x$ and $y$ vectors, joining them head-to-tail. 
    \item Calculate the magnitude and direction of the resultant. (Trigonometry)
    \item State the final answer, including the real-world direction
\end{enumerate}

\mypic{graph/graph1.png}{A sample answer for vector question}{0.8}

\subsection{Another question type}
\mypic{graph/teacherNote2.png}{Remainder, multiply first}{0.9}

\end{document}

% Referenc
% \section{Introduction}
% Write your introduction or overview of the day's topic here.
%
% \section{Main Concepts}
% \begin{definition}
% A function \( f: A \to B \) is said to be \textbf{injective} if for every \( a_1, a_2 \in A \), \( f(a_1) = f(a_2) \Rightarrow a_1 = a_2 \).
% \end{definition}
%
% \begin{theorem}
% Let \( f: \R \to \R \) be a continuous and strictly increasing function. Then \( f \) is injective.
% \end{theorem}
%
% \begin{proof}
% Assume \( f(a) = f(b) \). Since \( f \) is strictly increasing, if \( a < b \), then \( f(a) < f(b) \), contradiction. So \( a = b \).
% \end{proof}
%
% \section{Examples}
% \begin{example}
% Let \( f(x) = x^2 \). Then \( f \) is not injective on \( \R \), but it is injective on \( [0, \infty) \).
% \end{example}
%
% \section{Diagrams}
% \begin{center}
% \begin{tikzpicture}[scale=1]
% \draw[->] (-2,0) -- (2,0) node[right] {$x$};
% \draw[->] (0,-1) -- (0,4) node[above] {$y$};
% \draw[domain=-1.5:1.5, smooth, variable=\x, blue, thick] 
%     plot ({\x}, {\x*\x});
% \node at (1.2,3.2) {$y = x^2$};
% \end{tikzpicture}
% \end{center}
%
% \section{Summary}
% Summarize what you learned today.
