\newpage
\section{Gravational Potential Energy}
\subsection{Some boring definitions}
\begin{definition}[Gravitation Potential Energy]
    The energy storedf in a system of objects due to the force of gravity acting between those objects. In other words, the energy is stored collectively \textbf{among all } the objects in the system
\end{definition}

Whne the force of gravity acting on the two objects causes this stored GPE to be converted into kinetic energy, the kinetic energy is not \textbf{shared evenlly} between these two objects. 
In the class Example, the earth effectively gets \textbf{zero} and the care effectively gets \textbf{all of them}. Due to this reason, when we have two objects with a very large difference in mass, we can 
always assume that the GPE is \textbf{stored only in the smaller object}

\subsection{Formulas for GPE}
\subsubsection*{Formula 1}
\begin{equation}
    \Delta E_g = mg\Delta h
\end{equation}

\begin{center}
    $\Delta E_g$ is the change in Potential gravational energy(in J)\\
    $m$ is the mass of the object (in kg)\\
    $\Delta h$ is the change in height (in m)\\
    $g$ is the acceleration due to gravity (in $m/s^2$)
\end{center}

\subsubsection*{Fromula 2}
\begin{equation}
    E_g = mg\Delta h
\end{equation}
\begin{center}
    $E-g$ is related to the GPE of the object
\end{center}

\begin{remark}
    For all questions related to the \textbf{Gravational Potential Energy}, you must set your reference height in the diagram. Or, Mr McCumber will forget to add 0.5 for your test!
\end{remark}

In reality, as $h$ changes, the distance between the centres of the two objects changes, and therefore $g$ will change. However, 3 SD's of the 
constant of acceleration due to gravity can handle changes with up and down $1$ km