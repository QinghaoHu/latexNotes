\newpage
\section{String \& Elastic Potential Energy}
\subsection{The Force of String}
\begin{definition}
    (Spring Force): Can be wrote as $F_{spring}$. It is the force exerted by the spring on a object. 
\end{definition}

According to the Hooke's Law, the \textbf{force exerted by a string} is proportional to the string's displacement. So we can express the relationships between
them by some formulas:\\

Vector Version:
\begin{gather}
    \vec{F_x} = -k\Delta \vec{x}
\end{gather}

Scalar Version:
\begin{gather}
    F_x = k \Delta x
\end{gather}

\begin{center}
    $F_x$ is the force exerted by the string on whatever stretches it.\\
    $k$ is the constant of string\\
    $x$ is the displacement of the spring from its unstratched
\end{center}

An essential feature of Hooke's law is that the direction of the spring force is \textbf{opposite} to the direction of displacement from equilibrium. 

\begin{remark}
    When you use the Scalar version, you must clearly understand the direction of the force in you heart. 
\end{remark}


\subsubsection*{Deeper explanation about Hooke's law}
\begin{figure}[h!]
    \centering
    \includegraphics[width=0.5\textwidth]{graph/Hooke's Law.png}
\end{figure}

\begin{definition}
    (Elastic Region): Elastic objects obey Hooke's Law in this region. If the applied force removed, the object will naturally 
    return back to its original shape
\end{definition}

\begin{definition}
    (Elastic Limit): The maximum amount of deformation an object can withstand, and still return to its original shape. 
\end{definition}

\begin{definition}
    (Plastic region): The object no longer obeys Hooke's Law. The object's shape is now permanetly changed. 
\end{definition}

\begin{definition}
    (Fracture): The maximum amount of shape change the object can take, prior to failing (breaking).
\end{definition}

\subsection{Elastic Potential Energy}
\begin{definition}
    (elastic potential energy): The potential energy due to the stretching or compressing of an elastic material
\end{definition}

\begin{figure} [h!]
    \centering
    \includegraphics[width=0.5\textwidth]{graph/elastic Potential energy.png}
    \caption{The work done by a variable force is equal to the area under the the force-displacement graph}
\end{figure}

\subsubsection*{Formula}
\begin{gather}
    W = \frac{1}{2} * \Delta x * F_{spring}\\
    W = \frac{1}{2} * \Delta x * (k * \Delta x)\\
    W = \frac{1}{2} * k * (\Delta x)^2
\end{gather}

The work done by the spring force is the negative of this amount, and is also the negative of the change in Potential Energy.
That means that the work done stretching or compressing the spring is transformed into elastic potential energy.
\begin{equation}
    E_e = \frac{1}{2} k (\Delta x)^2
\end{equation}

\subsection{Ignore Gravational Potential Energy}
We can ignore the Gravational Potential Energy in the vertial spring question if all of these conditions are met:
\begin{itemize}
    \item The mass remains in contact with the spring
    \item We measure all changes in the length relative to the equilibrium position of the mass-spring system (ie. $x_{eq} = 0$)
\end{itemize}