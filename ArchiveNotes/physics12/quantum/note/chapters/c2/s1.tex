\section{Introduction to Quantum Theory}
\subsection{Why we need quantum theory}
Physicists such as J.J. Thomson discovered that Newton's laws failed to explain the behaviour of electrons and atoms. 
Similarly, although Maxwell correctly described electromagnetic phenomena in the every day world, his equations failed to describe the microscopic world. 
This microscopic world is called the quantum world, where \underline{\textit{quantum}} refers to a very small increment of energy. 
The study of the behaviour of these very small bundles of energy, called \underline{\textit{quantum theory}}.

\subsection{Particles and Waves}
According to Newton's laws and Maxwell's equations, energy can be carried from one point to another in two ways:
by particles and by waves. 
However, this is not accurate in the quantum world. 

\begin{definition}
    [Interference Effect] The net effect of the combination of two or more wave trains moving on intersecting or coincident paths.
\end{definition}

\subsubsection{Differences between particle and wave}

\columnratio{0.5, 0.5}

\begin{paracol}{2}
    \begin{rightcolumn}
        \textit{Particle}
        \begin{itemize}
            \item Do not show interference effects
            \item Deliver energy in discrete quantities
            \item Double slit experiment, (b) is a particle. 
        \end{itemize}
    \end{rightcolumn}
    \begin{leftcolumn}
        \textit{Wave}
        \begin{itemize}
            \item Do show interference effects
            \item Waves do not deliver energy in discrete quantities. Wave deliver their energy continuously over time and spread out over the screen.
            (a) is a wave.  
        \end{itemize}
    \end{leftcolumn}
\end{paracol}

\begin{figure}
    [!h]
    \centering
    \includegraphics[width=0.7\textwidth]{pictures/2.1.1.png}
\end{figure}

The energy carried by a wave is described by its intensity, which equals the amount of energy the wave transports per unit time across a 
surface of unit area. 
For the light wave in figure 1(a), the amount of energy absorbed by the screen depends on the intensity of the wave and the absorption time. 
The amount of absorbed energy can take on any non-negative value. 


\subsubsection{An interference Experiment with Electrons}
The separation of particles and waves is not found in the quantum world. 
There are only two types of behaviour are possible: waves exhibit interference; particles do not. \\

However, in reality, electron can exhibit interference, a property that classical theory says is possible only for waves. 

\begin{definition}
    [wave-particle duality] The property of matter that defines is dual nature of displaying both wave-like and particle-like characteristics. 
\end{definition}

The following properties is followed in the quantum world:
\begin{itemize}
    \item All quantum objects, including electromagnetic radiation and electrons, can exhibit interference. 
    \item All quantum objects, including electromagnetic radiation and electrons, transfer energy in distinct, or discrete, amounts. 
    There discrete "parcels" of energy are quanta. 
\end{itemize}