\section{Photons and the Quantum Theory of Light}
\subsection{The Work Function}
\begin{definition}
    [Work Function] The minimum energy required to remove a single electron from a piece of metal. 
\end{definition}
The work function can be calculated through this experiment:
\begin{figure}
    [!h]
    \centering
    \includegraphics[width=7cm]{pictures/2.1.2.png}
\end{figure}

We can manipulate the $\Delta v$ in a circuit.\\

When the electron is pulled out of the metal piece, we can record the $\Delta v$, and calculate the work function through this formula:
\[
    W = e\Delta v
\]

\subsection{The Photoelectric Effect}
Another way to extract electrons from a metal is by shining light onto it. 
Light striking a metal surface is absorbed by the electrons. 
\begin{definition}
    [The Photoelectric Effect] If an electron absorbs an amount of light energy above the metal's work function, it ejects from the metal in a phenomenon. 
\end{definition}

Experimental studies of the photoelectric effect carried out around 1900 revealed that no electrons are emitted unless the light's 
frequency is greater than the \underline{threshold frequency}\\

\begin{definition}
    [Threshold frequency ($f_0$)] the minimum frequency at which electrons are ejected from a metal. 
\end{definition}


Through the experiment, scientists discovered that thershold frequency is independent of the intensity of the light, 
which is conflict to the classical physics (energy carried by a light wave is proportional to the intensity of light).
Experiments found that when the frequency is below the threshold frequency, however, no electrons are ejected, no matter how great the light intensity.  

\subsubsection{Einstein's Quantum Theory of Light}


\begin{definition}
    [Photon] A thought particle which two important properties
    \begin{itemize}
        \item[-] Do not have any mass
        \item[-] Exhibit interference effects
    \end{itemize}

    According to Einstein, each photon carries a parcel of maximum kinetic energy according to the following equation:
    \begin{equation}
        E_{photon} = hf
    \end{equation}
        
    \begin{center}
        Where $f$ is the frequency of the wave and $h$ is the planck constant. 
    \end{center}
\end{definition}

$h = 6.63\times10^{-34}J \times s$

High intensity of light only means high amount of protons collide with metal\\

The absorption of light by an electron is just like a collision between two particles, a photon and an electron. 
Each photon is only responsible for removing one electron from the piece of matel. \\

If 
\begin{gather*}
    \text{Energy of the Proton} < W
\end{gather*}
even with high intensity, no electron will be eject\\

In any other cases, electron will be ejected. 
As a result, we get the formula:
\[
    W = hf_0
\]

and the $E_k$ of the electron can be calculated as this:
\begin{gather*}
    E_k = E_{photon} -  W\\
    E_k = hf - hf_0\\
    E_k = h(f - f_0)
\end{gather*}

This is a straight line.
Hence, the kinetic energy of an ejected electrons should be linearly proportional to $f$. 

\subsection{Photons Possess Energy and Momentum}
Einstein's quantum theory states that light energy can only be absorbed  or emitted in discrete parcels, that is, as single photons. 
The classical theory of electromagnetic waves predicts that a light wave with Energy $E$ also carries a certain amount of momentum:
\[
    P = \frac{E}{c}
\]

Einstein's quantum theory predicts that the momentum of a single photon is:
\begin{equation}
    P_{Photon} = \frac{hf}{c}
\end{equation}

The wavelength of a light wave is related to is frequency as:
\begin{equation*}
    f\lambda = c
\end{equation*}

We can sub this equation for c in the photon momentum equation:
\begin{gather}
    P_{Photon} = \frac{hf}{c} \nonumber \\
    P_{Photon} = \frac{hf}{f\lambda} \nonumber \\
    P_{Photon} = \frac{h}{\lambda}
\end{gather}

\subsubsection{Evidence of Photon Momentum}
Instead of using visible light, Compton directed a beam of high-energy X ray photons at a thin metal foil. 
The foil ejected both electrons and lower-energy X-ray photons. 
This effect, in which incident X-ray photons lose energy and scatter of a metal foil along with free electrons, is called the 
\underline{\textbf{Compton effect}}\\

X-ray photon acts like a particle in an elastic collision with an electron in the metal. \\

The electron absorb low-frequency x-ray and collide with high-frequency x-ray. \\

After the collision, the photon emerges from the collision with lower energy and a different momentum. 
The electron deflects with the kinetic energy and momentum lost by the photon. 

\begin{definition}
    [Compton effect] The scattering of a photon by a free or weakly bound electron, in which total energy and momentum are conserved, leading to a change in the photon’s wavelength.
\end{definition}

Compton's data indicated that the effect conserves both energy and momentum. 
Compton had to use equations of special relativity to analyze the collsion, including the equation for relativistic momentum.
He would not have obtained the correct results without using special relativity. 
In this way, Einstein's ideas on relativity and the speed of light influenced work that confirmed Einstein's ideas 
about the behaviour and characteristics of photons. 

\subsubsection{Photon Interactoin}
When a photon comes into ocntact with matter, an interaction takes place. Five main interactions can occur:
\begin{enumerate}
    \item A photon may simply reflect, as when photons of visible light undergo perfectly elastic collisions with a mirror. 
    \item A photon may free an electron and be absorbed in the process, as in the photoelectric effect
    \item A photon may emerge with less energy and momentum after freeing an electron. 
    After this interaction with matter, the photon still travels at the speed of light but with less energy and a low frequency. 
    This is the Compton effect.
    \item A photon may be absorbed by an individual atom and elevate an electron to a higher energy level within atom. 
    The electron remains within the atom but is in what is called an excited state. 
    \item A photon can undergo pair creaton, where it becomes converted into two particles with mass. 
    This process conserves energy and momentum because all the energy of the photon becomes converted into the kinetic energy of the new particles and their rest mass energy. 
\end{enumerate}

\begin{definition}
    [Pair Creation] the transformation of a photon into two particles with mass. 
\end{definition}

\subsection{Blackbody Radiation}
\begin{definition}
    [Blackbody] An object that absorbs all radiation reaching it. 
\end{definition}

\begin{definition}
    [Blackbody Radiation] Radiation emitted by a blackbody. 
\end{definition}

The specific problem that puzzled Planck is represented by the glowing oven. 
This oven emits radiation over a range of wave-lengths and frequencies. 
To the eye, the colour of the oven is determined by the wavelength of the largest radiation intensity. 

\begin{figure}
    [!h]
    \centering
    \includegraphics[width=10cm]{pictures/2.2.2.png}
\end{figure}

Experiments prior to Planck's work showed that the intensity curve has the same shape for a wide variety of objects. 
The blackbody intensity falls to zero at both long and short wavelengths, corresponding to low and high frequencies, 
respectively, with a peak in the middle. Planck tried to explain this behaviour. \\

At this time, physicists knew that electromagnetic waves from standing waves as they reflect back and forth inside an oven. 
These standing waves are just like the standing waves on a string. Standing waves on a string have frequencies that follow the pattern:
\[
    f_n = nf_0
\],
where $f_0$ is the fundamental frequency and $n \in \mathbb{Z}$

\begin{figure}
    [!h]
    \centering
    \includegraphics[width = 6cm]{pictures/2.2.3.png}
\end{figure}

According to the classical physics, each of these standing waves carries energy, and as their frequency increases, 
so does the total energy. 
As a result, the classical theory predicts that the blackbody intensity should become infinite as the frequency approaches infinite values. 

\subsubsection{Planck's Hypothesis}
The classical theory would be in conflict with the experimental intensity curves in the graph. 
The true intensity falls to zero at high frequencies.

Planck resolved this disagreement by hypothesizing that the energy in a blackbody comes in discrete parcels (quanta). 
He believed that each parcel has energy equal to $hf_n$, where $f_n$ is one of the standing wave frequencies and $h$ is a universal constant. 
This explanation can correctly predit the behaviour of wave in the experiment. 
However, he could give no reason or justification for his assumption about standing-wave quanta. 

\subsection{Wien's Law}
The wavelength at which the radiation intensity of a blackbody is largest is donoted by $\lambda_{max}$ and is determined by the temperature, $T$, 
of the blackbody through ane expression called Wien's law. 
\begin{equation}
    \lambda_{max} = \frac{2.90\times 10^{-3}m*K}{T}
\end{equation}
