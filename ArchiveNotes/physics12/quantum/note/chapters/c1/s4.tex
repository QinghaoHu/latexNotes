\section{Mass-Energy Equivalence}
In the previous section, we discussed the relativistic Momentum. The relativistic mass can be describe as this: 
\begin{equation*}
    m_{\text{relativistic}} = \frac{m}{\sqrt{1 - \frac{v^2}{c^2}}}
\end{equation*}
where $m$ is the mass of the object at rest. \\

Using the equations of special relativity, Herr Einstein concluded that the total energy, $E_{\text{total}}$ for an object with rest mass 
$m$ moving with speed $v$ is equal to:
\begin{equation*}
    E_{\text{total}} = \frac{mc^2}{\sqrt{1 - \frac{v^2}{c^2}}}
\end{equation*}

When the object is at rest, 
\begin{equation*}
    E_{\text{rest}} = mc^2
\end{equation*}

\begin{definition}
    [rest energy ($E_{\text{rest}}$)]the amount of energy an object at rest has with respect to an observer
\end{definition}

\begin{definition}
    [relativistic kinetic energy($E_k$)] the energy of an object in excess of its rest energy
\end{definition}

The $E_k$ can be solved by this:
\begin{gather*}
    E_{\text{total}} - E_{\text{rest}} = E_k\\
    E_k = \frac{mc^2}{\sqrt{1 - \frac{v^2}{c^2}}} - mc^2
\end{gather*}

Unlike various potential energy, where a force is acting on an object without moving it, 
rest energy is the property of matter itself. Like space and time form space-time theory, The conservation of energy principle is now 
the principle of \underline{\textit{conservation of mass-energy}}, which states that the rest enery is equal to rest mass times the speed of light squared!