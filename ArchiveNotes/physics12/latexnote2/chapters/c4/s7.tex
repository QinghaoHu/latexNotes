\section{Magnetic Force on Moving Charges}
\subsection{Magnetic Force}
\subsubsection{\textcolor{blue}{Why electron creates a magnetic field}}
A \underline{stationary} charged particle does not create a magnetic field. However, a charged particle that is \underline{moving 
creates a circular magnetic field} around it. The magnetic field is \underline{perpendicular} to the direction the particle is traveling. 

\begin{figure}[!h]
    \centering
    \includegraphics[width=0.7\textwidth]{pictures/4.7.1.png}
\end{figure}

\begin{definition}
    [Magnetic Force] The magnetic field that the moving particle creates can \underline{interact with another magnetic field that cause} \underline{magnetic force} to be exerted (on both the charged 
    particle and the thing that is creating the other magnetic field)
\end{definition}

\begin{theorem}
    The magnitude of the magnetic force exerted on the charged particle depends on four factors:
    \begin{enumerate}
        \item The charge of the particle ($F_M \propto q$)
        \item The \underline{speed} at which the particle is moving (relative to the magnetic field it is moving through)($F_M \propto v$)
        \item The strength of the magnitude field that the charged particle is moving through
        \item The angle in between the two magnetic fields ($0^{\circ}$ leads to the maximum magnetic force, $90^{\circ}$ leads to the minimum)
    \end{enumerate}

    It is time consuming to determine the field direction around a charged particle that is moving. As a result, people tends to find the angle between the \underline{direction of current traveling} and the magnetic field
\end{theorem}

\begin{theorem}
    The strength of force of magnetic is defined by this formula:
    \begin{equation*}
        F_M = \left|q\right| vB\sin\theta
    \end{equation*}
    \begin{center}
        $F_M$ is the magnitude of the magnetic force on the charged particle (in Newton)\\
        $q$ is the charge of the particle(in C)\\
        $v$ is the speed of the particle (related to the magnetic field is it going through (in $\tfrac{m}{s}$))\\
        $\theta$ is the angle in between the direction the particle is travelling and the magnetic field it is traveling through.
        $B$ is the strength of the magnetic field that the particle is going through (in $T$ or $Tesla$)
    \end{center}
    \begin{remark}
        Do not include the signs for any variable in this formula, you need to determine the direction of $F_M$  conceptually
    \end{remark}
\end{theorem}

\subsection{Right Hand Rule}
\begin{itemize}
    \item For use with positvely-charged particles
    \item Point the thumb of your right hand in the direction the \underline{particle} is moving
    \item Point the fingers of your right hand in the direction of the \underline{magnetic-field} the particle is travelling
    \item The palm of your right hand will face the direction of the force on the positvely charged particle. 
\end{itemize}