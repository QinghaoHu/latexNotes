\section{The Magnetic Force on a Straight Conductor}
\subsection{Definitions}
\begin{theorem}
    The magnetic force on a straight conductor can be described by this formula:
    \begin{equation*}
        F_M = BIL\sin\theta
    \end{equation*}
    \begin{center}
        $F_M$ is the magnitude of the magnetic force on the straight conductor (in $N$)\\
        $B$ is the strength of the magnetic field that the conductor is in(in $T$)\\
        $I$ is the current flowing through the conductor in $A$\\
        $L$ is the length of the conductor that is the magnetic field (in $m$)\\
        $\theta$ is the angle in between the current and the magnetic field
    \end{center}
\end{theorem}
\subsubsection{Derive a formula for the magnetic force on the conductor}

\begin{lemma}
    The magnitude of the magnetic force of one charged particle in the conductor can be evaluated by this formula:
    \begin{equation}
        F_{M(\text{single charge})} = \left|q\right|vB\sin\theta \label{eq4.9.3}
    \end{equation}
\end{lemma}

\begin{lemma}
    If one of the charged particles travels from one end of the conductor to the other end, we can rewrite $v$ into 
    \begin{equation}
        v = \frac{L}{\Delta t} \label{eq4.9.2}
    \end{equation}
\end{lemma}

\begin{lemma}
    Current is defined as the amount of \underline{charge} that passes through a location, per unit time. In the time of $\Delta t$, $n$ charges 
    pass through the end of the conductor
    \begin{equation}
        I = \frac{nq}{\Delta t} \label{eq4.9.4}
    \end{equation} 
\end{lemma}

\noindent\hrulefill
\begin{proof}
Assume we have a straight conductor with these parameters in magnetic field:
\begin{center}
    A length of $L$\\
    A magnetic Field strength of $B$\\
    A current of $I$\\
    Angle between the current and the magnetic field: $\theta$
\end{center}
\begin{figure}[!h]
    \centering
    \includegraphics[width=0.9\textwidth]{pictures/4.9.1.png}
\end{figure}


The $F_M$ on all of the charged particles in the conductor will be the \underline{sum} of all the $F_M$ on the individual charged particles. 
Each particle will experience an \underline{idential} $F_M$ because each particle has the same \underline{velocity and charge}. Also the magnetic 
field that the conductor is in is \underline{uniform}.\\

So we can multiple \ref{eq4.9.3} by $n$:
\begin{equation}
    F_{M(\text{on all charge})} = n \times \left |q\right |vB\sin\theta \label{eq4.9.1}
\end{equation}



Sub \ref{eq4.9.2} into \ref{eq4.9.1}:
\begin{equation}
    F_{M(\text{all charge})} = n \times \frac{\left|q\right|}{\Delta t}LB\sin\theta \label{eq4.9.5}
\end{equation}

Sub \ref{eq4.9.4} into \ref{eq4.9.5}
    \begin{equation*}
        F_{M(\text{all charge})} = ILB\sin\theta
    \end{equation*}
\end{proof}

\subsubsection{The direction of the force}
If you want to use the \underline{Right hand Rule \# 3}, you should point your thumb to the direction of the \underline{current flow}\\

In the contrary, if you want to use \underline{Left hand Rule \# 3}, you should point your thumb to the direction of the \underline{electron flow}

\subsection{What is Tesla}
\underline{1 Tesla} is the strength of the magnetic field that will cause a force of \underline{1 Newton} to be exerted one a \underline{1m} long straight conductor 
that has a current of \underline{1 A}
\begin{proof}
    \begin{gather*}
         F_M = BIL\sin\theta (\theta = 90^{\circ})\\
         B = \frac{F_M}{IL}\\
         1T = \frac{1N}{1A \times 1m}
    \end{gather*}
   
\end{proof}

If we have 2 Tesla of magnetic Field Strength:
\begin{gather*}
    2T = \frac{2N}{1A \times 1m}
\end{gather*}