\section{Electric Fields}
\begin{definition}[Field]
    The region where an appropriate object would feel a force!
\end{definition}
\begin{itemize}
    \item If there's a gravitational field, a \underline{mass} will feel a force. 
    \item If there's an electric field, a \underline{charge} will feel a force.
    \item If there is an magnetic field, a \underline{magnet} (or a moving charge) will feel a force.
\end{itemize}

Visualizing Electric Fields - Field lines show how a small \underline{\textit{positive}} charge would move.

\begin{figure}[!h]
    \centering
    \includegraphics[width=0.5\textwidth]{pictures/4.3.1.png}
    \caption{Electric field of Positive and negative charge}
\end{figure}

\begin{figure}[!h]
    \centering
    \includegraphics[width=0.5\textwidth]{pictures/4.3.3.png}
    \caption{Electric field between two charges}
\end{figure}

\subsubsection{Parallel Plates}
Two charged metal plates that are parallel to each other $\rightarrow$ "parallel plates"
\begin{itemize}
    \item The field strength outside of the plates is very weak and can be considered negligible
    \item The field lines in between the plates are equdistant stand. 
\end{itemize} 

\subsubsection{Formulas}
\begin{equation}
    \mathcal{E} = \frac{k \left | q \right |}{R^2}
\end{equation}
\begin{center}
    $\mathcal{E}$ is the magnitude of the \underline{electric field strength} around a point charge (in $\tfrac{N}{C}$)\\
    $k = 8.99 \times 10^9 \tfrac{Nm^2}{C^2}$\\
    $R$ is the distance away from the point charge (q) where you want to know the field strength (in $m$)
\end{center}

\noindent\hrulefill
\begin{remark}
    Electric fields are \underline{vector}. The direction of the field will be based on the direction of force that would be 
    exerted on a \underline{positively-charged} object!
\end{remark}
\noindent\hrulefill

\begin{equation}
    \vec{F_E} = q \times \vec{\mathcal{E}}
\end{equation}

\begin{center}
    $\vec{F_E}$ is the magnitude of the electrical force exerted on $q$ (in $N$)\\
    $q$ is the that is \underline{in the electric field}(in $C$)\\
    $\vec{\mathcal{E}}$ is the strength of the electrical field that the charge is in (in $\frac{N}{C}$)
\end{center}