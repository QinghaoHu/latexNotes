\section{Lecture 2.5}
\subsection{Uniform Circular Motion}
% \begin{redblock}
    \textbf{Direction}: The velocity of an object at any point along a circle has a direction that is 
    \red{tangential} to the circle
% \end{redblock}

\columnratio{0.6, 0.4}
\begin{paracol}{2}
    \begin{leftcolumn}
        \textbf{Question:} If an object is attached to a string, swung in a circular motion and then the string is released,
        which of the five paths shown here will the object take?\\
        \red{ANS: Path 2}
    \end{leftcolumn}

    \begin{rightcolumn}
        \begin{center}
            \includegraphics[width=0.3\textwidth]{pictures/circularMotion.png}
        \end{center}
    \end{rightcolumn}
\end{paracol}

\subsection{Centripetal acceleration}
From the \red{Newton second law}, we understand that \red{An object will accelerate in the same 
direction as the net force}.

If the centripetal force is directed toward the centre of the circle, then what direction is the acceleration in?
\red{ANS: Toward the circle}

In other words, the acceleration will always \red{perpendicular} to the velocity of the object.

\subsection{Formulas}
% \begin{cyanblock}
    Formula 1:
    \[
        \vec{a_{c}} = \frac{4 \pi ^2 R}{T^2}
    \]
    \[
        \vec{a_{c}} = 4\pi ^2 R f^2
    \]
    \[
        \vec{a_{c}} = \frac{V^2}{R}
    \]
    \begin{center}
        $\vec{a_{c}}$ is the acceleration of the object in $\frac{m}{s^2}$\\
        $R$ is the radius of the circular path that the object is moving around (in $m$)\\
        $T$ is the period of the object's motion \\
        $v$ is the speed of the object in (m/s)
    \end{center}
% \end{cyanblock}

% \begin{cyanblock}
    For clockwise:
    \begin{center}
        direction of acceleration = direction of velocity + 90 degree
    \end{center}
    else:
    \begin{center}
         direction of acceleration = direction of velocity - 90 degree
    \end{center}
% \end{cyanblock}