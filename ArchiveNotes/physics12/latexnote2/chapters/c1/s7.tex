\section{Review of Netwon's Laws of Montion}
\subsection{Netwon's First Laws}
Inertia is an object's \textcolor{blue}{resistance} to a change in its state of uniform Motion.
\begin{example}
    An object at rest will \textcolor{red}{remain at rest} And an object in motion will continue
    to \textcolor{red}{move in a straight line constant speed} UNLESS a non-zero net force acts on 
    the object
\end{example}

\subsection{Newton's second Law}
Newton's $2^{nd}$ Law is the formula that explains the behaviour of object when the forces on the object 
are not zero. \\
We can orgnize to these formulas:
\begin{equation}
    \vec{a} = \frac{\sum \vec{F}}{m}
\end{equation}
and\\
\begin{equation}
    \sum \vec{F} = m * \vec{a}
\end{equation}
\begin{center}
    $\vec{a}$ = acceleration of the object \\ 
    $m$ = mass of the object in kg \\ 
    $\sum \vec{F}$ = The sum of the net force
\end{center}

\subsection{Newton's Third Law}
For every force, there is another force, which is \textit{equal in magnitude} to the first force, 
but \textit{opposite in direction}. These two forces will \textit{act on separate objects},
unless they are "internal force"\\ 
This mean's that all forces Always \textcolor{blue}{come in pairs}, but two forces may not be 
acting on the same object. \\

To fit the Newton's $3^{rd}$ Law pair forces, the two force must:
\begin{enumerate}
    \item Be the same type of force
    \item $\vec{F_{A/B}} = -\vec{F_{B/A}}$
\end{enumerate}
\newpage
\subsection{Free Body Diagrams (FBD)}
\mypic{pictures/graph6.png}{An example Free Body Diagram}{0.7}

\subsection{Application of Newton's second Law}
Here is an example:
\mypic{pictures/graph7.png}{}{0.8}