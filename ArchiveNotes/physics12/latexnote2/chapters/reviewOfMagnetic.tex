\section{Intro to Magnetism}
The direction of the magnetic Field for a magnet: from \underline{\textit{South Pole}} to \underline{\textit{North Pole}}

\subsection{Solenoid}
Factors affecting the strength of a solenoid
\begin{itemize}
    \item The current flow through the solenoid
    \item The number of turns in the solenoid
    \item The diameter of the solenoid
    \item The substance that fills the core of the solenoid
\end{itemize}

\subsection{Magnetic Domain Theory}
Any piece of ferromagnetic material contains tiny, rotatable magnets called \underline{\textit{dipole}}\\

Each \underline{\textit{dipole}} has a north and south pole, which are indivisible. \\

A magnetic domain is created when these dipoles rotate so that in a region of the material, a group of the dipoles point in the same direction\\

\subsubsection{Factors affecting the magnetic field created by a permanent magnet}
\begin{itemize}
    \item can lose the magnetic field if it is dropped, jarred or heated (This may cause the dipoles to become randomly aligned).
    \item A permanent's magnet's field can be reversed. This may occur if the permanent magnet is placed in a stronger magnetic field. 
    \item Any permanent magnet can be broken into smaller pieces. 
    \item Any permanent magnet has a maximum possible magnetic field strength. once all of the dipoles in it are aligned
\end{itemize}