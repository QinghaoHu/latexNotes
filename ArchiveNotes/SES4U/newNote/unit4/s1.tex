\section{Cosmos Episode 7}
1. What year for the beginning of the Earth did James Ussher give based on his study of the Bible?
\begin{itemize}
    \item 23, October 4004 BC
\end{itemize}

2. What stable element does Uranium break down into after about 10 transformations?
\begin{itemize}
    \item Lead
\end{itemize}

3. What happened to the rocks that were around at the birth of the Earth?
\begin{itemize}
    \item Crushed, melted, and remade 
\end{itemize}

4. What did Clare Patterson need to build before he could completely rule out lead contaminatoni in his sample?
\begin{itemize}
    \item a new lab
\end{itemize}

5. What was the true age of the Earth found to be?
\begin{itemize}
    \item 4.5 Billion Years old. 
\end{itemize}

6. How did Clare Patterson conclude the oceans were being contaminated by lead gasoline?
\begin{itemize}
    \item Lead is concentrated at surface not great depth = recent deposition. 
    \item Amount of lead in ice and snow is much lower few hundred years ago. Before industrial revolution. 
\end{itemize}

\newpage
\section{The Impossible Hugeness of Deep Time}
\subsection{Age of Earth}
$4.54 \times 10^9$ years old / 4.54 Billion years old

\subsection{Order of the Event}
\begin{enumerate}
    \item Formation of the Earth
    \item Formation of Moon
    \item Liquid water
    \item Formation of Earth's atmosphere
    \item First living organisms 
    \item Buildup of oxygen in atmosphere
    \item First Eukaryotes
    \item Formatoin of supercontinent Rodinia
    \item Breakup of Rodinia
    \item Early multicellular organisms
    \item Cambrian explosion
    \item First land plants
    \item First amphibian, insect, tree, and shark fossils
    \item First reptiles
    \item Earliest flowering plants
    \item Earliest birds and mammals
    \item Early primates
    \item Present day
\end{enumerate}

\newpage
\section{Relative Age Dating}
\subsection{Principles}
There are five principles can be used to identify rock's relative age.\\

\textbf{\textcolor{blue}{Principle of superposition}}: In a sequence of undeformed sedimentary rocks, the oldest beds are on the bottom and the youngest are on top.  \\

\textbf{\textcolor{blue}{Principle of original horizontality}}: Sedimentary layers are horizontal, or nearly so, when originally deposited. 
Strata that are not horizontal have been deformed by movements of the Earth's crust. \\

\textbf{\textcolor{blue}{Principle of faunal succession}}: Groups of fossil plants and animals occur in the geologic record in a definite and determinable order. 
A period of geologic time can be recognized by its respective fossils. \\

\textbf{\textcolor{blue}{Principle of crosscutting relations}}: Geologic features, such as faults, and igenous intrusions are younger than the rocks they cut. \\

\textbf{\textcolor{blue}{Principle of inclusion}}: a rock body that contains inclusions of preexisting rocks is younger that the rocks from which the inclusions came from.

\begin{equation}
    N = N_0 (\frac{1}{2})^{\frac{t}{t_{half}}}
\end{equation}