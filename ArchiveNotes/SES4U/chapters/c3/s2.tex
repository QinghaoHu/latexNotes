\section{Rock cycle and Identification}
\subsection{Introduction to rock}
\subsubsection{Sedimentary Rock}
\textbf{Sedimentary Rock}: Generally form from the compaction and cementation of \textcolor{red}{\underline{sediments}}
\newline 

Now here we have two different categories\\

\textbf{Clastic}: compacted sediments, classified by size. \textcolor{blue}{ex. Sandstone, conglomerate, siltstone}
\\

\textbf{Organic or crystallite}
\begin{itemize}
    \item Evaporites
    \item Precipitates
    \item Biological matter \textcolor{blue}{ex. limestone}
\end{itemize}


\subsubsection{Metamorphic}
\textbf{Metamorphic Rock}: Rocks that are changed as a result of intense heat and/or pressure
\\

\textbf{Contact metamorphism}: Heat \textcolor{blue}{ex. schist, gneiss, marble}
\\

\textbf{Regional metamorphism}: Pressure

\subsubsection{Igneous}
\textbf{Igneous Rock}: Form from the cooling and solidification of lava or magma.
\\

\textbf{Intrusive}: Formed from magma that cools and solidites underground
\begin{itemize}
    \item Magma cools slowly
    \item Large crystal formed
\end{itemize}

\textbf{Extrusive}: Formed on the surface of Earth from lava.
\begin{itemize}
    \item Lava cools quickly
    \item small or no crystal
    \item May be visiculars (Contains air bubbles)
\end{itemize}