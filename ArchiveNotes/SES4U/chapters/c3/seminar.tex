\section{Unit 3 - seminar Testable Question}

\subsection{Formation and Identification of minerals}
\subsubsection{Question 1}
What is the definition of a mineral? \\ 

\underline{A solid inorganic substance that occurs naturally}

\subsubsection{Question 2}
Explain streak, hardness and cleavage/fracture
\begin{itemize}
    \item Streak: Used to determine the color of mineral in its powdered form, test by rubbing the mineral against a streak plate
    \item Hardness: use Moh's scale of hardness, compare hardness of an unknown minerals to the hardness 
    of 10 known minerals and/or to the household objects
    \item Cleavage/fracture: different minerals break in different ways depending on their atomic structure. 
\end{itemize}

\subsection{Gemstones}
\subsubsection{Question 1}
The disadvantage iof Chemical Vapor Deposition(CVD) diamonds?\\

Takes much more time to produce compared to a High Pressure Temperature (HPHT) diamonds

\subsubsection{Questoin 2}
Application of diamond:
\begin{itemize}
    \item Nanodiamonds: The super small diamonds that aloow for many medical implication
    \item dental implants
\end{itemize}

\subsubsection{Question 3}
Blood diamonds:
\begin{itemize}
    \item Mined in areas controlled by forces opposed to a country's government
    \item Sold to fund military action against that government
    \item Often extracted under conditions of forced labor
\end{itemize}

\subsection{Marine geology - Deep sea sedimentation}

What is a sedimentation?
\begin{itemize}
    \item The settling of solid particles in fluid
\end{itemize}

\noindent\hrulefill

What can sediments tell us about Earth's past climate?
\begin{itemize}
    \item Past temperatures, climate and rainfall.
\end{itemize}

\noindent\hrulefill

\subsection{Investigate technologies to predict earthquakes, volcanoes and tsunamis }
Which of the following gas are most often used when predicting volcanic eruptions?
\begin{itemize}
    \item Sulfur Dioxide
\end{itemize}

\noindent\hrulefill


Describe how the use of interferometry by Synthetic Aperture Radar (SAR) satelliets might help to detect earthquakes?
\begin{enumerate}
    \item Interferometry takes two or more images of the same area of the Earth at different times
    \item Analyzed for changes between images
    \item Changes could be signs of an earthquakes
\end{enumerate}

\subsection{Investigate the Canadian Lithoprobe Project and how it enhanced our understanding of the Earth’s interior.}
What was the main goal of the Canadian Lithprobe project:
\begin{itemize}
    \item To map the deep structure of Earth's crust using seismic technology
\end{itemize}

Which technology was most important in helping Lithprobe "see" inside of the Earth?
\begin{itemize}
    \item Seismic reflection and refrection surveys
\end{itemize}

The Canadian Lithoprobe Project helped scientists to better understand \underline{Ancient Continental Collisions}

\subsection{Investigate the structural engineering techniques and advance technology used to build earthquake-resistant infrastructure}
\subsubsection{1}
What materials are base isolators typicall made out of?
\begin{itemize}
    \item Lead, rubber, steel
\end{itemize}

\subsubsection{2}
How to turned mass dampers protect a building during an earthquake?
\begin{itemize}
    \item It opposes the motion of the structure by swinging in the opposite direction
\end{itemize}

\subsubsection{3}
Why is steel typically added to concrete when structures are constructed?
\begin{itemize}
    \item Concrete in strong under compression but weak under tension
    \item Adding steel allows it to bend without breaking during earthquakes
\end{itemize}

\subsection{How have Earthquakes and volcanoes affected the development of cities?}
According to the case study, what was the primary phenomenon that caused extensive damage in Christchurch, New Zealand, other than the seismic waves?
\begin{itemize}
    \item Soil liquefaction
\end{itemize}

\subsubsection{2}
Explain the fundamental difference between "brittle failure" and "ductile design" in building materials when faced with seismic stress.
\begin{itemize}
    \item Brittle failure occurs when rigid materials shatter or crack suddenly under stress
    \item Ductile design allows materials to flex and bend without breaking
\end{itemize}

\subsubsection{3}
For the “Earthquake Early Warning” (EEW) system, explain how it works, and a key limitation: 
\begin{itemize}
    \item Its primary function is to detect the fast P-waves to provide a warning to the public before the damaging S-waves arrive
    \item It provides a relatively short warning
\end{itemize}

\subsection{Minerals and human health}
What is hard water?
\begin{itemize}
    \item Water containing high concentrations of dissolved minerals
\end{itemize}

Where does it come from:
\begin{itemize}
    \item When rainwater, which is slightly acidic, filters through the ground, it dissolves minerals, which then goes into the groundwater
\end{itemize}

\subsection{Critical minerals}
List the four most important sectors that require critical minerals?
\begin{itemize}
    \item Electronics, renewable energy, medical devices, military defense
\end{itemize}

Define "critical Mineral"
\begin{itemize}
    \item Elements and compounds that are essential for modern technologies
    \item Face a high risk of supply chain destruption due to factors like rarity, geographical concentration, or geopolitical issues
\end{itemize}

Production of critical minerals
\begin{itemize}
    \item Upstream: extraction
    \item Midstream: Processing and refining
    \item Downstream: recycling
\end{itemize}

\subsection{The geological composition of famous buildings and landmarks}
What makes ancient Roman concrete special?
\begin{itemize}
    \item Primary binding agent in the concrete was a mixture of lime and volcanci ash, which helped to give it self-healing abilities
\end{itemize}

\noindent\hrulefill

Explain how and why limestone and granite was used in the Pyramids:
\begin{itemize}
    \item Limestone
    \item \begin{itemize}
        \item Makes up most of the outer layer to make the pyramid smooth and shiny
        \item Relatively easy to quarry and shape
    \end{itemize}
    \item granite
    \item \begin{itemize}
        \item Makes up the internal structural elements
        \item Used for its strength and status
        \item More difficult to work with
    \end{itemize}
\end{itemize}

\subsection{Mining technologies}
Two important methods of surface mining extraction
\begin{itemize}
    \item Open pit mining: a mine is dig out on the ground's surface
    \item Quarrying: A type of open pit mine that primarily extract rocks
\end{itemize}

Sub-surface mining extraction method:
\begin{itemize}
    \item Block Caving
    \item Ore deposit is undercut at different weak points
    \item The ore body then collapses under its own gravity into drawpoints where minerals are collected and transported
\end{itemize}

\subsection{Oil and natural gas}
Core mechanism of a seismic survey?
\begin{itemize}
    \item Emitting vibrations into the ground and recording the reflected waves from rock layers
\end{itemize}

Key process that controls whether hydrocarbons become oil or natural gas?
\begin{itemize}
    \item Kerogen breaks down into oil at lower temperatures and shallow depths
    \item Natural gas is produced at much higher temperature and deeper depth
\end{itemize}

\subsubsection{temperature-controlled distillation}
\begin{enumerate}
    \item Crude oil is heated until it vaporizes
    \item The hot vapor enters a distillation tower, where the bottom is very hot and the top is cooler
    \item Light hydrocarbons rise higher in the twer, heavier hydrocarbons stay lower
\end{enumerate}

\subsection{Asteroid mining}
Karman+'s long term plan after they mine materials from asteroids?
\begin{itemize}
    \item They want to build and sustain things in space, where things like manufacturing and fuel production happens there.
\end{itemize}

What is a carbonaceous asteroid?
\begin{itemize}
    \item The asteroid is rich in hydrated clays
\end{itemize}

How do catalytic converters in vehicles help with air pollution caused by vehicles?
\begin{itemize}
    \item They reduce pollutants such as carbon monoxide and hydrocarbons by converting them into less hamful gases
\end{itemize}

\subsection{Weathering and Erosion}
How water can cause both mechancial and chemical weathering in the same rock at the same time.
\begin{itemize}
    \item Mechancial - Water can physically break rock by freezing and expanding in cracks
    \item Chemical - Water can react with minerals and/or dissolve them
\end{itemize}

If there are two of the same type and size of rock, why does the rock with more surface area experience faster weathering?
\begin{itemize}
    \item Because more of the rock is exposed to weathering agents like water, air and acids
\end{itemize}