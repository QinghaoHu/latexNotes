\section{Uniform Distributions}
\subsection{Different Distributions}
Distributions of data can be classified by considering the general shape of its graph. This picture is a table of distributions which is copied from 
the teacher's note

\begin{figure}[!h]
    \centering
    \includegraphics[width=0.9\textwidth]{pictures/5.2.1.png}
    \caption{Thanks to \textbf{Mr Tang}}
\end{figure}

\subsection{Characteristics of Uniform Distribution}
If an distribution is considered Uniform, it will have following characteristics:
\begin{enumerate}
    \item Each outcome is EQUALLY likely in any single trial of experiment
    \item If $X$ is discrete and $n$ is the number of possible outcomes in the probability experiment, then 
    \begin{gather*}
        P(X = x) = P(x) = \frac{1}{n}\\
        E(x) = \frac{\Sigma x_i}{n} \\
        \sigma = \sqrt{
            \frac{1}{n} \sum (x - E(x))^2
        }
    \end{gather*}
    \item If $X$ is continous with values in range from $a$ to $b$, then the expected value $E(x)$ will be $\frac{a + b}{2}$
\end{enumerate}

\begin{example}
    I want to discuss
\end{example}

\begin{proof}
    
\end{proof}