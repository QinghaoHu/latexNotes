\chapter{Unit 1}
\section{Lecture 1}
\subsection{The Fundamental or Multiplicative Counting Principle}
If a task is made up of \textcolor{blue}{\textit{several stages}}, then the number of choices is the \textbf{product} 
of the number of possibilities at each stage. 

\subsection{Additive Counting Principle}
In a situation with actions that cannot occurred at the "same time" than the number
of possibilities is the sum of the possibilities of all the actions

\begin{center}
    !!! Remember, $0! = 1$
\end{center}

\section{Lecture 1.2}
A permutation is an \textcolor{red}{ARRANGEMENT} of items in a definite order.
\[
    nPr = \frac{n!}{(n - r)!}
\]
and 
\[
    nPr = P(n, r)
\]

\section{Like Term Permutations}
The number of permutations of a set of $n$ objects containing $a$ identical objects of one kind, 
$b$ identical objects of a second kind, $c$ identical objects of a third kindand so on is $\frac{n!}{a! * b! * c!}$

\section{Pascal Triangle}
Do what the hell you want to do about Pascal 

\section{Venn Diagrams}
Concepts:
\begin{itemize}
    \item In mathematics, a set is a well-defined collection of \textit{distinct} objects/elements
    \item A Venn diagram is used to organize the (number of) elements in different \textit{set} of data.
    \item Elements that are in set $a$ and set $b$ are described as the intersection of $A$ and $B$. The notation of $A \cap B$ describes this situation
    \item Elements that are in set $a$ or set $b$ are discribed as the combine of $A$ and $B$. The notation of of $A\cup B$ describes this situation
    \item The Complement, $A^{,}$ of a set $A$ is the set of all elements in the universal set that are \textit{NOT} elements of $A$.
\end{itemize}

\section{Combination}
\[
    nCr = C_{n}^r = \frac{n!}{r! * (n - r)!}
\]
\begin{center}
    Remainder, $0!$ = \textbf{$1$}
\end{center}
