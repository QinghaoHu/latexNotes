\section{Linear Correlation}
\subsection{Why Scatter plot?}
There are few advantages of a scatter plot:
\begin{itemize}
    \item Correlations
    \item Predictions
    \item Positive/negative
    \item Strong/weak
\end{itemize}

\subsection{Some boring Definitions}
\subsubsection{Linear Relationship}
A \textbf{linear} relationship is one in which a \textbf{change} in the independent (explanatory) variable corresponds a proportional change in the dependent
(response) variable.\\

We can use table to calculate the correlation of two variables:
\begin{figure}[h!]
    \centering
    \includegraphics[width=0.7\textwidth]{pictures/4.2.1.png}
\end{figure}

\begin{gather}
    \bar{x} = \frac{\sum x}{n}\\
    \bar{y} = \frac{\sum y}{n}
\end{gather}
In order to calculate the correlation coefficient of x and y, we need to get the sample standard deviation for x and y.
\begin{gather}
    s_x = \sqrt{\frac{\sum (x - \bar{x})}{n - 1}}\\
    s_y = \sqrt{\frac{\sum (y - \bar{y})}{n - 1}}
\end{gather}

Then, we need to calculate the covariance of $x$ and $y$:
\begin{gather}
    s_{xy} = \frac{\sum (x_i - \bar{x}) * (y_i - \bar{y})}{n - 1}
\end{gather}

Finally, the correlation coefficient is defined by this:
\begin{gather}
    r = \frac{s_{xy}}{s_x * s_y}
\end{gather}

\begin{figure}[h!]
    \centering
    \includegraphics[width=0.7\textwidth]{pictures/4.2.2.png}
\end{figure}