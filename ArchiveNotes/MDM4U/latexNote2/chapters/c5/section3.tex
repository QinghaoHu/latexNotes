\section{Binnomial Distributions}
\subsection{Example}
\begin{example}
    A probability experiment involves rolling a regular 6-sided die three times. Let the random variable X represent all
possible number of 4’s rolled. Complete the tree diagram, fill in the blanks, then show the distribution of X in a
table.

\begin{figure}[!h]
    \centering
    \includegraphics[width=1.0\textwidth]{pictures/5.3.1.png}
\end{figure}

This question is a typical example of \textbf{Binominal Distribution}
\end{example}

\noindent\hrulefill

\newpage
\subsection{Definitions}
% \noindent\hrulefill
\begin{definition}
    The distribution of a discrete random variable $X$ is a binomial probability distribution if:
    \begin{enumerate}
        \item The probability experiment is repeated a FIXED number of times (In the example, $n = 3$)
        \item The outcome of each trial can be CATEGORIZED into successful or failed outcomes
        \item The random varialbe $X$ represents the number of \textbf{successes}
        \item Each trial of the experiment is \textbf{Independent}
    \end{enumerate}
\end{definition}

\noindent\hrulefill

\begin{theorem}
    If the distribution of $X$ is a binominal probability, $N$ is the number of independent trials of the experiment and $p$ is the probability of 
    success of each independent trail, then:
    \begin{gather}
        P(x) = \binom{n}{x}p^x(1-p)^{n - x}\\
        E(x) = np\\
        \sigma = \sqrt{np(1 - p)}
    \end{gather}
\end{theorem}

\begin{remark}
    The binominal distribution is only work for Discrete random variable!
\end{remark}