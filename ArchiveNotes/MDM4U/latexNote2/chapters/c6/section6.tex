\section{Confidence Intervals}
\begin{theorem}
    Repeated sampling from a normally distributed population produces a normally distributed sample means. Hence, 
    the probability of observing a single sample mean, $\bar{x}$, within $z\sigma_{\bar{x}}$
    of $\mu_{\bar{x}}$ is $1-\alpha$\\

    As a result, we can get: 
    \begin{equation*}
        P(\mu_{\bar{x}} - z\sigma_{\bar{x}} < x < \mu_{\bar{x}} + z\sigma_{\bar{x}}) = 1 - \alpha
    \end{equation*}

    After rearrange, for a significance level of $\alpha$, we can get this:
    \begin{equation*}
        \bar{x} - z\frac{\sigma}{\sqrt{n}} < \mu_{\bar{x}} < \bar{x} + z\frac{\sigma}{\sqrt{n}}
    \end{equation*}

    Therefore, the boundaries for the interval estimate is $\bar{x} \pm z\frac{\sigma}{\sqrt{n}}$
\end{theorem}

\begin{definition}
    [Porportion] A ratio of the population that shares the same characteristic. 
\end{definition}

\begin{theorem}
    If one wants to find the proportion of a population that have a particular characteristic, then
    \begin{equation*}
        \mu_{\bar{p}} = \mu \indent \sigma_{\bar{p}} = \sqrt{\frac{\hat{p}(1 - \hat{p})}{n}}
    \end{equation*}
    \begin{center}
        $p$ and $\hat{p}$ are the proportion of the population and a sample, respectively that have the characteristic.\\
        $\mu_{\hat{p}}$ and $\sigma_{\hat{p}}$ are the mean and the standard deviation of the distribution of the sampling proportions. 
    \end{center}

    As well, 
    \begin{equation*}
        P(\mu_{\hat{p}} - z\sqrt{\frac{\hat{p}(1 - \hat{p})}{n}} < p < \mu_{\hat{p}} + z\sqrt{\frac{\hat{p}(1 - \hat{p})}{n}})
    \end{equation*}
    The boundaries for the interval estimate is defined as:
    \begin{equation*}
        \hat{p} \pm \sqrt{\frac{\hat{p}(1 - \hat{p})}{n}}
    \end{equation*}
\end{theorem}