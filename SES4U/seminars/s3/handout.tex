\documentclass[1pt]{article}
\usepackage[utf8]{inputenc}
\usepackage{geometry}
\usepackage{setspace}
\usepackage{enumitem}
\usepackage{ulem} % for \uline

\geometry{margin=1in}
\setstretch{1.2}

% A command to replace red text with blanks
\newcommand{\fillin}[1]{\uline{\hspace{#1}}}

\begin{document}

\section*{Oil \& Gas Industry}

\subsection*{Formation}
\subsubsection*{The Beginning: Dead marine life}
\begin{itemize}
    \item Only a tiny amount (of died organisms) sinks into deep, \fillin{5cm} water
\end{itemize}

\subsubsection*{Source Rock Formation}
\begin{itemize}
    \item With little oxygen and no tidal currents, it does not fully \fillin{3cm}.
    \item This turns into a black, organic-rich rock called \fillin{3cm}
\end{itemize}

\subsubsection*{Diagenesis}
\begin{itemize}
    \item \fillin{3cm} is the process that turns loose sediments into sedimentary rock. 
    \item It also alters buried organic matter to form \fillin{3cm}
\end{itemize}

\subsubsection*{CATEGENESIS}
\begin{itemize}
    \item A process where heat and pressure transform kerogen into \fillin{3cm}. 
    \item Geothermal \fillin{3cm} break down kerogen as it is buried deep in the Earth.
    \item Kerogen turns into oil: 60 to 130\textdegree C /2000m - 3800m depth
    \item Kerogen turns into natural gas: \>130\textdegree C/ \>4000m depth
\end{itemize}

\subsubsection*{Migration \& Accumulation}
\begin{itemize}
    \item Hydrocarbons move upward through \fillin{2cm} rocks because they are less dense
    \item Accumulates when oil and gas hit the cap rock, which is hard and \fillin{3cm}
\end{itemize}

\subsection*{Identification of Oil and Gas}
\subsubsection*{Geological indicators}
This includes identifying
\begin{itemize}
    \item \fillin{3cm} Rocks, \fillin{3cm} Rocks,\fillin{3cm} rocks and Structural Traps (anticlines, domes, salt domes, faults)
\end{itemize}

\subsubsection*{SEISMIC SURVEY}
\begin{itemize}
   \item Emmits \fillin{4cm} into the ground
   \item Records the \fillin{3cm} that are bounced back from subsurface rock layers
\end{itemize}

\subsection*{Oil Location}
\subsubsection*{Western Canada Sedimentary Basin}
\begin{itemize}
    \item The \fillin{3cm} has perfect conditions: thick \fillin{4cm}layers, organic-rich source rocks and traps that hold hydrocarbon
\end{itemize}

\subsubsection*{Alberta Oil Sands}
\begin{itemize}
    \item Only about 3 to 5 \% of all oil deposits are \fillin{2cm} enough to the surface to be mined.
    \item In situ(in place) used to mile the rest: stream is injected \fillin{3cm} to metl the oil, which is then pumped to the surface
\end{itemize}

\subsection*{The age of primitive extraction}
\subsubsection*{Process of modern oil extraction}
\begin{itemize}
    \item Use technology we discussed before to find the \fillin{3cm}
    \item A drilling rig bores a \fillin{3cm} down to the reservoir!
    \item Natural pressure may push oil upward at first, but \fillin{4cm} or other pumping system are later needed.
\end{itemize}

\subsubsection*{Early Extraction}
\begin{itemize}
    \item Cable-Tool drilling (drop \fillin{3cm})
    \item Inefficient, \fillin{3cm} wells (less than 70m)
\end{itemize}

\subsubsection*{Early Refining}
\begin{itemize}
    \item Backyard-level \fillin{4cm} (like moonshine)
    \item Impure Kerosene - easily \fillin{3cm}
\end{itemize}

\subsubsection*{Improved Extraction}
\begin{itemize}
    \item Built the world's first large-scale \fillin{3cm} network
    \item Used \fillin{3cm} for safe transport of oil
\end{itemize}

\subsubsection*{Improved Extraction}
\begin{itemize}
    \item Introduced \fillin{6cm} distillation.
    \item Developed \fillin{6cm} treatment
\end{itemize}

\subsubsection*{Temperature-Controlled Distillation}
\begin{itemize}
    \item The vapor enters a tall tower where the bottom is very \fillin{3cm} and the top is much cooler.
    \item Light molecules rise \fillin{3cm}, and heavy molecules stay \fillin{3cm}
    \item When the vapor reaches a place cooler than its bolling point, it turns back into \fillin{3cm}
\end{itemize}

\subsubsection*{Acid Treatment}
\begin{itemize}
    \item Mixed kerosene with \fillin{5cm}($H_2SO_4$)
    \item Sludge settles at the bottom
    \item \fillin{3cm} Kerosene remains on top
\end{itemize}

\subsubsection*{Acid Treatment}
\begin{itemize}
    \item Rockefeller add \fillin{6cm} ($NaOH$) to neutralize leftover acid and remove remaining
\end{itemize}

\subsection*{Monopoly}
\begin{itemize}
    \item With technology, Rockefeller could sell cheaper kerosene while still earning high profits
    \item Competitors could not match his efficiency, so he \fillin{3cm} them out at fair prices before they went bankrupt
\end{itemize}

\subsection*{Near-National Monoply}
\begin{itemize}
    \item Controlled about \fillin{2cm}\% of U.S. refining capacity by 1880s
    \item This made Standard Oil the first nationwide \fillin{6cm} in the world
\end{itemize}

\subsection*{Controled the Politics}
\begin{itemize}
    \item He donoted money to \fillin{3cm} political parties, ensuring that whichever side won would support his business interests
    \item Through his huge economic power, Standard Oil became a "\fillin{5cm}" in U.S. politics
\end{itemize}

\end{document}
