\section{Seminar Testable Question}
\subsection{Formation of crust and continents}
\textcolor{blue}{1. What is planetary differentiation and explain how it contributed to the creation of the Earth's layers?}
\begin{itemize}
    \item Planetary differentiation is a process by which a planet organizes itself based on density. 
    \item Denser materials sank to the core while lighter materials floated upward, forming the different layers of the Earth. 
\end{itemize}

\textcolor{blue}{What are two important characteristics of a craton?}
\begin{itemize}
    \item It's the oldest and most stable part of a continent
    \item Makes up around 10\% of Earth's landmass
\end{itemize}

\subsection{Relative and absolute age dating}
\subsubsection{Define Absolute age dating}
It determines the exact age of a rock or fossil in years, often using radioactive decay. 

\subsubsection{State the name of any three of five principles of relative age dating and explain}
\begin{itemize}
    \item \textbf{Original Horizontality}: Sedimentary rocks are generally deposited in horizontal layers. 
    \item \textbf{Cross Cutting Relationship}: An intrusion or fault line is younger than the rock it cuts across. 
    \item \textbf{Inclusions}: Inclusions in a rock layer must be older than the rock layer that contains them. 
\end{itemize}

\subsection{Fossil Dating}
\subsubsection{How are fossils found on mountaintops, where there are few sediments available?}
Tectonic plates push areas of rock together to expose fossils. 

\subsubsection{For a fossil to be considered an index fossil, it must fulfill five criteria. What are three of those criteria}
\begin{itemize}
    \item Unique
    \item Globally widespread
    \item Abundant
\end{itemize}

\subsubsection{Difference between a mould and cast fossil}
\begin{itemize}
    \item \textbf{Mould}: The space left by an organism after it decays. 
    \item \textbf{Cast}: when a mould is filled with sediments. 
\end{itemize}

\subsection{The Geologic Time Scale}
\subsubsection{What is the relationship between Eons and eras?}
Eons are the longest divisions of geological time, and are divided into eons. 

\subsubsection{How are eons divided?}
Eons are divided by major global changes

\subsubsection{How do rock layers act as a "Ruler" for Earth's history?}
\begin{itemize}
    \item Older rock layers usually lie beneath younger layers, creating a time order
    \item Earth rock layer can provide evidence of Earth's previous environment. 
\end{itemize}

\subsection{Famous Geologists}
\subsubsection{What is the principle of Uniformitarianism}
Present-day geological processess explain past geological features. 

\subsubsection{Alfred Wegener's theory of Continental Drift was initially rejected by the scientific community because}: 
HIs maps showing continental fit were geographically inaccurate

\subsubsection{Explain how Mary Anning's fossil discoveries provided critical evidence that supported Charies Lyell's concept of Deep Time}
\begin{itemize}
    \item He discoveries of complete skeletons of creatures which had no living equivalents provided proof of extinction
    \item This evidence of extinction required vast periods of time. 
\end{itemize}

\subsection{Formation of the atmosphere and oceans}
\subsubsection{Explain how banded iron formations form}
\begin{itemize}
    \item Oxygen reacts with dissolved iron in water, and settles on the ocean floor. 
\end{itemize}

\subsection{Early Life on Earth}
\subsubsection{What were the three main gases thought have been involved in the Primordial Soup?}
\begin{itemize}
    \item Methane, Hydrogen, Ammonia
\end{itemize}

\subsubsection{Introductoin of amino acids on Earth}
Space
\begin{itemize}
    \item amino acids found in asteroids and comets that survived crash on Earth
    \item Asteroids and comets have been found on Earth with amino acids
    \item Difficult to test due to technical limitations
\end{itemize}

\subsection{Paleozoic era}
\subsubsection{Which factor best explains why many animals appear to "suddenly" appear in the fossil record during the Cambrian Period}
\begin{itemize}
    \item Many animals evolved hard body parts the fosslize easily. 
\end{itemize}

\subsubsection{Three main differences between pre-cambrian life and life during the cambrian period. }
Pre-Cambrian life: 
\begin{itemize}
    \item Unicellular organisms
    \item Organisms were mostly disc/bolb shaped
\end{itemize}

Cambrian life:
\begin{itemize}
    \item Developed new body-plans that allowed for increased movement
    \item More diversity of organisms
    \item Predator-prey relationships. 
\end{itemize}

\subsection{The Mesozoic Era}
\subsubsection{How did seaways form as a result of Pangea breaking apart?}
\begin{itemize}
    \item The heat beneath Pangaea caused the continent to expand, and its brittle lithosphere began to crack. 
    \item Some of the large cracks, called rifts, gradually widened and the landmass began spreading apart. 
    \item Ocean flooded the rift valleys to form seaways, and large blocks of crust collapsed to form deep valleys. 
\end{itemize}

\subsubsection{What are the names of the two major landmasses that supercontinent Pangea broke into?}
Laurasia and Gondwana

\subsection{Constructing dinosaurs from fossils}
\subsubsection{Why is it so difficult to accurately reconstruct a dinosaur?}
\begin{itemize}
    \item Most of a dinosaur was composed of soft tissues and fats, which do not fossilize
    \item Scientists can only predict how they looked baesd on the fossilzed skeleton and related present-day animals
\end{itemize}

\subsubsection{What are the desscendants of dinosaurs?}
\begin{itemize}
    \item Birds
\end{itemize}

\subsubsection{What is the rarest form of fossils}
\begin{itemize}
    \item Soft Tissues
\end{itemize}

\subsection{The Cenozoic Era}
\subsubsection{What did the Cenozoic Era begin?}
\begin{itemize}
    \item 66 Milion Years ago
\end{itemize}

\subsubsection{Which factor played the most important rule in maintaining the extreme warmth of the Early Cenozoic Era?}
High levels of atmospheric carbon dioxide caused by intense volcanic activity

\subsubsection{What kind of habitat became more common as forests shrank}
Grasslands and open plains

\subsubsection{Why is the Paleocene-Eocene Thermal Maximum important?}
Because it shows how adding greenhouse gases can quickly warm Earth's climate. 