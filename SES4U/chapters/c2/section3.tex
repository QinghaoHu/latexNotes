\section{Kepler's Law of Planetory Motion}

\begin{figure}[h!]
    \centering
    \includegraphics[width=0.5\textwidth]{pictures/Kepler'sEllipse1.jpg}
\end{figure}

The sun is at one focus\\

There are two points in this diagram:
\begin{itemize}
    \item The point closet to the Sun is called \textbf{\textit{Perihelion}}. \textit{Peri} means \textit{near} in the Latin.
    \item The point farthest from the Sun is called \textbf{\textit{Aphelion}}. \textit{Ape} means \textit{far} in the Latin.
\end{itemize}

\subsection{Kepler's first law}
\begin{definition}
    Planet's orbit in ellipses with the Sun at one focus
\end{definition}

Ellipses can be classified based on their \textbf{eccentricity}

\[
    e = \frac{c}{a}
\]

\begin{center}
    e = Eccentricity \\
    c = Distance from centre to a focus (in m(or Au))\\
    a = Length of semi-major axis (in m(or Au))
\end{center}

The eccentricity of Earth's orbit is 0.02.\\

The most eccentric planetary orbit in our solar system is \textit{Mercury}, which has a eccentricity of 0.2.\\

Coments tend to the the largest eccentricity very close to 1.\\

Here I want to discuss about the meaning of eccentricity:
\begin{itemize}
    \item If an ellipse has an eccentricity of \textbf{0}, the object is orbit its sun in a \textbf{perfect circle}.
    \item If an ellipse has an eccentricity of \textbf{1}, the object is not in an \textbf{orbit}. 
\end{itemize}

\subsection{Kepler's second law}
\begin{definition}
    A line segment joining a planet and the sun \textbf{sweeps out equal areas in equal amount time}
\end{definition}

By the second law, we can make a conclusion. A planet moves fastest with it is at the \textbf{perihelion} and slowest when it is at the \textbf{aphelion}.

\subsection{Kepler's Third Law}
\begin{definition}
    The square of the orbital period of a planet directly proportional to the cube of the length 
    of the semi-major axis of its orbit
\end{definition}

\[
    p^2 = a^3
\]
\begin{center}
    $p$ = orbital period in (years)\\
    $a$ = Length of semi-major axis (in Au)
\end{center}

\begin{figure}[h!]
    \centering
    \includegraphics[width=0.7\textwidth]{pictures/Kepler'sLaw3.png}
    \caption{This is a log graph}
\end{figure}

The semi-major axis of an orbit is sometimes referred to as the \textbf{average} distance from the sun.