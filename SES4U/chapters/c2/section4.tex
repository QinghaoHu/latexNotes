\newpage
\section{Sun-Earth-Moon System}
\subsection{Earth's Rotationsc}
\subsubsection{Foucault Pendulum}
The direction that a pendulum swings appears to change as the Earth rotates under it.\\

One rotation of the earth takes \textbf{23hrs 56 minutes}\\

However, we define one day as the time it takes the \textbf{Sun} to return to the same position in the sky. This time takes \textbf{24 hrs}, longer than one rotation due to Earth's motion around the sun. 

\subsection{Earth's Orbit}
The \textbf{Ecliptic plane} is an imaginary plane on which Earth's orbit lies. The Earth is tilted by $23.5$ degree in relation to this plane. One orbit the Earth takes 365.25 day.\\

The tilt changes the Sun's position in hte sky throughout the year and accounts for the change of seasons. \\

The tropics is defined as the region between 23.5 degree north and 23.5 degree south latitude. Outside of the tropics, it is impossible for the sun to appear at exactly 90 degree. 

\subsection{Tide}
The tides happen due to the moon's \textbf{gravational pull} on the Earth.\\

\textbf{Spring} tides happen when the Sun and moon align. The tides are \textbf{higher} than normal. \\

\textbf{Neap} tides happen when the Sun and moon are perpendicular. These tides are \textbf{lower} than normal. \\

One cycle of the tide happens every \textbf{12} cycles. 

\subsection{Eclipses}
\begin{definition}
    Eclipses occur when one object passes through the \textbf{shadow} of another object. 
\end{definition}
A full shadowm is called the \textbf{umbra} and a partial shadow is called the \textbf{penumbra}.

\begin{figure}[h!]
    \centering
    \includegraphics[width=0.4\textwidth]{pictures/Eclipses.png}
\end{figure}

\subsubsection{LUNAR ECLIPSE}
The moon is in the \textbf{Earth's shadow}.

\subsection{SOLAR ECLIPSE}
The Earth is in the \textbf{Moon's shadow}.
