\section{Mineral Identification}
\subsubsection{Density}
Steps to determine the density of a mineral:
\begin{enumerate}
    \item Use a balance to weight its \underline{mass} (in $kg$)
    \item Fill a graduate cylinder with water, throw the mineral in to see how much the water rises in a graduate cylinder. (in $mL$)
    \item Calculate $\frac{\text{mass}}{\text{volume}}$
\end{enumerate}

\subsubsection{Crystal}
Minerals are amde of atoms in a repeating pattern and often from \underline{crystal}\\

Possible options:
\begin{itemize}
    \item Hexagon
    \item Cube
    \item Pyramid
    \item Rectangle
\end{itemize}

\subsubsection{Luster}
\underline{Luster} is the way the mineral's surface reflect light. 

\subsubsection{Hardness}
\underline{Hardness} is a measure of how easily a mineral can be scratched. It is measured on a scale of 1 to 10 
 called Mohs scale. 

\subsubsection{Streak}
    The \underline{Streak} is the color of a material's powder. 
\begin{remark}
    If the mineral is harder than the streak plate, it won't leave a streak.
\end{remark}

\subsubsection{Acid test}
Some minerals cause hydrochloric acid to bubble and fizz. 


