\chapter{Unit 2}
\section{Kepler's Law}

\subsection{Kepler's first law}
\begin{definition}
    Planet's orbit in ellipses with the Sun at one focus
\end{definition}

Ellipses can be classified based on their \textbf{eccentricity}

\[
    e = \frac{c}{a}
\]

\begin{center}
    e = Eccentricity \\
    c = Distance from centre to a focus (in m(or Au))\\
    a = Length of semi-major axis (in m(or Au))
\end{center}

\subsection{Kepler's second law}
\begin{definition}
    A line segment joining a planet and the sun \textbf{sweeps out equal areas in equal amount time}
\end{definition}

\subsection{Kepler's Third Law}
\begin{definition}
    The square of the orbital period of a planet directly proportional to the cube of the length 
    of the semi-major axis of its orbit
\end{definition}

\[
    p^2 = a^3
\]
\begin{center}
    $p$ = orbital period in (years)\\
    $a$ Length of semi-major axis (in Au)
\end{center}

The semi-major axis of an orbit is sometimes referred to as the \textbf{average} distance from the sun