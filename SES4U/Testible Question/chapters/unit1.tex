\chapter{Astronomy}
\section{The Sun}
\subsection{Question 1}
    Which of the following explains why the sun's corona is much hotter than photosphere:\\
    \red{Energy from the Sun's magnetic field heats the outer layer}

\subsection{Question 2}
Which process is reponsible for the sun's energy production:\\
\red{Nuclear fusion of hydrogen in to helium}

\subsection{Question 3}
Why do sunspots appear darker than the surrounding region of the sun's surface and what cause them:
\begin{itemize}
    \item \red{Sunspots look darker because they are cooler}
    \item \red{Strong magnetic fields in this region block the normal flow of hot plasma}
\end{itemize}

\section{Stellar Evolution}
\subsection{Question 1}
Which of the following is not affected directly by a star's mass?:\\
\red{Chemical Composition}

\subsection{Question 2}
\begin{itemize}
    \item \red{Nuclear fusion at the core creates pressure that pushes outward on a star, while the star's mass causes gravity to pull inwards on the star}
    \item \red{These two forces balanced out}
\end{itemize}

\subsection{Question 3}
If the cores of supergiant stars can only fuse elements up to iron, why do heavier elements exist in the universe?
\red{During supernova}

\section{Black Holes}
\subsection{Question 1}
What did Einstein say about gravity in his general theory of relativity?\\
\red{Gravity is a curvature in space-time caused by its mass}

\subsection{Question 2}
Why blackhole cannot be seen with the naked eye, scientists can still find a black hole's size and location:\\
\red{Through some methods:}
\begin{enumerate}
    \item Observe a black hole's effect on objects surrounding it, including its gravitational pull and how it impacts a star's orbit
    \item Measure the x-ray emitted from the hot gass inside the blackhole
    \item Measure the gravitational waves given off when toe black holes collide
\end{enumerate}

\section{The Milky Way Galaxy}
\subsection{Question 1}
Where is our solar system located in the Milky Way Galaxy? \\
\red{In a smaller spiral arm called the Orion Arm}

\subsection{Question 2}
Why do astronomers use infrared and radio telescope to study the Milky Way?\\
\red{Because they can see through all the dust and gas that blocks visible light}

\subsection{Question 3}
Why is the solar system a "good spot"?
\begin{itemize}
    \item \red{We live in a quiet and stable part of the Galaxy}
    \item \red{Just enough gas and dust to keep stars forming, but not too much we'd be hit by radiation or frequent large explosions}
\end{itemize}

\section{Other Galaxy}
\subsection{Question 1}
Why are galaxy clusters considered to be great laboratories?
\begin{itemize}
    \item They are ideal for studying chemical composition and history of nucleosynthesis.
    \item They retain imprints needed for studying histories of galaxy formations.
    \item They are an excellent way to study dark matter.
\end{itemize}

\subsection{Question 2}
Differences between spiral and elliptical galaxies
\begin{itemize}
    \item Spiral galaxy have spiral arms and are formed in to diskes. Ellipticals lack any arms and are not flattened into diskesk
    \item Spiral galaxies have young and old stars, while elliptical galaxies only have old stars
    \item Compared to spiral, elliptical galaxies contain less gas and dust
    \item Ellipticals have insufficient gas needed to create new stars
\end{itemize}

\section{The Big Bang}
\subsection{Possible shapes of the Universe}
\red{Open, Flat and Closed}

\subsection{Important evidences about Big Bang}
\begin{itemize}
    \item Red Shift and Hubble's Law
    \item Cosmic microwave: start from a singularity and dense and hot point. Radiation from the early Universe
    \item Light element abundances: The abundance of those elements are the same
\end{itemize}

\section{How will our universe end}
\subsection{Why no new stars form during big freeze}
Gas clouds are very dispersed as the Universe expands, preventing stars from forming

\subsection{Final state of Big Crunch}
Become back to a hot point.

\subsection{Similarity}
All three scenarios are determined by the balance between the univers's expansion and gravity.

\subsection{Three ideas}
\begin{itemize}
    \item Big Freeze: Continus to expand, becoming colder and more spread out untier all activity stop
    \item the big crunch, reverse
    \item the big rip: The universe's accelerating expansion becomes so strong that it tears everything apart, even fundamental particles.
\end{itemize}

\section{Multiverse Universe}
\subsection{Many worlds theory}
Every outcome of an event branches into its own separate Universe at a singular point in time

\subsection{Main Idea}
There is an infinite amount of universes outside our own universe, all with different variations

\section{James Webb Space telescope}
Mainly infrared

\subsection{Instruments}
NIRSPEC: microshutter array, only performs spectroscopy and does not actually take pictures spectrocopy 
NIRCAM: take picture of near infrared range camera
NIRISS: Black out distracting light
MIRI. mid infrared image, need to be cooler mid-infread instrument

\section{Dark Matter}
It can bend light with its gravity\\
A mysterious force causing the accelerated expansion of the universe

\subsection{How scientist discover dark energy}
\begin{itemize}
    \item distant supernova were moving away faster than expected
    \item Unknown force was pushing space apart and counteracting gravity
\end{itemize}

\section{Astrobiology}
\begin{redblock}
    \textbf{Goldilocks zone:} The region around a star where the conditions are just right for liquid water to exist
\end{redblock}

\subsection{extremophiles}
How life can surivie through harsh environments.\\
Suggest that life could still exist on other planets or moon without requiring earth-like conditions\\

\section{exoplanets}
\subsection{Radial velocity method}
By observing the "wobble" of a star caused by the gravational pull of an orbiting planet\\
By measuring these spectral shifts, astronomers can infer the presence of the exoplanet\\