\documentclass[10pt]{beamer}

\usepackage{subfig}
\usepackage{amssymb,amsmath,mathtools}
\usepackage{amsfonts,booktabs}
\usepackage{lmodern,textcomp}
\usepackage{color}
\usepackage{tikz}
\usepackage[latin1]{inputenc}
\usepackage{natbib}
\usepackage{multicol}
\usepackage{graphicx}
\usepackage{caption}

\usetheme{Madrid}
\usecolortheme{default}

\title[SES4U]
{History of space science to 1700s}
\subtitle{Myths + Geocentric system + Heliocentric system}
\author{Qinghao Hu}


\begin{document}

\frame{\titlepage}

\begin{frame}
  \frametitle{Table of Contents}
  \tableofcontents
\end{frame}

\section{Expectation}
\begin{frame}
  \frametitle{Expectation}
  \begin{enumerate}
    \item Describe how early cultures created myths to explain the universe
    \item Explain the differences between the "Geocentric System" and "Heliocentric system"
    \item Describe how was humans' understanding about the Universe develop in chronological order
    \item Describe how did each scientist contribute to "Geocentric System" and "Heliocentric System"
  \end{enumerate}
\end{frame}

\section{Introduction}
\begin{frame}
  \frametitle{Myth}
\end{frame}

\end{document}

% `\begin{frame}{Image with Description Below}
% \begin{center}
%     \includegraphics[width=0.7\textwidth]{example-image-b.jpg}
%     \captionof{figure}{A descriptive caption for the image}

%     \vspace{0.5cm} % space between image and text

%     This is a short description or explanation related to the image.
%     It can be one or two lines to avoid clutter.
% \end{center}
% \end{frame}
'

% \begin{frame}{Image with Description Below}
% \begin{center}
%     \includegraphics[width=0.7\textwidth]{example-image-b.jpg}
%     \captionof{figure}{A descriptive caption for the image}

%     \vspace{0.5cm} % space between image and text

%     This is a short description or explanation related to the image.
%     It can be one or two lines to avoid clutter.
% \end{center}
% \end{frame}