
\documentclass[11pt]{report}

% --- Packages ---
\usepackage[utf8]{inputenc}
\usepackage{amsmath, amssymb, amsthm}
\usepackage{geometry}
\usepackage{fancyhdr}
\usepackage{graphicx}
\usepackage{tikz}
\usepackage{enumitem}
\usepackage{hyperref}
\usepackage{xcolor}
\usepackage{indentfirst}

% --- Page Setup ---
\geometry{margin=1in}
\pagestyle{fancy}
\fancyhf{}
\rhead{Qinghao Hu}
\lhead{SPH4U}
\cfoot{\thepage}

% --- Theorem Environments ---
\newtheorem{theorem}{Theorem}[section]
\newtheorem{definition}[theorem]{Definition}
\newtheorem{lemma}[theorem]{Lemma}
\newtheorem{proposition}[theorem]{Proposition}
\newtheorem{corollary}[theorem]{Corollary}
\theoremstyle{remark}
\newtheorem*{remark}{Remark}
\newtheorem*{example}{Example}

% --- Custom Commands ---
\newcommand{\R}{\mathbb{R}}
\newcommand{\N}{\mathbb{N}}
\newcommand{\Z}{\mathbb{Z}}
\newcommand{\Q}{\mathbb{Q}}
\newcommand{\C}{\mathbb{C}}
\newcommand{\ds}{\displaystyle}

\newcommand{\mypic}[3]{
    \begin{figure}[h!]
        \centering
        \includegraphics[width=#3\textwidth]{#1}
        \caption{#2}
    \end{figure}
}

% --- Title ---
\title{\textbf{Grade 12 Chemistry} \\ \large SCH4U}
\author{Qinghao Hu}
\date{\today}

\begin{document}

\maketitle
\newpage
\tableofcontents
\newpage

\chapter{Unit 1}
\section{Intro to Quantum}
\subsection{Gold Foil Experiment}
\begin{itemize}
    \item Rutherford's concluded that the atom is mostly empty space with a very dense, positively charged nucleus
    \item Electrons rotate around the nucleus like planets around the sun
    \item Thomson's Model of atom: There are only -e in an atom
    \item By Dalton, elements consist of indivisible small particles (atoms)
\end{itemize}

\subsection{Niels Bohr}
He suggested that there are stable orbits around the nucleus, where electrons can orbit indefinitely without losing energy

\begin{center}
    \begin{itemize}
        \item When Given energy, electrons can "jump" to a higher energy level. 
        \item Enery levels are discrete, a specific amount of energy is required
        \item So the atom will release/absorb a specific wave length, in other words, a specific color of light
    \end{itemize}    
\end{center}

\subsection{Photoelectric Effect)}
Ejection of electrons did not depend upon light intensity, but rather its frequency

Most importantly, light can be either wave or elections.

\subsection{Heisenberge Uncertainty Principle}
Quantum is probability, you do not know the exact position of Quantum. \\
Balling ball example by Mr. Cheung.

\subsection{Electron Cloud}
Shrodinger's Equation describes the behaviour of an electron in 3 dimensional space. \\
You need four parameters:\\ 
\begin{equation}
    (n, l, m_{l}, m_{s})
\end{equation}
\begin{center}
    $n$ describes the energy levels\\
    $l$ describes which kind of house\\ 
    $m_{l}$ describes which room \\ 
    $m_{s}$ describes which people \\
\end{center}


\end{document}

% Referenc
% \section{Introduction}
% Write your introduction or overview of the day's topic here.
%
% \section{Main Concepts}
% \begin{definition}
% A function \( f: A \to B \) is said to be \textbf{injective} if for every \( a_1, a_2 \in A \), \( f(a_1) = f(a_2) \Rightarrow a_1 = a_2 \).
% \end{definition}
%
% \begin{theorem}
% Let \( f: \R \to \R \) be a continuous and strictly increasing function. Then \( f \) is injective.
% \end{theorem}
%
% \begin{proof}
% Assume \( f(a) = f(b) \). Since \( f \) is strictly increasing, if \( a < b \), then \( f(a) < f(b) \), contradiction. So \( a = b \).
% \end{proof}
%
% \section{Examples}
% \begin{example}
% Let \( f(x) = x^2 \). Then \( f \) is not injective on \( \R \), but it is injective on \( [0, \infty) \).
% \end{example}
%
% \section{Diagrams}
% \begin{center}
% \begin{tikzpicture}[scale=1]
% \draw[->] (-2,0) -- (2,0) node[right] {$x$};
% \draw[->] (0,-1) -- (0,4) node[above] {$y$};
% \draw[domain=-1.5:1.5, smooth, variable=\x, blue, thick] 
%     plot ({\x}, {\x*\x});
% \node at (1.2,3.2) {$y = x^2$};
% \end{tikzpicture}
% \end{center}
%
% \section{Summary}
% Summarize what you learned today.
